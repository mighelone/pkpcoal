% Generated by Sphinx.
\def\sphinxdocclass{report}
\documentclass[letterpaper,10pt,english]{sphinxmanual}
\usepackage[utf8]{inputenc}
\DeclareUnicodeCharacter{00A0}{\nobreakspace}
\usepackage[T1]{fontenc}
\usepackage{babel}
\usepackage{times}
\usepackage[Bjarne]{fncychap}
\usepackage{longtable}
\usepackage{sphinx}
\usepackage{multirow}


\title{PKP Documentation}
\date{August 12, 2012}
\release{1.0.0}
\author{Martin Pollack}
\newcommand{\sphinxlogo}{}
\renewcommand{\releasename}{Release}
\makeindex

\makeatletter
\def\PYG@reset{\let\PYG@it=\relax \let\PYG@bf=\relax%
    \let\PYG@ul=\relax \let\PYG@tc=\relax%
    \let\PYG@bc=\relax \let\PYG@ff=\relax}
\def\PYG@tok#1{\csname PYG@tok@#1\endcsname}
\def\PYG@toks#1+{\ifx\relax#1\empty\else%
    \PYG@tok{#1}\expandafter\PYG@toks\fi}
\def\PYG@do#1{\PYG@bc{\PYG@tc{\PYG@ul{%
    \PYG@it{\PYG@bf{\PYG@ff{#1}}}}}}}
\def\PYG#1#2{\PYG@reset\PYG@toks#1+\relax+\PYG@do{#2}}

\def\PYG@tok@gd{\def\PYG@tc##1{\textcolor[rgb]{0.63,0.00,0.00}{##1}}}
\def\PYG@tok@gu{\let\PYG@bf=\textbf\def\PYG@tc##1{\textcolor[rgb]{0.50,0.00,0.50}{##1}}}
\def\PYG@tok@gt{\def\PYG@tc##1{\textcolor[rgb]{0.00,0.25,0.82}{##1}}}
\def\PYG@tok@gs{\let\PYG@bf=\textbf}
\def\PYG@tok@gr{\def\PYG@tc##1{\textcolor[rgb]{1.00,0.00,0.00}{##1}}}
\def\PYG@tok@cm{\let\PYG@it=\textit\def\PYG@tc##1{\textcolor[rgb]{0.25,0.50,0.56}{##1}}}
\def\PYG@tok@vg{\def\PYG@tc##1{\textcolor[rgb]{0.73,0.38,0.84}{##1}}}
\def\PYG@tok@m{\def\PYG@tc##1{\textcolor[rgb]{0.13,0.50,0.31}{##1}}}
\def\PYG@tok@mh{\def\PYG@tc##1{\textcolor[rgb]{0.13,0.50,0.31}{##1}}}
\def\PYG@tok@cs{\def\PYG@tc##1{\textcolor[rgb]{0.25,0.50,0.56}{##1}}\def\PYG@bc##1{\colorbox[rgb]{1.00,0.94,0.94}{##1}}}
\def\PYG@tok@ge{\let\PYG@it=\textit}
\def\PYG@tok@vc{\def\PYG@tc##1{\textcolor[rgb]{0.73,0.38,0.84}{##1}}}
\def\PYG@tok@il{\def\PYG@tc##1{\textcolor[rgb]{0.13,0.50,0.31}{##1}}}
\def\PYG@tok@go{\def\PYG@tc##1{\textcolor[rgb]{0.19,0.19,0.19}{##1}}}
\def\PYG@tok@cp{\def\PYG@tc##1{\textcolor[rgb]{0.00,0.44,0.13}{##1}}}
\def\PYG@tok@gi{\def\PYG@tc##1{\textcolor[rgb]{0.00,0.63,0.00}{##1}}}
\def\PYG@tok@gh{\let\PYG@bf=\textbf\def\PYG@tc##1{\textcolor[rgb]{0.00,0.00,0.50}{##1}}}
\def\PYG@tok@ni{\let\PYG@bf=\textbf\def\PYG@tc##1{\textcolor[rgb]{0.84,0.33,0.22}{##1}}}
\def\PYG@tok@nl{\let\PYG@bf=\textbf\def\PYG@tc##1{\textcolor[rgb]{0.00,0.13,0.44}{##1}}}
\def\PYG@tok@nn{\let\PYG@bf=\textbf\def\PYG@tc##1{\textcolor[rgb]{0.05,0.52,0.71}{##1}}}
\def\PYG@tok@no{\def\PYG@tc##1{\textcolor[rgb]{0.38,0.68,0.84}{##1}}}
\def\PYG@tok@na{\def\PYG@tc##1{\textcolor[rgb]{0.25,0.44,0.63}{##1}}}
\def\PYG@tok@nb{\def\PYG@tc##1{\textcolor[rgb]{0.00,0.44,0.13}{##1}}}
\def\PYG@tok@nc{\let\PYG@bf=\textbf\def\PYG@tc##1{\textcolor[rgb]{0.05,0.52,0.71}{##1}}}
\def\PYG@tok@nd{\let\PYG@bf=\textbf\def\PYG@tc##1{\textcolor[rgb]{0.33,0.33,0.33}{##1}}}
\def\PYG@tok@ne{\def\PYG@tc##1{\textcolor[rgb]{0.00,0.44,0.13}{##1}}}
\def\PYG@tok@nf{\def\PYG@tc##1{\textcolor[rgb]{0.02,0.16,0.49}{##1}}}
\def\PYG@tok@si{\let\PYG@it=\textit\def\PYG@tc##1{\textcolor[rgb]{0.44,0.63,0.82}{##1}}}
\def\PYG@tok@s2{\def\PYG@tc##1{\textcolor[rgb]{0.25,0.44,0.63}{##1}}}
\def\PYG@tok@vi{\def\PYG@tc##1{\textcolor[rgb]{0.73,0.38,0.84}{##1}}}
\def\PYG@tok@nt{\let\PYG@bf=\textbf\def\PYG@tc##1{\textcolor[rgb]{0.02,0.16,0.45}{##1}}}
\def\PYG@tok@nv{\def\PYG@tc##1{\textcolor[rgb]{0.73,0.38,0.84}{##1}}}
\def\PYG@tok@s1{\def\PYG@tc##1{\textcolor[rgb]{0.25,0.44,0.63}{##1}}}
\def\PYG@tok@gp{\let\PYG@bf=\textbf\def\PYG@tc##1{\textcolor[rgb]{0.78,0.36,0.04}{##1}}}
\def\PYG@tok@sh{\def\PYG@tc##1{\textcolor[rgb]{0.25,0.44,0.63}{##1}}}
\def\PYG@tok@ow{\let\PYG@bf=\textbf\def\PYG@tc##1{\textcolor[rgb]{0.00,0.44,0.13}{##1}}}
\def\PYG@tok@sx{\def\PYG@tc##1{\textcolor[rgb]{0.78,0.36,0.04}{##1}}}
\def\PYG@tok@bp{\def\PYG@tc##1{\textcolor[rgb]{0.00,0.44,0.13}{##1}}}
\def\PYG@tok@c1{\let\PYG@it=\textit\def\PYG@tc##1{\textcolor[rgb]{0.25,0.50,0.56}{##1}}}
\def\PYG@tok@kc{\let\PYG@bf=\textbf\def\PYG@tc##1{\textcolor[rgb]{0.00,0.44,0.13}{##1}}}
\def\PYG@tok@c{\let\PYG@it=\textit\def\PYG@tc##1{\textcolor[rgb]{0.25,0.50,0.56}{##1}}}
\def\PYG@tok@mf{\def\PYG@tc##1{\textcolor[rgb]{0.13,0.50,0.31}{##1}}}
\def\PYG@tok@err{\def\PYG@bc##1{\fcolorbox[rgb]{1.00,0.00,0.00}{1,1,1}{##1}}}
\def\PYG@tok@kd{\let\PYG@bf=\textbf\def\PYG@tc##1{\textcolor[rgb]{0.00,0.44,0.13}{##1}}}
\def\PYG@tok@ss{\def\PYG@tc##1{\textcolor[rgb]{0.32,0.47,0.09}{##1}}}
\def\PYG@tok@sr{\def\PYG@tc##1{\textcolor[rgb]{0.14,0.33,0.53}{##1}}}
\def\PYG@tok@mo{\def\PYG@tc##1{\textcolor[rgb]{0.13,0.50,0.31}{##1}}}
\def\PYG@tok@mi{\def\PYG@tc##1{\textcolor[rgb]{0.13,0.50,0.31}{##1}}}
\def\PYG@tok@kn{\let\PYG@bf=\textbf\def\PYG@tc##1{\textcolor[rgb]{0.00,0.44,0.13}{##1}}}
\def\PYG@tok@o{\def\PYG@tc##1{\textcolor[rgb]{0.40,0.40,0.40}{##1}}}
\def\PYG@tok@kr{\let\PYG@bf=\textbf\def\PYG@tc##1{\textcolor[rgb]{0.00,0.44,0.13}{##1}}}
\def\PYG@tok@s{\def\PYG@tc##1{\textcolor[rgb]{0.25,0.44,0.63}{##1}}}
\def\PYG@tok@kp{\def\PYG@tc##1{\textcolor[rgb]{0.00,0.44,0.13}{##1}}}
\def\PYG@tok@w{\def\PYG@tc##1{\textcolor[rgb]{0.73,0.73,0.73}{##1}}}
\def\PYG@tok@kt{\def\PYG@tc##1{\textcolor[rgb]{0.56,0.13,0.00}{##1}}}
\def\PYG@tok@sc{\def\PYG@tc##1{\textcolor[rgb]{0.25,0.44,0.63}{##1}}}
\def\PYG@tok@sb{\def\PYG@tc##1{\textcolor[rgb]{0.25,0.44,0.63}{##1}}}
\def\PYG@tok@k{\let\PYG@bf=\textbf\def\PYG@tc##1{\textcolor[rgb]{0.00,0.44,0.13}{##1}}}
\def\PYG@tok@se{\let\PYG@bf=\textbf\def\PYG@tc##1{\textcolor[rgb]{0.25,0.44,0.63}{##1}}}
\def\PYG@tok@sd{\let\PYG@it=\textit\def\PYG@tc##1{\textcolor[rgb]{0.25,0.44,0.63}{##1}}}

\def\PYGZbs{\char`\\}
\def\PYGZus{\char`\_}
\def\PYGZob{\char`\{}
\def\PYGZcb{\char`\}}
\def\PYGZca{\char`\^}
\def\PYGZsh{\char`\#}
\def\PYGZpc{\char`\%}
\def\PYGZdl{\char`\$}
\def\PYGZti{\char`\~}
% for compatibility with earlier versions
\def\PYGZat{@}
\def\PYGZlb{[}
\def\PYGZrb{]}
\makeatother

\begin{document}

\maketitle
\tableofcontents
\phantomsection\label{index::doc}


This is the documentation of the PKP program. The general structur of the classes is the following:
- There are pyrolysis program (like CPD and FG-DVC) specific ones to write the configuration files, launch their program and read the specific output files and suppport the other classes with the pyrolysis program specific data like species-column. Every of the program has also it's own class calculating the species and energy balance.
-The class independent Fitting procedures.

The manual is also structured the same way. Firstly the specific classes for CPD and FG-DVC and afterwards the fitting classes.

Required packages for PKP are (actual all packages except of scipy and numpy were already automatically installed with python):
\begin{itemize}
\item {} 
scipy

\item {} 
numpy

\item {} 
os

\item {} 
StringIO

\item {} 
pylab

\item {} 
platform

\end{itemize}

Contents:


\chapter{The CPD Classes}
\label{CPDClasses::doc}\label{CPDClasses:the-pkp-code-documentation}\label{CPDClasses:the-cpd-classes}
The CPD classes are contained in the files:
- CPD\_Fit\_lin\_regr.py  (Writes the CPD input file and launches of the program)
- CPD\_Compos\_and\_Energy.py  (The species and energy balance)
- Fit\_one\_run.py  (`CPD\_Result' reads the CPD output and contains the output specific information.)
\phantomsection\label{CPDClasses:ss-readgen}

\section{The Class writing the CPD instruction File and launching CPD}
\label{CPDClasses:ss-readgen}\label{CPDClasses:the-class-writing-the-cpd-instruction-file-and-launching-cpd}\label{CPDClasses:my-reference-label}\index{SetterAndLauncher (class in CPD\_Fit\_lin\_regr)}

\begin{fulllineitems}
\phantomsection\label{CPDClasses:CPD_Fit_lin_regr.SetterAndLauncher}\pysigline{\strong{class }\code{CPD\_Fit\_lin\_regr.}\bfcode{SetterAndLauncher}}
This class is able to write the CPD input file and launch the CPD program. Before writing the CPD input file (method `writeInstructFile') set all parameter using the corresponding methods (SetCoalParameter, SetOperateCond, SetNumericalParam, CalcCoalParam). After writing the instruct file, the .Run method can be used.
\index{CalcCoalParam() (CPD\_Fit\_lin\_regr.SetterAndLauncher method)}

\begin{fulllineitems}
\phantomsection\label{CPDClasses:CPD_Fit_lin_regr.SetterAndLauncher.CalcCoalParam}\pysiglinewithargsret{\bfcode{CalcCoalParam}}{}{}
Calculates the CPD coal parameter mdel, mw, p0, sig and sets the as an attribute of the class.

\end{fulllineitems}

\index{Run() (CPD\_Fit\_lin\_regr.SetterAndLauncher method)}

\begin{fulllineitems}
\phantomsection\label{CPDClasses:CPD_Fit_lin_regr.SetterAndLauncher.Run}\pysiglinewithargsret{\bfcode{Run}}{\emph{CPD\_exe}, \emph{Input\_File}}{}
Launches CPD\_exe and inputs Input\_File. If the CPD executable is in another directory than the Python script enter the whole path for CPD\_exe.

\end{fulllineitems}

\index{SetCoalParameter() (CPD\_Fit\_lin\_regr.SetterAndLauncher method)}

\begin{fulllineitems}
\phantomsection\label{CPDClasses:CPD_Fit_lin_regr.SetterAndLauncher.SetCoalParameter}\pysiglinewithargsret{\bfcode{SetCoalParameter}}{\emph{fcar}, \emph{fhyd}, \emph{fnit}, \emph{foxy}, \emph{VMdaf}}{}
Set the mass fraction of carbon (UA), hydrogen  (UA), nitrogen (UA), oxygen (UA) and the daf fraction of volatile matter (PA).

\end{fulllineitems}

\index{SetNumericalParam() (CPD\_Fit\_lin\_regr.SetterAndLauncher method)}

\begin{fulllineitems}
\phantomsection\label{CPDClasses:CPD_Fit_lin_regr.SetterAndLauncher.SetNumericalParam}\pysiglinewithargsret{\bfcode{SetNumericalParam}}{\emph{dt}, \emph{timax}}{}
Sets the numerical parameter and the maximum simulation time. dt is a vector with the following information: {[}dt\_initial,print-increment,dt\_max{]}

\end{fulllineitems}

\index{SetOperateCond() (CPD\_Fit\_lin\_regr.SetterAndLauncher method)}

\begin{fulllineitems}
\phantomsection\label{CPDClasses:CPD_Fit_lin_regr.SetterAndLauncher.SetOperateCond}\pysiglinewithargsret{\bfcode{SetOperateCond}}{\emph{pressure}, \emph{TimeTemp}}{}
Stes the operating condtions pressure and the time-temperature array. TimeTemp is an 2D-Array. One line of this array is {[}time\_i,temp\_i{]}.

\end{fulllineitems}

\index{writeInstructFile() (CPD\_Fit\_lin\_regr.SetterAndLauncher method)}

\begin{fulllineitems}
\phantomsection\label{CPDClasses:CPD_Fit_lin_regr.SetterAndLauncher.writeInstructFile}\pysiglinewithargsret{\bfcode{writeInstructFile}}{\emph{Dirpath}}{}
Writes the File `CPD\_input.dat' into the directory Dirpath.

\end{fulllineitems}


\end{fulllineitems}



\section{The Class making the Species and the Energy Balance}
\label{CPDClasses:the-class-making-the-species-and-the-energy-balance}\index{SpeciesBalance (class in Compos\_and\_Energy)}

\begin{fulllineitems}
\phantomsection\label{CPDClasses:Compos_and_Energy.SpeciesBalance}\pysigline{\strong{class }\code{Compos\_and\_Energy.}\bfcode{SpeciesBalance}}
This is the parent Class for the CPD and FG-DVC specific Species Balances, containing general methods like the Dulong formular.
\index{Dulong() (Compos\_and\_Energy.SpeciesBalance method)}

\begin{fulllineitems}
\phantomsection\label{CPDClasses:Compos_and_Energy.SpeciesBalance.Dulong}\pysiglinewithargsret{\bfcode{Dulong}}{}{}
Uses the Dulong formular to calculate the Higher heating value. The output is in J/kg.

\end{fulllineitems}

\index{SpeciesIndex() (Compos\_and\_Energy.SpeciesBalance method)}

\begin{fulllineitems}
\phantomsection\label{CPDClasses:Compos_and_Energy.SpeciesBalance.SpeciesIndex}\pysiglinewithargsret{\bfcode{SpeciesIndex}}{\emph{species}}{}
Returns the column number of the input species.

\end{fulllineitems}

\index{correctUA() (Compos\_and\_Energy.SpeciesBalance method)}

\begin{fulllineitems}
\phantomsection\label{CPDClasses:Compos_and_Energy.SpeciesBalance.correctUA}\pysiglinewithargsret{\bfcode{correctUA}}{}{}
The difference of the UA to 1 is putted to coal.

\end{fulllineitems}


\end{fulllineitems}

\index{CPD\_SpeciesBalance (class in Compos\_and\_Energy)}

\begin{fulllineitems}
\phantomsection\label{CPDClasses:Compos_and_Energy.CPD_SpeciesBalance}\pysiglinewithargsret{\strong{class }\code{Compos\_and\_Energy.}\bfcode{CPD\_SpeciesBalance}}{\emph{CPD\_ResultObject}, \emph{UAC}, \emph{UAH}, \emph{UAN}, \emph{UAO}, \emph{PAVM}, \emph{PAFC}, \emph{HHV}, \emph{MTar}}{}
This class calculates the Species and the Energy balance for CPD. See the manual for the formulars and more details.
\index{\_CPD\_SpeciesBalance\_\_CheckOthers() (Compos\_and\_Energy.CPD\_SpeciesBalance method)}

\begin{fulllineitems}
\phantomsection\label{CPDClasses:Compos_and_Energy.CPD_SpeciesBalance._CPD_SpeciesBalance__CheckOthers}\pysiglinewithargsret{\bfcode{\_CPD\_SpeciesBalance\_\_CheckOthers}}{}{}
If the yield of nitrogen (is equal the UA of nitrogen) is lower than the species `Other' the difference is set equal Methane.

\end{fulllineitems}

\index{\_CPD\_SpeciesBalance\_\_CheckOxygen() (Compos\_and\_Energy.CPD\_SpeciesBalance method)}

\begin{fulllineitems}
\phantomsection\label{CPDClasses:Compos_and_Energy.CPD_SpeciesBalance._CPD_SpeciesBalance__CheckOxygen}\pysiglinewithargsret{\bfcode{\_CPD\_SpeciesBalance\_\_CheckOxygen}}{}{}
Checks weather the amount of Oxygen in the light gases is lower than in the Ultimate Analysis. If not, the amount of these species decreases. If yes, the tar contains oxygen.

\end{fulllineitems}

\index{\_CPD\_SpeciesBalance\_\_QPyro() (Compos\_and\_Energy.CPD\_SpeciesBalance method)}

\begin{fulllineitems}
\phantomsection\label{CPDClasses:Compos_and_Energy.CPD_SpeciesBalance._CPD_SpeciesBalance__QPyro}\pysiglinewithargsret{\bfcode{\_CPD\_SpeciesBalance\_\_QPyro}}{}{}
Calculates the heat of the pyrolysis process, assuming heat of tar formation is equal zero.

\end{fulllineitems}

\index{\_CPD\_SpeciesBalance\_\_Q\_React() (Compos\_and\_Energy.CPD\_SpeciesBalance method)}

\begin{fulllineitems}
\phantomsection\label{CPDClasses:Compos_and_Energy.CPD_SpeciesBalance._CPD_SpeciesBalance__Q_React}\pysiglinewithargsret{\bfcode{\_CPD\_SpeciesBalance\_\_Q\_React}}{}{}
Calculates the Heat of Reaction of the coal cumbustion.

\end{fulllineitems}

\index{\_CPD\_SpeciesBalance\_\_TarComp() (Compos\_and\_Energy.CPD\_SpeciesBalance method)}

\begin{fulllineitems}
\phantomsection\label{CPDClasses:Compos_and_Energy.CPD_SpeciesBalance._CPD_SpeciesBalance__TarComp}\pysiglinewithargsret{\bfcode{\_CPD\_SpeciesBalance\_\_TarComp}}{}{}
Calculates and returns the tar composition.

\end{fulllineitems}

\index{\_CPD\_SpeciesBalance\_\_closeResultFile() (Compos\_and\_Energy.CPD\_SpeciesBalance method)}

\begin{fulllineitems}
\phantomsection\label{CPDClasses:Compos_and_Energy.CPD_SpeciesBalance._CPD_SpeciesBalance__closeResultFile}\pysiglinewithargsret{\bfcode{\_CPD\_SpeciesBalance\_\_closeResultFile}}{}{}
Closes the CPD Composition file.

\end{fulllineitems}

\index{\_CPD\_SpeciesBalance\_\_correctYields() (Compos\_and\_Energy.CPD\_SpeciesBalance method)}

\begin{fulllineitems}
\phantomsection\label{CPDClasses:Compos_and_Energy.CPD_SpeciesBalance._CPD_SpeciesBalance__correctYields}\pysiglinewithargsret{\bfcode{\_CPD\_SpeciesBalance\_\_correctYields}}{}{}
Correct the amount of the yields `Other'.

\end{fulllineitems}

\index{\_CPD\_SpeciesBalance\_\_hfRaw() (Compos\_and\_Energy.CPD\_SpeciesBalance method)}

\begin{fulllineitems}
\phantomsection\label{CPDClasses:Compos_and_Energy.CPD_SpeciesBalance._CPD_SpeciesBalance__hfRaw}\pysiglinewithargsret{\bfcode{\_CPD\_SpeciesBalance\_\_hfRaw}}{}{}
Calculates the heat of formation of the coal molecule and writes it into the output file.

\end{fulllineitems}

\index{\_CPD\_SpeciesBalance\_\_hfTar() (Compos\_and\_Energy.CPD\_SpeciesBalance method)}

\begin{fulllineitems}
\phantomsection\label{CPDClasses:Compos_and_Energy.CPD_SpeciesBalance._CPD_SpeciesBalance__hfTar}\pysiglinewithargsret{\bfcode{\_CPD\_SpeciesBalance\_\_hfTar}}{}{}
Calculates the heat of formation of tar and writes it into the CPD Composition file.

\end{fulllineitems}

\index{\_CPD\_SpeciesBalance\_\_nysEq15() (Compos\_and\_Energy.CPD\_SpeciesBalance method)}

\begin{fulllineitems}
\phantomsection\label{CPDClasses:Compos_and_Energy.CPD_SpeciesBalance._CPD_SpeciesBalance__nysEq15}\pysiglinewithargsret{\bfcode{\_CPD\_SpeciesBalance\_\_nysEq15}}{}{}
Calculates the stoichiometric coefficients of the devolatilization reaction.

\end{fulllineitems}


\end{fulllineitems}



\section{The Reading Class containg the CPD specific output information}
\label{CPDClasses:the-reading-class-containg-the-cpd-specific-output-information}\index{CPD\_Result (class in Fit\_one\_run)}

\begin{fulllineitems}
\phantomsection\label{CPDClasses:Fit_one_run.CPD_Result}\pysiglinewithargsret{\strong{class }\code{Fit\_one\_run.}\bfcode{CPD\_Result}}{\emph{FilePath}}{}
Reads the CPD input and saves the values in one array. The results include the yields and the rates. The rates were calculated using a CDS. This class also contains the dictionaries for the columns in the array - the name of the species. These dictionaries are CPD-Version dependent and the only thing which has to be changed for the case of a new release of CPD with a new order of species in the result files.
\index{DictCols2Yields() (Fit\_one\_run.CPD\_Result method)}

\begin{fulllineitems}
\phantomsection\label{CPDClasses:Fit_one_run.CPD_Result.DictCols2Yields}\pysiglinewithargsret{\bfcode{DictCols2Yields}}{}{}
Returns the whole Dictionary Columns of the matrix to Yield names

\end{fulllineitems}

\index{DictYields2Cols() (Fit\_one\_run.CPD\_Result method)}

\begin{fulllineitems}
\phantomsection\label{CPDClasses:Fit_one_run.CPD_Result.DictYields2Cols}\pysiglinewithargsret{\bfcode{DictYields2Cols}}{}{}
Returns the whole Dictionary Yield names to Columns of the matrix

\end{fulllineitems}

\index{FinalYields() (Fit\_one\_run.CPD\_Result method)}

\begin{fulllineitems}
\phantomsection\label{CPDClasses:Fit_one_run.CPD_Result.FinalYields}\pysiglinewithargsret{\bfcode{FinalYields}}{}{}
Returns the last line of the Array, containing the yields at the time=time\_End

\end{fulllineitems}

\index{Name() (Fit\_one\_run.CPD\_Result method)}

\begin{fulllineitems}
\phantomsection\label{CPDClasses:Fit_one_run.CPD_Result.Name}\pysiglinewithargsret{\bfcode{Name}}{}{}
returns `CPD' as the name of the Program

\end{fulllineitems}

\index{Rates\_all() (Fit\_one\_run.CPD\_Result method)}

\begin{fulllineitems}
\phantomsection\label{CPDClasses:Fit_one_run.CPD_Result.Rates_all}\pysiglinewithargsret{\bfcode{Rates\_all}}{}{}
Returns the whole result matrix of the Rates.

\end{fulllineitems}

\index{Yields\_all() (Fit\_one\_run.CPD\_Result method)}

\begin{fulllineitems}
\phantomsection\label{CPDClasses:Fit_one_run.CPD_Result.Yields_all}\pysiglinewithargsret{\bfcode{Yields\_all}}{}{}
Returns the whole result matrix of the yields.

\end{fulllineitems}


\end{fulllineitems}



\chapter{The FG-DVC Classes}
\label{FGDVCClasses:the-fg-dvc-classes}\label{FGDVCClasses::doc}
The FG-FVC classes are contained in the files:
- FGDVC\_Fit\_lin\_regr.py  (Writes the FG-DVC `instruct.ini' and launches FG-DVC)
- FGDVC\_Compos\_and\_Energy.py  (The species and energy balance)
- Fit\_one\_run.py  (`FGDVC\_Result' reads the FG-DVC output and contains the output specific information.)
\phantomsection\label{FGDVCClasses:ss-readgen}

\section{The Class writing the FG-DVC instruction file and launches FG-DVC}
\label{FGDVCClasses:ss-readgen}\label{FGDVCClasses:the-class-writing-the-fg-dvc-instruction-file-and-launches-fg-dvc}\label{FGDVCClasses:my-reference-label}\index{SetterAndLauncher (class in FGDVC\_Fit\_lin\_regr)}

\begin{fulllineitems}
\phantomsection\label{FGDVCClasses:FGDVC_Fit_lin_regr.SetterAndLauncher}\pysigline{\strong{class }\code{FGDVC\_Fit\_lin\_regr.}\bfcode{SetterAndLauncher}}
This class is able to write the `instruct.ini' and launch the `fgdvcd.exe'. Before writing the instruct.ini (method `writeInstructFile') set all parameter using the corresponding methods (at least necessary: set1CoalLocation, set2KinLocation, set3PolyLocation, set5Pressure, set7Ramp). After writing the instruct file, the .Run method can be used.
\index{Run() (FGDVC\_Fit\_lin\_regr.SetterAndLauncher method)}

\begin{fulllineitems}
\phantomsection\label{FGDVCClasses:FGDVC_Fit_lin_regr.SetterAndLauncher.Run}\pysiglinewithargsret{\bfcode{Run}}{\emph{PathFromEXE}}{}
Lauchnes fgdvcd.exe. The `PathFromEXE' should not include the .exe and end with a backslash.

\end{fulllineitems}

\index{set1CoalLocation() (FGDVC\_Fit\_lin\_regr.SetterAndLauncher method)}

\begin{fulllineitems}
\phantomsection\label{FGDVCClasses:FGDVC_Fit_lin_regr.SetterAndLauncher.set1CoalLocation}\pysiglinewithargsret{\bfcode{set1CoalLocation}}{\emph{PathCoalFile}}{}
Sets the coal composition file directory.

\end{fulllineitems}

\index{set2KinLocation() (FGDVC\_Fit\_lin\_regr.SetterAndLauncher method)}

\begin{fulllineitems}
\phantomsection\label{FGDVCClasses:FGDVC_Fit_lin_regr.SetterAndLauncher.set2KinLocation}\pysiglinewithargsret{\bfcode{set2KinLocation}}{\emph{PathKinFile}}{}
Sets the coal kinetic file directory.

\end{fulllineitems}

\index{set3PolyLocation() (FGDVC\_Fit\_lin\_regr.SetterAndLauncher method)}

\begin{fulllineitems}
\phantomsection\label{FGDVCClasses:FGDVC_Fit_lin_regr.SetterAndLauncher.set3PolyLocation}\pysiglinewithargsret{\bfcode{set3PolyLocation}}{\emph{PathPolyFile}}{}
Sets the coal polymer file directory.

\end{fulllineitems}

\index{set4RunID() (FGDVC\_Fit\_lin\_regr.SetterAndLauncher method)}

\begin{fulllineitems}
\phantomsection\label{FGDVCClasses:FGDVC_Fit_lin_regr.SetterAndLauncher.set4RunID}\pysiglinewithargsret{\bfcode{set4RunID}}{\emph{pyrolysisORgeology}}{}
Sets weather the pyrolysis process or the geolegy process shall be modeled. For more information see FG-DVC manual.

\end{fulllineitems}

\index{set5Pressure() (FGDVC\_Fit\_lin\_regr.SetterAndLauncher method)}

\begin{fulllineitems}
\phantomsection\label{FGDVCClasses:FGDVC_Fit_lin_regr.SetterAndLauncher.set5Pressure}\pysiglinewithargsret{\bfcode{set5Pressure}}{\emph{PressureIn\_atm}}{}
Sets the operating pressure (float).

\end{fulllineitems}

\index{set6Theorie() (FGDVC\_Fit\_lin\_regr.SetterAndLauncher method)}

\begin{fulllineitems}
\phantomsection\label{FGDVCClasses:FGDVC_Fit_lin_regr.SetterAndLauncher.set6Theorie}\pysiglinewithargsret{\bfcode{set6Theorie}}{\emph{Theorie}, \emph{ResidenceTime}}{}
Sets the theory: 13 for no or partial tar cracking, 15 for full tar cracking. The residence input should be 0.0 for no tar cracking or a time greater zero for partial tar pressure. Full tar pressure also requires 0.0 as input, as it is the characteristic input of FG-DVC (although writing here 0.0, the full tar cracking is activated).

\end{fulllineitems}

\index{set7File() (FGDVC\_Fit\_lin\_regr.SetterAndLauncher method)}

\begin{fulllineitems}
\phantomsection\label{FGDVCClasses:FGDVC_Fit_lin_regr.SetterAndLauncher.set7File}\pysiglinewithargsret{\bfcode{set7File}}{\emph{THistoryFileLocation}}{}
For the case a temperature history shall be imported, this method should be used. `THistoryFileLocation' is the directory of the .txt file containing two columns: first column the time in seconds, the second the temperature indegree Celsius-

\end{fulllineitems}

\index{set7Ramp() (FGDVC\_Fit\_lin\_regr.SetterAndLauncher method)}

\begin{fulllineitems}
\phantomsection\label{FGDVCClasses:FGDVC_Fit_lin_regr.SetterAndLauncher.set7Ramp}\pysiglinewithargsret{\bfcode{set7Ramp}}{\emph{timeTotal}, \emph{dt}, \emph{dT}, \emph{finalPyrolysisTemp}, \emph{initialT}, \emph{HeatingRate}}{}
Sets the following time relevant and temperature history relevant parameter: the total simulation time `timeTotal', the constant numerical time step `dt', the maximum temperture step `dT', the final pyrolysis temperature `finalPyrolysisTemp', the temperature at t=0 `initialT', and the constant heating rate `HeatingRate'. All these parameter are required for the case a linear heating rate should be modeled.

\end{fulllineitems}

\index{set9AshMoisture() (FGDVC\_Fit\_lin\_regr.SetterAndLauncher method)}

\begin{fulllineitems}
\phantomsection\label{FGDVCClasses:FGDVC_Fit_lin_regr.SetterAndLauncher.set9AshMoisture}\pysiglinewithargsret{\bfcode{set9AshMoisture}}{\emph{AshContent}, \emph{MoistureContent}}{}
Sets the amount of ash and moisture in the coal. By initializing the SetterAndLauncher object, both of these values are setted equal zero.

\end{fulllineitems}

\index{setTRamp\_or\_TFile() (FGDVC\_Fit\_lin\_regr.SetterAndLauncher method)}

\begin{fulllineitems}
\phantomsection\label{FGDVCClasses:FGDVC_Fit_lin_regr.SetterAndLauncher.setTRamp_or_TFile}\pysiglinewithargsret{\bfcode{setTRamp\_or\_TFile}}{\emph{selectRamp\_or\_File}}{}
Select weather the time should be defined using a linear ramp (`selectRamp\_or\_File'='Ramp') or with a input file (`selectRamp\_or\_File'='File').

\end{fulllineitems}

\index{writeInstructFile() (FGDVC\_Fit\_lin\_regr.SetterAndLauncher method)}

\begin{fulllineitems}
\phantomsection\label{FGDVCClasses:FGDVC_Fit_lin_regr.SetterAndLauncher.writeInstructFile}\pysiglinewithargsret{\bfcode{writeInstructFile}}{\emph{Filepath}}{}
Writes the File `instruct.ini' into the directory `Filepath', which should end with FGDVC\_8-2-3/FGDVC .

\end{fulllineitems}


\end{fulllineitems}



\subsection{The Class generating the Coal File}
\label{FGDVCClasses:the-class-generating-the-coal-file}\index{WriteFGDVCCoalFile (class in ReadInputFiles)}

\begin{fulllineitems}
\phantomsection\label{FGDVCClasses:ReadInputFiles.WriteFGDVCCoalFile}\pysiglinewithargsret{\strong{class }\code{ReadInputFiles.}\bfcode{WriteFGDVCCoalFile}}{\emph{CoalGenFile}}{}
writes the file, which will be inputted into the FG-DVC coal generator
\index{setCoalComp() (ReadInputFiles.WriteFGDVCCoalFile method)}

\begin{fulllineitems}
\phantomsection\label{FGDVCClasses:ReadInputFiles.WriteFGDVCCoalFile.setCoalComp}\pysiglinewithargsret{\bfcode{setCoalComp}}{\emph{Carbon}, \emph{Hydrogen}, \emph{Oxygen}, \emph{Nitrogen}, \emph{Sulfur}, \emph{SulfurPyrite}}{}
Enter the coal composition with values in percent which have to sum up to 100

\end{fulllineitems}

\index{write() (ReadInputFiles.WriteFGDVCCoalFile method)}

\begin{fulllineitems}
\phantomsection\label{FGDVCClasses:ReadInputFiles.WriteFGDVCCoalFile.write}\pysiglinewithargsret{\bfcode{write}}{\emph{CoalsDirectory}, \emph{CoalResultFileName}}{}
writes the FG-DVC coal generator input file

\end{fulllineitems}


\end{fulllineitems}



\section{The Class making the Species and the Energy Balance}
\label{FGDVCClasses:the-class-making-the-species-and-the-energy-balance}\index{SpeciesBalance (class in Compos\_and\_Energy)}

\begin{fulllineitems}
\phantomsection\label{FGDVCClasses:Compos_and_Energy.SpeciesBalance}\pysigline{\strong{class }\code{Compos\_and\_Energy.}\bfcode{SpeciesBalance}}
This is the parent Class for the CPD and FG-DVC specific Species Balances, containing general methods like the Dulong formular.
\index{Dulong() (Compos\_and\_Energy.SpeciesBalance method)}

\begin{fulllineitems}
\phantomsection\label{FGDVCClasses:Compos_and_Energy.SpeciesBalance.Dulong}\pysiglinewithargsret{\bfcode{Dulong}}{}{}
Uses the Dulong formular to calculate the Higher heating value. The output is in J/kg.

\end{fulllineitems}

\index{SpeciesIndex() (Compos\_and\_Energy.SpeciesBalance method)}

\begin{fulllineitems}
\phantomsection\label{FGDVCClasses:Compos_and_Energy.SpeciesBalance.SpeciesIndex}\pysiglinewithargsret{\bfcode{SpeciesIndex}}{\emph{species}}{}
Returns the column number of the input species.

\end{fulllineitems}

\index{correctUA() (Compos\_and\_Energy.SpeciesBalance method)}

\begin{fulllineitems}
\phantomsection\label{FGDVCClasses:Compos_and_Energy.SpeciesBalance.correctUA}\pysiglinewithargsret{\bfcode{correctUA}}{}{}
The difference of the UA to 1 is putted to coal.

\end{fulllineitems}


\end{fulllineitems}

\index{FGDVC\_SpeciesBalance (class in Compos\_and\_Energy)}

\begin{fulllineitems}
\phantomsection\label{FGDVCClasses:Compos_and_Energy.FGDVC_SpeciesBalance}\pysiglinewithargsret{\strong{class }\code{Compos\_and\_Energy.}\bfcode{FGDVC\_SpeciesBalance}}{\emph{FGDVC\_ResultObject}, \emph{UAC}, \emph{UAH}, \emph{UAN}, \emph{UAO}, \emph{PAVM}, \emph{PAFC}, \emph{HHV}, \emph{MTar}}{}
This class calculates the Species and the Energy balance for FG-DVC. See the manual for the formulars and more details.
\index{\_FGDVC\_SpeciesBalance\_\_EnergyBalance() (Compos\_and\_Energy.FGDVC\_SpeciesBalance method)}

\begin{fulllineitems}
\phantomsection\label{FGDVCClasses:Compos_and_Energy.FGDVC_SpeciesBalance._FGDVC_SpeciesBalance__EnergyBalance}\pysiglinewithargsret{\bfcode{\_FGDVC\_SpeciesBalance\_\_EnergyBalance}}{}{}
Calculates the heat of formation of tar.

\end{fulllineitems}

\index{\_FGDVC\_SpeciesBalance\_\_TarComp() (Compos\_and\_Energy.FGDVC\_SpeciesBalance method)}

\begin{fulllineitems}
\phantomsection\label{FGDVCClasses:Compos_and_Energy.FGDVC_SpeciesBalance._FGDVC_SpeciesBalance__TarComp}\pysiglinewithargsret{\bfcode{\_FGDVC\_SpeciesBalance\_\_TarComp}}{}{}
Calculates the Tar composition using analyisis of the Ultimate Analysis.

\end{fulllineitems}

\index{\_FGDVC\_SpeciesBalance\_\_closeFile() (Compos\_and\_Energy.FGDVC\_SpeciesBalance method)}

\begin{fulllineitems}
\phantomsection\label{FGDVCClasses:Compos_and_Energy.FGDVC_SpeciesBalance._FGDVC_SpeciesBalance__closeFile}\pysiglinewithargsret{\bfcode{\_FGDVC\_SpeciesBalance\_\_closeFile}}{}{}
Closes the FG-DVC Composition file.

\end{fulllineitems}

\index{\_FGDVC\_SpeciesBalance\_\_correctYields() (Compos\_and\_Energy.FGDVC\_SpeciesBalance method)}

\begin{fulllineitems}
\phantomsection\label{FGDVCClasses:Compos_and_Energy.FGDVC_SpeciesBalance._FGDVC_SpeciesBalance__correctYields}\pysiglinewithargsret{\bfcode{\_FGDVC\_SpeciesBalance\_\_correctYields}}{}{}
Further, only the yields of char, tar, CO, CO2, H2O, CH4, N2, H2, O2. Modifies the yields, merge species like Olefins parafins, HCN to tar. Further nitrogen evolves only as N2, so UAN=Yield of nitrogen.

\end{fulllineitems}

\index{\_FGDVC\_SpeciesBalance\_\_writeEnergyResults() (Compos\_and\_Energy.FGDVC\_SpeciesBalance method)}

\begin{fulllineitems}
\phantomsection\label{FGDVCClasses:Compos_and_Energy.FGDVC_SpeciesBalance._FGDVC_SpeciesBalance__writeEnergyResults}\pysiglinewithargsret{\bfcode{\_FGDVC\_SpeciesBalance\_\_writeEnergyResults}}{}{}
Writes the Energy results into the result file.

\end{fulllineitems}

\index{\_FGDVC\_SpeciesBalance\_\_writeSpeciesResults() (Compos\_and\_Energy.FGDVC\_SpeciesBalance method)}

\begin{fulllineitems}
\phantomsection\label{FGDVCClasses:Compos_and_Energy.FGDVC_SpeciesBalance._FGDVC_SpeciesBalance__writeSpeciesResults}\pysiglinewithargsret{\bfcode{\_FGDVC\_SpeciesBalance\_\_writeSpeciesResults}}{}{}
Writes the Species Balance results into the result file.

\end{fulllineitems}


\end{fulllineitems}



\section{The Reading Class containg the CPD specific output information}
\label{FGDVCClasses:the-reading-class-containg-the-cpd-specific-output-information}\index{FGDVC\_Result (class in Fit\_one\_run)}

\begin{fulllineitems}
\phantomsection\label{FGDVCClasses:Fit_one_run.FGDVC_Result}\pysiglinewithargsret{\strong{class }\code{Fit\_one\_run.}\bfcode{FGDVC\_Result}}{\emph{FilePath}}{}
Reads the FG-DVC input and saves the values in one array. The results include the yields (from `gasyields.txt') and the rates. The rates for all species except the solids (here a CDS is used) are imported from `gasrates.txt'. The H\_2 yields were calculated by subtract all other species except parafins and olefins from the total yields (see FG-DVC manual). This H\_2-yieldsd curve was smoothed and derived using a CDS to generate the H\_2 rates. The parafins and olefins are added into the tar. This class also contains the dictionaries for the columns in the array - the name of the species. These dictionaries are FG-DVC-Version dependent and the only thing which has to be changed for the case of a new release of FG-DVC with a new order of species in the result files (this was made for Versions 8.2.2. and 8.2.3.).
\index{DictCols2Yields() (Fit\_one\_run.FGDVC\_Result method)}

\begin{fulllineitems}
\phantomsection\label{FGDVCClasses:Fit_one_run.FGDVC_Result.DictCols2Yields}\pysiglinewithargsret{\bfcode{DictCols2Yields}}{}{}
Returns the whole Dictionary Columns of the matrix to Yield names

\end{fulllineitems}

\index{DictYields2Cols() (Fit\_one\_run.FGDVC\_Result method)}

\begin{fulllineitems}
\phantomsection\label{FGDVCClasses:Fit_one_run.FGDVC_Result.DictYields2Cols}\pysiglinewithargsret{\bfcode{DictYields2Cols}}{}{}
Returns the whole Dictionary Yield names to Columns of the matrix

\end{fulllineitems}

\index{FilePath() (Fit\_one\_run.FGDVC\_Result method)}

\begin{fulllineitems}
\phantomsection\label{FGDVCClasses:Fit_one_run.FGDVC_Result.FilePath}\pysiglinewithargsret{\bfcode{FilePath}}{}{}
Returns the FG-DVC File path

\end{fulllineitems}

\index{FinalYields() (Fit\_one\_run.FGDVC\_Result method)}

\begin{fulllineitems}
\phantomsection\label{FGDVCClasses:Fit_one_run.FGDVC_Result.FinalYields}\pysiglinewithargsret{\bfcode{FinalYields}}{}{}
Returns the last line of the Array, containing the yields at the time=time\_End

\end{fulllineitems}

\index{Name() (Fit\_one\_run.FGDVC\_Result method)}

\begin{fulllineitems}
\phantomsection\label{FGDVCClasses:Fit_one_run.FGDVC_Result.Name}\pysiglinewithargsret{\bfcode{Name}}{}{}
returns `FG-DVC' as the name of the Program

\end{fulllineitems}

\index{Rates\_all() (Fit\_one\_run.FGDVC\_Result method)}

\begin{fulllineitems}
\phantomsection\label{FGDVCClasses:Fit_one_run.FGDVC_Result.Rates_all}\pysiglinewithargsret{\bfcode{Rates\_all}}{}{}
Returns the whole result matrix of the Rates.

\end{fulllineitems}

\index{Yields\_all() (Fit\_one\_run.FGDVC\_Result method)}

\begin{fulllineitems}
\phantomsection\label{FGDVCClasses:Fit_one_run.FGDVC_Result.Yields_all}\pysiglinewithargsret{\bfcode{Yields\_all}}{}{}
Returns the whole result matrix of the yields.

\end{fulllineitems}


\end{fulllineitems}



\chapter{The Fitting Classes}
\label{FittingClasses:the-fitting-classes}\label{FittingClasses::doc}

\section{The General Fitting Support Class}
\label{FittingClasses:the-general-fitting-support-class}\index{Fit\_one\_run (class in Fit\_one\_run)}

\begin{fulllineitems}
\phantomsection\label{FittingClasses:Fit_one_run.Fit_one_run}\pysiglinewithargsret{\strong{class }\code{Fit\_one\_run.}\bfcode{Fit\_one\_run}}{\emph{ResultObject}}{}
Imports from the Result objects the arrays. It provides the fitting objects with the yields and rates over time for the specific species. This class futher offers the option to plot the generated fitting results.
\index{Dt() (Fit\_one\_run.Fit\_one\_run method)}

\begin{fulllineitems}
\phantomsection\label{FittingClasses:Fit_one_run.Fit_one_run.Dt}\pysiglinewithargsret{\bfcode{Dt}}{}{}
Returns the vector with the time steps dt.

\end{fulllineitems}

\index{DtC() (Fit\_one\_run.Fit\_one\_run method)}

\begin{fulllineitems}
\phantomsection\label{FittingClasses:Fit_one_run.Fit_one_run.DtC}\pysiglinewithargsret{\bfcode{DtC}}{}{}
Returns the vector with the time steps dt\_C. This time steps are for points between the original points, so the lenght of this vector is the lenght of the time vector minus one.

\end{fulllineitems}

\index{Interpolate() (Fit\_one\_run.Fit\_one\_run method)}

\begin{fulllineitems}
\phantomsection\label{FittingClasses:Fit_one_run.Fit_one_run.Interpolate}\pysiglinewithargsret{\bfcode{Interpolate}}{\emph{Species}}{}
Outputs the interpolation object Species(time).

\end{fulllineitems}

\index{LineNumberMaxRate() (Fit\_one\_run.Fit\_one\_run method)}

\begin{fulllineitems}
\phantomsection\label{FittingClasses:Fit_one_run.Fit_one_run.LineNumberMaxRate}\pysiglinewithargsret{\bfcode{LineNumberMaxRate}}{\emph{Species}}{}
Returns the line with the maximum Rate of the inputted species.

\end{fulllineitems}

\index{MassCoal() (Fit\_one\_run.Fit\_one\_run method)}

\begin{fulllineitems}
\phantomsection\label{FittingClasses:Fit_one_run.Fit_one_run.MassCoal}\pysiglinewithargsret{\bfcode{MassCoal}}{}{}
returns the Vector with the solid mass(t)

\end{fulllineitems}

\index{MassVM\_s() (Fit\_one\_run.Fit\_one\_run method)}

\begin{fulllineitems}
\phantomsection\label{FittingClasses:Fit_one_run.Fit_one_run.MassVM_s}\pysiglinewithargsret{\bfcode{MassVM\_s}}{}{}
Returns the Vector with the mass of the volatile Matter over time

\end{fulllineitems}

\index{NPoints() (Fit\_one\_run.Fit\_one\_run method)}

\begin{fulllineitems}
\phantomsection\label{FittingClasses:Fit_one_run.Fit_one_run.NPoints}\pysiglinewithargsret{\bfcode{NPoints}}{}{}
returns number of Points for each species over time. Is equal the number of time points.

\end{fulllineitems}

\index{Name() (Fit\_one\_run.Fit\_one\_run method)}

\begin{fulllineitems}
\phantomsection\label{FittingClasses:Fit_one_run.Fit_one_run.Name}\pysiglinewithargsret{\bfcode{Name}}{}{}
Returns the Name of the imported Result object (e.g. `CPD')

\end{fulllineitems}

\index{Rate() (Fit\_one\_run.Fit\_one\_run method)}

\begin{fulllineitems}
\phantomsection\label{FittingClasses:Fit_one_run.Fit_one_run.Rate}\pysiglinewithargsret{\bfcode{Rate}}{\emph{species}}{}
Returns the Vector of the species rate(t). The species can be inputted with the Column number (integer) or the name corresponding to the dictionary saved in the result class (string).

\end{fulllineitems}

\index{RateSingleSpec() (Fit\_one\_run.Fit\_one\_run method)}

\begin{fulllineitems}
\phantomsection\label{FittingClasses:Fit_one_run.Fit_one_run.RateSingleSpec}\pysiglinewithargsret{\bfcode{RateSingleSpec}}{\emph{NameSpecies}}{}
Returns the Rate of the species (inputted as string) by calculate it from the yields by using a CDS

\end{fulllineitems}

\index{SpeciesIndex() (Fit\_one\_run.Fit\_one\_run method)}

\begin{fulllineitems}
\phantomsection\label{FittingClasses:Fit_one_run.Fit_one_run.SpeciesIndex}\pysiglinewithargsret{\bfcode{SpeciesIndex}}{\emph{species}}{}
Returns the species column number (integer) of the recieved species name (string)

\end{fulllineitems}

\index{SpeciesName() (Fit\_one\_run.Fit\_one\_run method)}

\begin{fulllineitems}
\phantomsection\label{FittingClasses:Fit_one_run.Fit_one_run.SpeciesName}\pysiglinewithargsret{\bfcode{SpeciesName}}{\emph{ColumnNumber}}{}
Returns the species name (string) of the recieved column number (integer)

\end{fulllineitems}

\index{SpeciesNames() (Fit\_one\_run.Fit\_one\_run method)}

\begin{fulllineitems}
\phantomsection\label{FittingClasses:Fit_one_run.Fit_one_run.SpeciesNames}\pysiglinewithargsret{\bfcode{SpeciesNames}}{}{}
Returns a list with all species names (including time and temperature).

\end{fulllineitems}

\index{Time() (Fit\_one\_run.Fit\_one\_run method)}

\begin{fulllineitems}
\phantomsection\label{FittingClasses:Fit_one_run.Fit_one_run.Time}\pysiglinewithargsret{\bfcode{Time}}{}{}
Returns the time vector

\end{fulllineitems}

\index{Yield() (Fit\_one\_run.Fit\_one\_run method)}

\begin{fulllineitems}
\phantomsection\label{FittingClasses:Fit_one_run.Fit_one_run.Yield}\pysiglinewithargsret{\bfcode{Yield}}{\emph{species}}{}
Returns the Vector of the species yield(t). The species can be inputted with the Column number (integer) or the name corresponding to the dictionary saved in the result class (string).

\end{fulllineitems}

\index{plt\_InputVectors() (Fit\_one\_run.Fit\_one\_run method)}

\begin{fulllineitems}
\phantomsection\label{FittingClasses:Fit_one_run.Fit_one_run.plt_InputVectors}\pysiglinewithargsret{\bfcode{plt\_InputVectors}}{\emph{xVector}, \emph{y1Vector}, \emph{y2Vector}, \emph{y3Vector}, \emph{y4Vector}, \emph{y1Name}, \emph{y2Name}, \emph{y3Name}, \emph{y4Name}}{}
Plots the y input Vectors vs. the x input vector.

\end{fulllineitems}

\index{plt\_RateVsTime() (Fit\_one\_run.Fit\_one\_run method)}

\begin{fulllineitems}
\phantomsection\label{FittingClasses:Fit_one_run.Fit_one_run.plt_RateVsTime}\pysiglinewithargsret{\bfcode{plt\_RateVsTime}}{\emph{ColumnNumber}}{}
Plots the original rates output of the pyrolysis program (as e.g. CPD) of the species marke with the columns number

\end{fulllineitems}

\index{plt\_YieldVsTime() (Fit\_one\_run.Fit\_one\_run method)}

\begin{fulllineitems}
\phantomsection\label{FittingClasses:Fit_one_run.Fit_one_run.plt_YieldVsTime}\pysiglinewithargsret{\bfcode{plt\_YieldVsTime}}{\emph{ColumnNumber}}{}
Plots the original yield output of the pyrolysis program (as e.g. CPD) of the species marke with the columns number

\end{fulllineitems}


\end{fulllineitems}



\section{A simple two point astimator for constant rate}
\label{FittingClasses:a-simple-two-point-astimator-for-constant-rate}\index{TwoPointEstimator (class in Fit\_one\_run)}

\begin{fulllineitems}
\phantomsection\label{FittingClasses:Fit_one_run.TwoPointEstimator}\pysigline{\strong{class }\code{Fit\_one\_run.}\bfcode{TwoPointEstimator}}
Solves the devolatilization reaction analytically using two arbitrary selected points and the constant rate model. Unprecise. Should only be used for tests.

\end{fulllineitems}



\section{The Least Square Optimization Class}
\label{FittingClasses:the-least-square-optimization-class}\index{TwoPointEstimator (class in Fit\_one\_run)}

\begin{fulllineitems}
\pysigline{\strong{class }\code{Fit\_one\_run.}\bfcode{TwoPointEstimator}}
Solves the devolatilization reaction analytically using two arbitrary selected points and the constant rate model. Unprecise. Should only be used for tests.

\end{fulllineitems}



\section{The Model parent Class}
\label{FittingClasses:the-model-parent-class}\index{Model (class in Fit\_one\_run)}

\begin{fulllineitems}
\phantomsection\label{FittingClasses:Fit_one_run.Model}\pysiglinewithargsret{\strong{class }\code{Fit\_one\_run.}\bfcode{Model}}{\emph{Name}}{}
Parent class of the children ConstantRateModel, the three Arrhenius Models (notations) and the Kobayashi models.
\index{ErrorRate() (Fit\_one\_run.Model method)}

\begin{fulllineitems}
\phantomsection\label{FittingClasses:Fit_one_run.Model.ErrorRate}\pysiglinewithargsret{\bfcode{ErrorRate}}{\emph{fgdvc}, \emph{Species}}{}
Returns the absolute deviation per point between the fitted and the original rate curve.

\end{fulllineitems}

\index{ErrorYield() (Fit\_one\_run.Model method)}

\begin{fulllineitems}
\phantomsection\label{FittingClasses:Fit_one_run.Model.ErrorYield}\pysiglinewithargsret{\bfcode{ErrorYield}}{\emph{fgdvc}, \emph{Species}}{}
Returns the absolute deviation per point between the fitted and the original yield curve.

\end{fulllineitems}

\index{ParamVector() (Fit\_one\_run.Model method)}

\begin{fulllineitems}
\phantomsection\label{FittingClasses:Fit_one_run.Model.ParamVector}\pysiglinewithargsret{\bfcode{ParamVector}}{}{}
Returns the Vector containing the kinetic parameter of the Model (refering to the child model).

\end{fulllineitems}

\index{alpha\_d() (Fit\_one\_run.Model method)}

\begin{fulllineitems}
\phantomsection\label{FittingClasses:Fit_one_run.Model.alpha_d}\pysiglinewithargsret{\bfcode{alpha\_d}}{\emph{fgdvc}}{}
returns alpha\_d which is the final yield of tar devided by amount of volatile Matter.

\end{fulllineitems}

\index{beta\_d() (Fit\_one\_run.Model method)}

\begin{fulllineitems}
\phantomsection\label{FittingClasses:Fit_one_run.Model.beta_d}\pysiglinewithargsret{\bfcode{beta\_d}}{\emph{fgdvc}, \emph{species}}{}
returns beta\_d which is the final yield of a species devided by amount of volatile Matter.

\end{fulllineitems}

\index{calcRate() (Fit\_one\_run.Model method)}

\begin{fulllineitems}
\phantomsection\label{FittingClasses:Fit_one_run.Model.calcRate}\pysiglinewithargsret{\bfcode{calcRate}}{\emph{fgdvc}, \emph{time}, \emph{T}, \emph{Name}}{}
Generates the Rates using the yields vector and a CDS.

\end{fulllineitems}

\index{deriveC() (Fit\_one\_run.Model method)}

\begin{fulllineitems}
\phantomsection\label{FittingClasses:Fit_one_run.Model.deriveC}\pysiglinewithargsret{\bfcode{deriveC}}{\emph{fgdvc}, \emph{yVector}}{}
Returns a CDS of the inputted yVector.

\end{fulllineitems}

\index{plot() (Fit\_one\_run.Model method)}

\begin{fulllineitems}
\phantomsection\label{FittingClasses:Fit_one_run.Model.plot}\pysiglinewithargsret{\bfcode{plot}}{\emph{fgdvc}, \emph{Species}}{}
Plot the yield and the rates over time with two curves: one is the original data, the other the fitting curve. Also file `PyrolysisProgramName-Species.out' (e.g. `CPD-CO2.out') containing the time (s), yields (kg/kg), rates (kg/(kg s)).

\end{fulllineitems}

\index{pltRate() (Fit\_one\_run.Model method)}

\begin{fulllineitems}
\phantomsection\label{FittingClasses:Fit_one_run.Model.pltRate}\pysiglinewithargsret{\bfcode{pltRate}}{\emph{fgdvc}, \emph{xValueToPlot}, \emph{yValueToPlot}}{}
Plots the rates (to select with yValueToPlot) over Time or Temperature (to slect with xValueToPlot).

\end{fulllineitems}

\index{pltYield() (Fit\_one\_run.Model method)}

\begin{fulllineitems}
\phantomsection\label{FittingClasses:Fit_one_run.Model.pltYield}\pysiglinewithargsret{\bfcode{pltYield}}{\emph{fgdvc}, \emph{xValueToPlot}, \emph{yValueToPlot}}{}
Plots the yields (to select with yValueToPlot) over Time or Temperature (to slect with xValueToPlot).

\end{fulllineitems}

\index{setParamVector() (Fit\_one\_run.Model method)}

\begin{fulllineitems}
\phantomsection\label{FittingClasses:Fit_one_run.Model.setParamVector}\pysiglinewithargsret{\bfcode{setParamVector}}{\emph{ParameterList}}{}
Sets the Vector containing the kinetic parameter of the Model (refering to the child model).

\end{fulllineitems}


\end{fulllineitems}



\section{The Model children Classes}
\label{FittingClasses:the-model-children-classes}\index{ConstantRateModel (class in Fit\_one\_run)}

\begin{fulllineitems}
\phantomsection\label{FittingClasses:Fit_one_run.ConstantRateModel}\pysiglinewithargsret{\strong{class }\code{Fit\_one\_run.}\bfcode{ConstantRateModel}}{\emph{InitialParameterVector}}{}
The model calculating the mass with m(t)=m\_s0+(m\_s0-m\_s,e)*e**(-k*(t-t\_start)) from the ODE dm/dt = -k*(m-m\_s,e). The Parameter to optimize are k and t\_start.

\end{fulllineitems}

\index{ArrheniusModel (class in Fit\_one\_run)}

\begin{fulllineitems}
\phantomsection\label{FittingClasses:Fit_one_run.ArrheniusModel}\pysiglinewithargsret{\strong{class }\code{Fit\_one\_run.}\bfcode{ArrheniusModel}}{\emph{InitialParameterVector}}{}
The Arrhenius model in the standart notation: dm/dt=A*(T**b)*exp(-E/T)*(m\_s-m) with the parameter a,b,E to optimize.
\index{ConvertKinFactors() (Fit\_one\_run.ArrheniusModel method)}

\begin{fulllineitems}
\phantomsection\label{FittingClasses:Fit_one_run.ArrheniusModel.ConvertKinFactors}\pysiglinewithargsret{\bfcode{ConvertKinFactors}}{\emph{ParameterVector}}{}
Dummy. Function actual has to convert the parameter into the standart Arrhenius notation.

\end{fulllineitems}

\index{calcMass() (Fit\_one\_run.ArrheniusModel method)}

\begin{fulllineitems}
\phantomsection\label{FittingClasses:Fit_one_run.ArrheniusModel.calcMass}\pysiglinewithargsret{\bfcode{calcMass}}{\emph{fgdvc}, \emph{time}, \emph{T}, \emph{Name}}{}
Outputs the mass(t) using the model specific equation.

\end{fulllineitems}


\end{fulllineitems}

\index{ArrheniusModelAlternativeNotation1 (class in Fit\_one\_run)}

\begin{fulllineitems}
\phantomsection\label{FittingClasses:Fit_one_run.ArrheniusModelAlternativeNotation1}\pysiglinewithargsret{\strong{class }\code{Fit\_one\_run.}\bfcode{ArrheniusModelAlternativeNotation1}}{\emph{InitialParameterVector}}{}
Arrhenius model with a notation having a better optimization behaviour: dm/dt=exp{[}k0-a*(T0/T(t)-1){]}*(ms-m). See the documentation for the reference. The parameters to optimize are k0 and a.
\index{ConvertKinFactors() (Fit\_one\_run.ArrheniusModelAlternativeNotation1 method)}

\begin{fulllineitems}
\phantomsection\label{FittingClasses:Fit_one_run.ArrheniusModelAlternativeNotation1.ConvertKinFactors}\pysiglinewithargsret{\bfcode{ConvertKinFactors}}{\emph{ParameterVector}}{}
Converts the own kinetic factors back to the standard Arrhenius kinetic factors.

\end{fulllineitems}

\index{ConvertKinFactorsToOwnNotation() (Fit\_one\_run.ArrheniusModelAlternativeNotation1 method)}

\begin{fulllineitems}
\phantomsection\label{FittingClasses:Fit_one_run.ArrheniusModelAlternativeNotation1.ConvertKinFactorsToOwnNotation}\pysiglinewithargsret{\bfcode{ConvertKinFactorsToOwnNotation}}{\emph{fgdvc}, \emph{ParameterVector}, \emph{Species}}{}
Converts the standard Arrhenius kinetic factors backk to the factors of the own notation.

\end{fulllineitems}

\index{calcMass() (Fit\_one\_run.ArrheniusModelAlternativeNotation1 method)}

\begin{fulllineitems}
\phantomsection\label{FittingClasses:Fit_one_run.ArrheniusModelAlternativeNotation1.calcMass}\pysiglinewithargsret{\bfcode{calcMass}}{\emph{fgdvc}, \emph{time}, \emph{T}, \emph{Name}}{}
Outputs the mass(t) using the model specific equation.

\end{fulllineitems}


\end{fulllineitems}

\index{ArrheniusModelAlternativeNotation2 (class in Fit\_one\_run)}

\begin{fulllineitems}
\phantomsection\label{FittingClasses:Fit_one_run.ArrheniusModelAlternativeNotation2}\pysiglinewithargsret{\strong{class }\code{Fit\_one\_run.}\bfcode{ArrheniusModelAlternativeNotation2}}{\emph{fgdvc}, \emph{InitialParameterVector}}{}
Arrhenius model with a notation having a better optimization behaviour: dm/dt=exp{[}c*(b1*(1/T(t)-1/T\_min)-b2*(1/T(t)-1/T\_max)){]}*(ms-m); with c=(1/T\_max-1/Tmin)**(-1). See the documentation for the reference. The parameters to optimize are b1 and b2.
\index{ConvertKinFactors() (Fit\_one\_run.ArrheniusModelAlternativeNotation2 method)}

\begin{fulllineitems}
\phantomsection\label{FittingClasses:Fit_one_run.ArrheniusModelAlternativeNotation2.ConvertKinFactors}\pysiglinewithargsret{\bfcode{ConvertKinFactors}}{\emph{ParameterVector}}{}
Converts the own kinetic factors back to the standard Arrhenius kinetic factors.

\end{fulllineitems}

\index{ConvertKinFactorsToOwnNotation() (Fit\_one\_run.ArrheniusModelAlternativeNotation2 method)}

\begin{fulllineitems}
\phantomsection\label{FittingClasses:Fit_one_run.ArrheniusModelAlternativeNotation2.ConvertKinFactorsToOwnNotation}\pysiglinewithargsret{\bfcode{ConvertKinFactorsToOwnNotation}}{\emph{fgdvc}, \emph{ParameterVector}}{}
Converts the standard Arrhenius kinetic factors backk to the factors of the own notation.

\end{fulllineitems}

\index{calcMass() (Fit\_one\_run.ArrheniusModelAlternativeNotation2 method)}

\begin{fulllineitems}
\phantomsection\label{FittingClasses:Fit_one_run.ArrheniusModelAlternativeNotation2.calcMass}\pysiglinewithargsret{\bfcode{calcMass}}{\emph{fgdvc}, \emph{time}, \emph{T}, \emph{Name}}{}
Outputs the mass(t) using the model specific equation.

\end{fulllineitems}


\end{fulllineitems}

\index{Kobayashi (class in Fit\_one\_run)}

\begin{fulllineitems}
\phantomsection\label{FittingClasses:Fit_one_run.Kobayashi}\pysiglinewithargsret{\strong{class }\code{Fit\_one\_run.}\bfcode{Kobayashi}}{\emph{fgdvc}, \emph{InitialParameterVector}}{}
Calculates the devolatilization reaction using the Kobayashi model. The Arrhenius equation inside are in the standard notation.
\index{ConvertKinFactors() (Fit\_one\_run.Kobayashi method)}

\begin{fulllineitems}
\phantomsection\label{FittingClasses:Fit_one_run.Kobayashi.ConvertKinFactors}\pysiglinewithargsret{\bfcode{ConvertKinFactors}}{\emph{ParameterVector}}{}
Outputs the Arrhenius equation factors in the shape {[}A1,E1,A2,E2,alpha1,alpha2{]}. Here where the real Arrhenius model is in use only a dummy function.

\end{fulllineitems}

\index{calcMass() (Fit\_one\_run.Kobayashi method)}

\begin{fulllineitems}
\phantomsection\label{FittingClasses:Fit_one_run.Kobayashi.calcMass}\pysiglinewithargsret{\bfcode{calcMass}}{\emph{fgdvc}, \emph{time}, \emph{T}, \emph{Name}}{}
Outputs the mass(t) using the model specific equation.

\end{fulllineitems}


\end{fulllineitems}

\index{KobayashiA2 (class in Fit\_one\_run)}

\begin{fulllineitems}
\phantomsection\label{FittingClasses:Fit_one_run.KobayashiA2}\pysiglinewithargsret{\strong{class }\code{Fit\_one\_run.}\bfcode{KobayashiA2}}{\emph{fgdvc}, \emph{InitialParameterVector}}{}
Calculates the devolatilization reaction using the Kobayashi model. The Arrhenius equation inside are in the secend alternative notation (see class ArrheniusModelAlternativeNotation2).
\index{ConvertKinFactors() (Fit\_one\_run.KobayashiA2 method)}

\begin{fulllineitems}
\phantomsection\label{FittingClasses:Fit_one_run.KobayashiA2.ConvertKinFactors}\pysiglinewithargsret{\bfcode{ConvertKinFactors}}{\emph{ParameterVector}}{}
Converts the alternative notaion Arrhenius factors into the satndard Arrhenius factors and return them in the shape  {[}A1,E1{]}, {[}A2,E2{]}

\end{fulllineitems}

\index{calcMass() (Fit\_one\_run.KobayashiA2 method)}

\begin{fulllineitems}
\phantomsection\label{FittingClasses:Fit_one_run.KobayashiA2.calcMass}\pysiglinewithargsret{\bfcode{calcMass}}{\emph{fgdvc}, \emph{time}, \emph{T}, \emph{Name}}{}
Outputs the mass(t) using the model specific equation.

\end{fulllineitems}


\end{fulllineitems}



\section{The Classes reading the user-Input}
\label{FittingClasses:the-classes-reading-the-user-input}\index{ReadFile (class in ReadInputFiles)}

\begin{fulllineitems}
\phantomsection\label{FittingClasses:ReadInputFiles.ReadFile}\pysiglinewithargsret{\strong{class }\code{ReadInputFiles.}\bfcode{ReadFile}}{\emph{InputFile}}{}
general parent class for the reading objects CPDFile and FGDVCFile
\index{Fitting() (ReadInputFiles.ReadFile method)}

\begin{fulllineitems}
\phantomsection\label{FittingClasses:ReadInputFiles.ReadFile.Fitting}\pysiglinewithargsret{\bfcode{Fitting}}{\emph{FileNote}}{}
outputs the Fitting mode for Pyrolysis Program output (string: `constantRate','Arrhenius','Kobayashi'). Possible input: `constantRate', `Arrhenius' or `Kobayashi'

\end{fulllineitems}

\index{UsePyrolProgr() (ReadInputFiles.ReadFile method)}

\begin{fulllineitems}
\phantomsection\label{FittingClasses:ReadInputFiles.ReadFile.UsePyrolProgr}\pysiglinewithargsret{\bfcode{UsePyrolProgr}}{\emph{FileNote}}{}
gets the information, whether Pyrolsis Program will be in use. Enter `Yes' or `True' for the case it should be used

\end{fulllineitems}

\index{getText() (ReadInputFiles.ReadFile method)}

\begin{fulllineitems}
\phantomsection\label{FittingClasses:ReadInputFiles.ReadFile.getText}\pysiglinewithargsret{\bfcode{getText}}{\emph{FileNote}}{}
output the data of the line below the FileNote as a string

\end{fulllineitems}

\index{getValue() (ReadInputFiles.ReadFile method)}

\begin{fulllineitems}
\phantomsection\label{FittingClasses:ReadInputFiles.ReadFile.getValue}\pysiglinewithargsret{\bfcode{getValue}}{\emph{FileNote}}{}
output the data of the line below the FileNote as a float

\end{fulllineitems}

\index{readLines() (ReadInputFiles.ReadFile method)}

\begin{fulllineitems}
\phantomsection\label{FittingClasses:ReadInputFiles.ReadFile.readLines}\pysiglinewithargsret{\bfcode{readLines}}{}{}
reads the input File line by line

\end{fulllineitems}


\end{fulllineitems}

\index{OperCondInput (class in ReadInputFiles)}

\begin{fulllineitems}
\phantomsection\label{FittingClasses:ReadInputFiles.OperCondInput}\pysiglinewithargsret{\strong{class }\code{ReadInputFiles.}\bfcode{OperCondInput}}{\emph{InputFile}}{}
Reads the input file for the operating conditions and also writes the temperature-history file, required by FG-DVC.
\index{getTimePoints() (ReadInputFiles.OperCondInput method)}

\begin{fulllineitems}
\phantomsection\label{FittingClasses:ReadInputFiles.OperCondInput.getTimePoints}\pysiglinewithargsret{\bfcode{getTimePoints}}{\emph{FileNoteBegin}, \emph{FileNoteEnd}}{}
reads the time points in the shape `time, temperature' for the lines between the line with the FileNoteBegin and the line with the FileNoteEnd

\end{fulllineitems}

\index{writeFGDVCtTHist() (ReadInputFiles.OperCondInput method)}

\begin{fulllineitems}
\phantomsection\label{FittingClasses:ReadInputFiles.OperCondInput.writeFGDVCtTHist}\pysiglinewithargsret{\bfcode{writeFGDVCtTHist}}{\emph{tTPoints}, \emph{dt}, \emph{OutputFilePath}}{}
Writes output file for FG-DVC containing in first column time in s, in the second tempearure in degree Celsius. FG-DVC will import this file. The time-temperature array has to be a numpy.array, dt a float, OutputFilePath a string.

\end{fulllineitems}


\end{fulllineitems}



\chapter{The Code of the Main Part}
\label{MainProgramCode::doc}\label{MainProgramCode:the-code-of-the-main-part}
This is the file having access to the several classes. So far it is the central procedure of PKP.

\begin{Verbatim}[commandchars=\\\{\}]
\PYG{k+kn}{import} \PYG{n+nn}{CPD\PYGZus{}Fit\PYGZus{}lin\PYGZus{}regr}        \PYG{c}{\PYGZsh{}writes CPD-instruct File, launches CPD}
\PYG{k+kn}{import} \PYG{n+nn}{FGDVC\PYGZus{}Fit\PYGZus{}lin\PYGZus{}regr}      \PYG{c}{\PYGZsh{}writes FG-DVC-instruct File, launches FG-DVC and fittes using eq. (68 ) (BachelorThesis)}
\PYG{k+kn}{import} \PYG{n+nn}{Fit\PYGZus{}one\PYGZus{}run}             \PYG{c}{\PYGZsh{}fittes the kinetic parameter for CPD output using eq. (46) (BachelorThesis)}
\PYG{k+kn}{import} \PYG{n+nn}{Compos\PYGZus{}and\PYGZus{}Energy}       \PYG{c}{\PYGZsh{}Species balance and energy balance for CPD and FG-DVC}
\PYG{k+kn}{import} \PYG{n+nn}{ReadInputFiles}          \PYG{c}{\PYGZsh{}reads the user input files, writes FG-DVC coalsd.exe coal generation file}
\PYG{k+kn}{import} \PYG{n+nn}{os}
\PYG{k+kn}{import} \PYG{n+nn}{numpy} \PYG{k+kn}{as} \PYG{n+nn}{np}
\PYG{k+kn}{import} \PYG{n+nn}{platform}
\PYG{c}{\PYGZsh{}}
\PYG{k}{def} \PYG{n+nf}{DAF}\PYG{p}{(}\PYG{n}{PAFC\PYGZus{}asRecieved}\PYG{p}{,}\PYG{n}{PAVM\PYGZus{}asRecieved}\PYG{p}{)}\PYG{p}{:}
    \PYG{l+s+sd}{"""calculates PAFC, PAVM  from the as recieved state to the daf state of coal"""}
    \PYG{n}{fractionFC}\PYG{o}{=}\PYG{n}{PAFC\PYGZus{}asRecieved}\PYG{o}{/}\PYG{p}{(}\PYG{n}{PAFC\PYGZus{}asRecieved}\PYG{o}{+}\PYG{n}{PAVM\PYGZus{}asRecieved}\PYG{p}{)}
    \PYG{n}{fractionVM}\PYG{o}{=}\PYG{n}{PAVM\PYGZus{}asRecieved}\PYG{o}{/}\PYG{p}{(}\PYG{n}{PAFC\PYGZus{}asRecieved}\PYG{o}{+}\PYG{n}{PAVM\PYGZus{}asRecieved}\PYG{p}{)}
    \PYG{k}{return} \PYG{l+m+mf}{100.}\PYG{o}{*}\PYG{n}{fractionFC}\PYG{p}{,} \PYG{l+m+mf}{100.}\PYG{o}{*}\PYG{n}{fractionVM}
\PYG{c}{\PYGZsh{}}
\PYG{c}{\PYGZsh{}Which operating Sytem?}
\PYG{n}{oSystem}\PYG{o}{=}\PYG{n}{platform}\PYG{o}{.}\PYG{n}{system}\PYG{p}{(}\PYG{p}{)}
\PYG{c}{\PYGZsh{}Directories:}
\PYG{c}{\PYGZsh{}gets the current directory:}
\PYG{n}{workingDir}\PYG{o}{=}\PYG{n}{os}\PYG{o}{.}\PYG{n}{getcwd}\PYG{p}{(}\PYG{p}{)}\PYG{o}{+}\PYG{l+s}{'}\PYG{l+s}{/}\PYG{l+s}{'}
\PYG{c}{\PYGZsh{}FG-DVC Library coals folder name:}
\PYG{n}{FG\PYGZus{}LibCoalDir}\PYG{o}{=}\PYG{l+s}{'}\PYG{l+s}{input}\PYG{l+s}{'}
\PYG{c}{\PYGZsh{}FG-DVC coal generation (coalsd.exe) folder name:}
\PYG{n}{FG\PYGZus{}GenCoalDir}\PYG{o}{=}\PYG{l+s}{'}\PYG{l+s}{coals}\PYG{l+s}{'}
\PYG{c}{\PYGZsh{}FG-DVC fgdvcd.exe folder name:}
\PYG{n}{FG\PYGZus{}ExeCoalDir}\PYG{o}{=}\PYG{l+s}{'}\PYG{l+s}{FGDVC}\PYG{l+s}{'}
\PYG{c}{\PYGZsh{}Input file name for coalsd.exe:}
\PYG{n}{FG\PYGZus{}CoalGenFileName}\PYG{o}{=}\PYG{l+s}{'}\PYG{l+s}{CoalGen\PYGZus{}FGDVC.txt}\PYG{l+s}{'}
\PYG{c}{\PYGZsh{}File name for generated coal file:}
\PYG{n}{FG\PYGZus{}CoalName}\PYG{o}{=}\PYG{l+s}{'}\PYG{l+s}{GenCoal}\PYG{l+s}{'}
\PYG{c}{\PYGZsh{}}
\PYG{c}{\PYGZsh{}}
\PYG{c}{\PYGZsh{}}
\PYG{c}{\PYGZsh{}get parameters from input files:}
\PYG{c}{\PYGZsh{}}
\PYG{c}{\PYGZsh{}Coal File:}
\PYG{n}{CoalInput}\PYG{o}{=}\PYG{n}{ReadInputFiles}\PYG{o}{.}\PYG{n}{ReadFile}\PYG{p}{(}\PYG{n}{workingDir}\PYG{o}{+}\PYG{l+s}{'}\PYG{l+s}{Coal.inp}\PYG{l+s}{'}\PYG{p}{)}
\PYG{n}{PAFC\PYGZus{}asrec}\PYG{o}{=}\PYG{n}{CoalInput}\PYG{o}{.}\PYG{n}{getValue}\PYG{p}{(}\PYG{l+s}{'}\PYG{l+s}{Fixed Carbon:}\PYG{l+s}{'}\PYG{p}{)}
\PYG{n}{PAVM\PYGZus{}asrec}\PYG{o}{=}\PYG{n}{CoalInput}\PYG{o}{.}\PYG{n}{getValue}\PYG{p}{(}\PYG{l+s}{'}\PYG{l+s}{Volatile Matter:}\PYG{l+s}{'}\PYG{p}{)}
\PYG{c}{\PYGZsh{}gets daf values, as CPD needs daf as input:}
\PYG{n}{PAFC\PYGZus{}daf}\PYG{p}{,} \PYG{n}{PAVM\PYGZus{}daf} \PYG{o}{=} \PYG{n}{DAF}\PYG{p}{(}\PYG{n}{PAFC\PYGZus{}asrec}\PYG{p}{,}\PYG{n}{PAVM\PYGZus{}asrec}\PYG{p}{)}
\PYG{n}{UAC}\PYG{o}{=}\PYG{n}{CoalInput}\PYG{o}{.}\PYG{n}{getValue}\PYG{p}{(}\PYG{l+s}{'}\PYG{l+s}{UA Carbon:}\PYG{l+s}{'}\PYG{p}{)}
\PYG{n}{UAH}\PYG{o}{=}\PYG{n}{CoalInput}\PYG{o}{.}\PYG{n}{getValue}\PYG{p}{(}\PYG{l+s}{'}\PYG{l+s}{UA Hydrogen:}\PYG{l+s}{'}\PYG{p}{)}
\PYG{n}{UAN}\PYG{o}{=}\PYG{n}{CoalInput}\PYG{o}{.}\PYG{n}{getValue}\PYG{p}{(}\PYG{l+s}{'}\PYG{l+s}{UA Nitrogen:}\PYG{l+s}{'}\PYG{p}{)}
\PYG{n}{UAO}\PYG{o}{=}\PYG{n}{CoalInput}\PYG{o}{.}\PYG{n}{getValue}\PYG{p}{(}\PYG{l+s}{'}\PYG{l+s}{UA Oxygen:}\PYG{l+s}{'}\PYG{p}{)}
\PYG{n}{HHV}\PYG{o}{=}\PYG{n}{CoalInput}\PYG{o}{.}\PYG{n}{getValue}\PYG{p}{(}\PYG{l+s}{'}\PYG{l+s}{Higher Heating Value, as recieved, in J/kg:}\PYG{l+s}{'}\PYG{p}{)}
\PYG{n}{MTar}\PYG{o}{=}\PYG{n}{CoalInput}\PYG{o}{.}\PYG{n}{getValue}\PYG{p}{(}\PYG{l+s}{'}\PYG{l+s}{Tar Molecule weight, MTar:}\PYG{l+s}{'}\PYG{p}{)}
\PYG{n}{WeightY}\PYG{o}{=}\PYG{n}{CoalInput}\PYG{o}{.}\PYG{n}{getValue}\PYG{p}{(}\PYG{l+s}{'}\PYG{l+s}{Weight-Parameter yields for fitting the kinetics:}\PYG{l+s}{'}\PYG{p}{)}
\PYG{n}{WeightR}\PYG{o}{=}\PYG{n}{CoalInput}\PYG{o}{.}\PYG{n}{getValue}\PYG{p}{(}\PYG{l+s}{'}\PYG{l+s}{Weight-Parameter rates for fitting the kinetics:}\PYG{l+s}{'}\PYG{p}{)}
\PYG{c}{\PYGZsh{}}
\PYG{c}{\PYGZsh{}CPD Properties:}
\PYG{n}{CPDInput}\PYG{o}{=}\PYG{n}{ReadInputFiles}\PYG{o}{.}\PYG{n}{ReadFile}\PYG{p}{(}\PYG{n}{workingDir}\PYG{o}{+}\PYG{l+s}{'}\PYG{l+s}{CPD.inp}\PYG{l+s}{'}\PYG{p}{)}
\PYG{n}{CPDselect}\PYG{o}{=}\PYG{n}{CPDInput}\PYG{o}{.}\PYG{n}{UsePyrolProgr}\PYG{p}{(}\PYG{l+s}{'}\PYG{l+s}{useCPD?:}\PYG{l+s}{'}\PYG{p}{)}
\PYG{n}{CPD\PYGZus{}FittingKineticParameter\PYGZus{}Select}\PYG{o}{=}\PYG{n}{CPDInput}\PYG{o}{.}\PYG{n}{Fitting}\PYG{p}{(}\PYG{l+s}{"}\PYG{l+s}{selected fitting Approximation: }\PYG{l+s}{'}\PYG{l+s}{constantRate}\PYG{l+s}{'}\PYG{l+s}{, }\PYG{l+s}{'}\PYG{l+s}{Arrhenius}\PYG{l+s}{'}\PYG{l+s}{, }\PYG{l+s}{'}\PYG{l+s}{Kobayashi}\PYG{l+s}{'}\PYG{l+s}{ or }\PYG{l+s}{'}\PYG{l+s}{None}\PYG{l+s}{'}\PYG{l+s}{; selectedFit:}\PYG{l+s}{"}\PYG{p}{)}
\PYG{n}{CPDdt}\PYG{o}{=}\PYG{p}{[}\PYG{l+m+mi}{0}\PYG{p}{,}\PYG{l+m+mi}{1}\PYG{p}{,}\PYG{l+m+mi}{2}\PYG{p}{]} \PYG{c}{\PYGZsh{}0:initila dt, 1: print increment, 2: dt max}
\PYG{n}{CPDdt}\PYG{p}{[}\PYG{l+m+mi}{0}\PYG{p}{]}\PYG{o}{=}\PYG{p}{(}\PYG{n}{CPDInput}\PYG{o}{.}\PYG{n}{getValue}\PYG{p}{(}\PYG{l+s}{'}\PYG{l+s}{initial time step in s:}\PYG{l+s}{'}\PYG{p}{)}\PYG{p}{)}
\PYG{n}{CPDdt}\PYG{p}{[}\PYG{l+m+mi}{1}\PYG{p}{]}\PYG{o}{=}\PYG{p}{(}\PYG{n}{CPDInput}\PYG{o}{.}\PYG{n}{getValue}\PYG{p}{(}\PYG{l+s}{'}\PYG{l+s}{print increment, writeValue:}\PYG{l+s}{'}\PYG{p}{)}\PYG{p}{)}
\PYG{c}{\PYGZsh{}}
\PYG{c}{\PYGZsh{}FG-DVC Properties:}
\PYG{n}{FGDVCInput}\PYG{o}{=}\PYG{n}{ReadInputFiles}\PYG{o}{.}\PYG{n}{ReadFile}\PYG{p}{(}\PYG{n}{workingDir}\PYG{o}{+}\PYG{l+s}{'}\PYG{l+s}{FGDVC.inp}\PYG{l+s}{'}\PYG{p}{)}
\PYG{n}{FG\PYGZus{}select}\PYG{o}{=}\PYG{n}{FGDVCInput}\PYG{o}{.}\PYG{n}{UsePyrolProgr}\PYG{p}{(}\PYG{l+s}{'}\PYG{l+s}{use FG-DVC?:}\PYG{l+s}{'}\PYG{p}{)}
\PYG{n}{FG\PYGZus{}FittingKineticParameter\PYGZus{}Select}\PYG{o}{=}\PYG{n}{FGDVCInput}\PYG{o}{.}\PYG{n}{Fitting}\PYG{p}{(}\PYG{l+s}{"}\PYG{l+s}{selected fitting Approximation: }\PYG{l+s}{'}\PYG{l+s}{constantRate}\PYG{l+s}{'}\PYG{l+s}{, }\PYG{l+s}{'}\PYG{l+s}{Arrhenius}\PYG{l+s}{'}\PYG{l+s}{, }\PYG{l+s}{'}\PYG{l+s}{Kobayashi}\PYG{l+s}{'}\PYG{l+s}{ or }\PYG{l+s}{'}\PYG{l+s}{None}\PYG{l+s}{'}\PYG{l+s}{; selectedFit:}\PYG{l+s}{"}\PYG{p}{)}
\PYG{n}{FG\PYGZus{}CoalSelection}\PYG{o}{=}\PYG{n+nb}{int}\PYG{p}{(}\PYG{n}{FGDVCInput}\PYG{o}{.}\PYG{n}{getValue}\PYG{p}{(}\PYG{l+s}{'}\PYG{l+s}{Choose Coal: 0 interpolate between library coals and generate own coal. Set 1 to 8 for a library coal.}\PYG{l+s}{'}\PYG{p}{)}\PYG{p}{)}
\PYG{n}{FG\PYGZus{}MainDir}\PYG{o}{=}\PYG{n}{FGDVCInput}\PYG{o}{.}\PYG{n}{getText}\PYG{p}{(}\PYG{l+s}{'}\PYG{l+s}{main directory FG-DVC:}\PYG{l+s}{'}\PYG{p}{)}
\PYG{n}{FG\PYGZus{}DirOut}\PYG{o}{=}\PYG{n}{FGDVCInput}\PYG{o}{.}\PYG{n}{getText}\PYG{p}{(}\PYG{l+s}{'}\PYG{l+s}{directory fgdvc-output:}\PYG{l+s}{'}\PYG{p}{)}
\PYG{n}{FG\PYGZus{}TarCacking}\PYG{o}{=}\PYG{n}{FGDVCInput}\PYG{o}{.}\PYG{n}{getValue}\PYG{p}{(}\PYG{l+s}{'}\PYG{l+s}{Model tar cracking? If no, set tar residence time equal 0. For a partial tar cracking enter the tar residence time in s. For full tar cracking write -1.}\PYG{l+s}{'}\PYG{p}{)}
\PYG{c}{\PYGZsh{}}
\PYG{c}{\PYGZsh{}Operating Condition File:}
\PYG{n}{OpCondInp}\PYG{o}{=}\PYG{n}{ReadInputFiles}\PYG{o}{.}\PYG{n}{OperCondInput}\PYG{p}{(}\PYG{l+s}{'}\PYG{l+s}{OperCond.inp}\PYG{l+s}{'}\PYG{p}{)}
\PYG{n}{CPD\PYGZus{}pressure}\PYG{o}{=}\PYG{n}{OpCondInp}\PYG{o}{.}\PYG{n}{getValue}\PYG{p}{(}\PYG{l+s}{'}\PYG{l+s}{pressure in atm:}\PYG{l+s}{'}\PYG{p}{)}
\PYG{n}{FG\PYGZus{}pressure}\PYG{o}{=}\PYG{n}{OpCondInp}\PYG{o}{.}\PYG{n}{getValue}\PYG{p}{(}\PYG{l+s}{'}\PYG{l+s}{pressure in atm:}\PYG{l+s}{'}\PYG{p}{)}
\PYG{c}{\PYGZsh{}FG\PYGZus{}Tstart=FGDVCInput.getValue('The starting temperature in K:')}
\PYG{c}{\PYGZsh{}FG\PYGZus{}Tend=FGDVCInput.getValue('The final pyrolysis temperature in K:')}
\PYG{c}{\PYGZsh{}FG\PYGZus{}HeatingRate=FGDVCInput.getValue('The Heating rate in K/s:')}
\PYG{n}{CPD\PYGZus{}TimeTemp}\PYG{o}{=}\PYG{n}{OpCondInp}\PYG{o}{.}\PYG{n}{getTimePoints}\PYG{p}{(}\PYG{l+s}{'}\PYG{l+s}{Time History: first column time in seconds, second column: Temperature in K. Last point must contain the final time.}\PYG{l+s}{'}\PYG{p}{,}\PYG{l+s}{'}\PYG{l+s}{End Time History}\PYG{l+s}{'}\PYG{p}{)}
\PYG{n}{CPDdt}\PYG{p}{[}\PYG{l+m+mi}{2}\PYG{p}{]}\PYG{o}{=}\PYG{n}{OpCondInp}\PYG{o}{.}\PYG{n}{getValue}\PYG{p}{(}\PYG{l+s}{'}\PYG{l+s}{FG-DVC: constant (numerical) time step; CPD: maximum time step}\PYG{l+s}{'}\PYG{p}{)}
\PYG{n}{FG\PYGZus{}dt}\PYG{o}{=}\PYG{n}{OpCondInp}\PYG{o}{.}\PYG{n}{getValue}\PYG{p}{(}\PYG{l+s}{'}\PYG{l+s}{FG-DVC: constant (numerical) time step; CPD: maximum time step}\PYG{l+s}{'}\PYG{p}{)}
\PYG{n}{FG\PYGZus{}T\PYGZus{}t\PYGZus{}History}\PYG{o}{=}\PYG{n}{FG\PYGZus{}MainDir}\PYG{o}{+}\PYG{l+s}{'}\PYG{l+s}{tTHistory.txt}\PYG{l+s}{'}
\PYG{c}{\PYGZsh{}makes for CPD time in milliseconds:}
\PYG{n}{CPD\PYGZus{}TimeTemp}\PYG{p}{[}\PYG{p}{:}\PYG{p}{,}\PYG{l+m+mi}{0}\PYG{p}{]}\PYG{o}{=}\PYG{n}{CPD\PYGZus{}TimeTemp}\PYG{p}{[}\PYG{p}{:}\PYG{p}{,}\PYG{l+m+mi}{0}\PYG{p}{]}\PYG{o}{*}\PYG{l+m+mf}{1.e3}
\PYG{n}{CPD\PYGZus{}t\PYGZus{}max}\PYG{o}{=}\PYG{n}{CPD\PYGZus{}TimeTemp}\PYG{p}{[}\PYG{o}{-}\PYG{l+m+mi}{1}\PYG{p}{,}\PYG{l+m+mi}{0}\PYG{p}{]}\PYG{o}{*}\PYG{l+m+mf}{1.e-3} \PYG{c}{\PYGZsh{}tmax in s, not ms}
\PYG{c}{\PYGZsh{}}
\PYG{c}{\PYGZsh{}}
\PYG{c}{\PYGZsh{}}
\PYG{c}{\PYGZsh{}}
\PYG{c}{\PYGZsh{}\PYGZsh{}\PYGZsh{}\PYGZsh{}CPD\PYGZsh{}\PYGZsh{}\PYGZsh{}\PYGZsh{}}
\PYG{k}{if} \PYG{n}{CPDselect}\PYG{o}{==}\PYG{n+nb+bp}{True}\PYG{p}{:}
    \PYG{c}{\PYGZsh{}launches CPD}
    \PYG{n}{CPD}\PYG{o}{=}\PYG{n}{CPD\PYGZus{}Fit\PYGZus{}lin\PYGZus{}regr}\PYG{o}{.}\PYG{n}{SetterAndLauncher}\PYG{p}{(}\PYG{p}{)}
    \PYG{n}{CPD}\PYG{o}{.}\PYG{n}{SetCoalParameter}\PYG{p}{(}\PYG{n}{UAC}\PYG{p}{,}\PYG{n}{UAH}\PYG{p}{,}\PYG{n}{UAN}\PYG{p}{,}\PYG{n}{UAO}\PYG{p}{,}\PYG{n}{PAVM\PYGZus{}daf}\PYG{p}{)}
    \PYG{n}{CPD}\PYG{o}{.}\PYG{n}{CalcCoalParam}\PYG{p}{(}\PYG{p}{)}
    \PYG{n}{CPD}\PYG{o}{.}\PYG{n}{SetOperateCond}\PYG{p}{(}\PYG{n}{CPD\PYGZus{}pressure}\PYG{p}{,}\PYG{n}{CPD\PYGZus{}TimeTemp}\PYG{p}{)}
    \PYG{n}{CPD}\PYG{o}{.}\PYG{n}{SetNumericalParam}\PYG{p}{(}\PYG{n}{CPDdt}\PYG{p}{,}\PYG{n}{CPD\PYGZus{}t\PYGZus{}max}\PYG{p}{)}
    \PYG{n}{CPD}\PYG{o}{.}\PYG{n}{writeInstructFile}\PYG{p}{(}\PYG{n}{workingDir}\PYG{p}{)}
    \PYG{k}{if} \PYG{n}{oSystem}\PYG{o}{==}\PYG{l+s}{'}\PYG{l+s}{Linux}\PYG{l+s}{'}\PYG{p}{:}
        \PYG{n}{CPD}\PYG{o}{.}\PYG{n}{Run}\PYG{p}{(}\PYG{l+s}{'}\PYG{l+s}{./}\PYG{l+s}{'}\PYG{o}{+}\PYG{l+s}{'}\PYG{l+s}{CPD.out}\PYG{l+s}{'}\PYG{p}{,}\PYG{l+s}{'}\PYG{l+s}{IN.dat}\PYG{l+s}{'}\PYG{p}{)}   \PYG{c}{\PYGZsh{}first Arg: CPD-executeable, second: Input data containing CPD input file and the output files}
    \PYG{k}{elif} \PYG{n}{oSystem}\PYG{o}{==}\PYG{l+s}{'}\PYG{l+s}{Windows}\PYG{l+s}{'}\PYG{p}{:}
        \PYG{n}{CPD}\PYG{o}{.}\PYG{n}{Run}\PYG{p}{(}\PYG{l+s}{'}\PYG{l+s}{CPD.out}\PYG{l+s}{'}\PYG{p}{,}\PYG{l+s}{'}\PYG{l+s}{IN.dat}\PYG{l+s}{'}\PYG{p}{)}   \PYG{c}{\PYGZsh{}first Arg: CPD-executeable, second: Input data containing CPD input file and the output files}
    \PYG{k}{else}\PYG{p}{:}
        \PYG{k}{print} \PYG{l+s}{"}\PYG{l+s}{The name of the opearting system couldn}\PYG{l+s}{'}\PYG{l+s}{t be found.}\PYG{l+s}{"}
    \PYG{c}{\PYGZsh{}}
    \PYG{c}{\PYGZsh{}\PYGZsh{}\PYGZsh{}calibration of the kinetic parameter:}
    \PYG{c}{\PYGZsh{}read result:}
    \PYG{n}{CPDFile}\PYG{o}{=}\PYG{n}{Fit\PYGZus{}one\PYGZus{}run}\PYG{o}{.}\PYG{n}{CPD\PYGZus{}Result}\PYG{p}{(}\PYG{n}{workingDir}\PYG{p}{)}
    \PYG{c}{\PYGZsh{} creates object, required for fitting procedures}
    \PYG{n}{CPDFit}\PYG{o}{=}\PYG{n}{Fit\PYGZus{}one\PYGZus{}run}\PYG{o}{.}\PYG{n}{Fit\PYGZus{}one\PYGZus{}run}\PYG{p}{(}\PYG{n}{CPDFile}\PYG{p}{)}
    \PYG{c}{\PYGZsh{}Array for the comparison of the sum of the individual species(t) with the (1-solid(t))}
    \PYG{n}{SumSingleYieldsCalc}\PYG{o}{=}\PYG{n}{np}\PYG{o}{.}\PYG{n}{zeros}\PYG{p}{(}\PYG{n+nb}{len}\PYG{p}{(}\PYG{n}{CPDFit}\PYG{o}{.}\PYG{n}{Yield}\PYG{p}{(}\PYG{l+s}{'}\PYG{l+s}{Time}\PYG{l+s}{'}\PYG{p}{)}\PYG{p}{)}\PYG{p}{)} \PYG{c}{\PYGZsh{}Array initialized for the sum of the single yields (calculated)}
    \PYG{n}{SumSingleYieldsCPD}\PYG{o}{=}\PYG{n}{np}\PYG{o}{.}\PYG{n}{zeros}\PYG{p}{(}\PYG{n+nb}{len}\PYG{p}{(}\PYG{n}{CPDFit}\PYG{o}{.}\PYG{n}{Yield}\PYG{p}{(}\PYG{l+s}{'}\PYG{l+s}{Time}\PYG{l+s}{'}\PYG{p}{)}\PYG{p}{)}\PYG{p}{)}  \PYG{c}{\PYGZsh{}Array initialized for the sum of the single yields (CPD output)}
    \PYG{n}{SolidYieldsCalc}\PYG{o}{=}\PYG{n}{np}\PYG{o}{.}\PYG{n}{zeros}\PYG{p}{(}\PYG{n+nb}{len}\PYG{p}{(}\PYG{n}{CPDFit}\PYG{o}{.}\PYG{n}{Yield}\PYG{p}{(}\PYG{l+s}{'}\PYG{l+s}{Time}\PYG{l+s}{'}\PYG{p}{)}\PYG{p}{)}\PYG{p}{)}     \PYG{c}{\PYGZsh{}Array initialized for the yields of the Solids (calculated)}
    \PYG{n}{SolidYieldsCPD}\PYG{o}{=}\PYG{n}{np}\PYG{o}{.}\PYG{n}{zeros}\PYG{p}{(}\PYG{n+nb}{len}\PYG{p}{(}\PYG{n}{CPDFit}\PYG{o}{.}\PYG{n}{Yield}\PYG{p}{(}\PYG{l+s}{'}\PYG{l+s}{Time}\PYG{l+s}{'}\PYG{p}{)}\PYG{p}{)}\PYG{p}{)}     \PYG{c}{\PYGZsh{}Array initialized for the yields of the Solids (CPD output)}
    \PYG{c}{\PYGZsh{}\PYGZsh{}CONSTANT RATE}
    \PYG{k}{if} \PYG{n}{CPD\PYGZus{}FittingKineticParameter\PYGZus{}Select}\PYG{o}{==}\PYG{l+s}{'}\PYG{l+s}{constantRate}\PYG{l+s}{'}\PYG{p}{:} \PYG{c}{\PYGZsh{}CR means ConstantRate}
        \PYG{n}{PredictionVector}\PYG{o}{=}\PYG{p}{[}\PYG{l+m+mi}{50}\PYG{p}{,}\PYG{l+m+mf}{0.01}\PYG{p}{]} \PYG{c}{\PYGZsh{}first argument is k, second is t\PYGZus{}start}
        \PYG{n}{LSCPD}\PYG{o}{=}\PYG{n}{Fit\PYGZus{}one\PYGZus{}run}\PYG{o}{.}\PYG{n}{LeastSquarsEstimator}\PYG{p}{(}\PYG{p}{)} \PYG{c}{\PYGZsh{}LS means LeastSquares}
        \PYG{n}{LSCPD}\PYG{o}{.}\PYG{n}{setOptimizer}\PYG{p}{(}\PYG{l+s}{'}\PYG{l+s}{fmin}\PYG{l+s}{'}\PYG{p}{)}
        \PYG{n}{LSCPD}\PYG{o}{.}\PYG{n}{setTolerance}\PYG{p}{(}\PYG{l+m+mf}{1.e-18}\PYG{p}{)}
        \PYG{n}{LSCPD}\PYG{o}{.}\PYG{n}{setWeights}\PYG{p}{(}\PYG{n}{WeightY}\PYG{p}{,}\PYG{n}{WeightR}\PYG{p}{)}
        \PYG{n}{CRCPD}\PYG{o}{=}\PYG{n}{Fit\PYGZus{}one\PYGZus{}run}\PYG{o}{.}\PYG{n}{ConstantRateModel}\PYG{p}{(}\PYG{n}{PredictionVector}\PYG{p}{)}
        \PYG{n}{outfile} \PYG{o}{=} \PYG{n+nb}{open}\PYG{p}{(}\PYG{l+s}{'}\PYG{l+s}{CPD-Results\PYGZus{}const\PYGZus{}rate.txt}\PYG{l+s}{'}\PYG{p}{,} \PYG{l+s}{'}\PYG{l+s}{w}\PYG{l+s}{'}\PYG{p}{)}
        \PYG{n}{outfile}\PYG{o}{.}\PYG{n}{write}\PYG{p}{(}\PYG{l+s}{"}\PYG{l+s}{Species}\PYG{l+s+se}{\PYGZbs{}t}\PYG{l+s+se}{\PYGZbs{}t}\PYG{l+s}{k [1/s]}\PYG{l+s+se}{\PYGZbs{}t}\PYG{l+s+se}{\PYGZbs{}t}\PYG{l+s}{t\PYGZus{}start [s]}\PYG{l+s+se}{\PYGZbs{}n}\PYG{l+s+se}{\PYGZbs{}n}\PYG{l+s}{"}\PYG{p}{)}
        \PYG{k}{for} \PYG{n}{Spec} \PYG{o+ow}{in} \PYG{n+nb}{range}\PYG{p}{(}\PYG{l+m+mi}{2}\PYG{p}{,}\PYG{n+nb}{len}\PYG{p}{(}\PYG{n}{CPDFit}\PYG{o}{.}\PYG{n}{SpeciesNames}\PYG{p}{(}\PYG{p}{)}\PYG{p}{)}\PYG{p}{,}\PYG{l+m+mi}{1}\PYG{p}{)}\PYG{p}{:}
            \PYG{n}{CRCPD}\PYG{o}{.}\PYG{n}{setParamVector}\PYG{p}{(}\PYG{n}{PredictionVector}\PYG{p}{)}
            \PYG{n}{CRCPD}\PYG{o}{.}\PYG{n}{setParamVector}\PYG{p}{(}\PYG{n}{LSCPD}\PYG{o}{.}\PYG{n}{estimate\PYGZus{}T}\PYG{p}{(}\PYG{n}{CPDFit}\PYG{p}{,}\PYG{n}{CRCPD}\PYG{p}{,}\PYG{n}{PredictionVector}\PYG{p}{,}\PYG{n}{Spec}\PYG{p}{)}\PYG{p}{)}
            \PYG{n}{CRCPD}\PYG{o}{.}\PYG{n}{plot}\PYG{p}{(}\PYG{n}{CPDFit}\PYG{p}{,}\PYG{n}{Spec}\PYG{p}{)}
            \PYG{n}{Solution}\PYG{o}{=}\PYG{n}{CRCPD}\PYG{o}{.}\PYG{n}{ParamVector}\PYG{p}{(}\PYG{p}{)}
            \PYG{n}{outfile}\PYG{o}{.}\PYG{n}{write}\PYG{p}{(}\PYG{n+nb}{str}\PYG{p}{(}\PYG{n}{CPDFit}\PYG{o}{.}\PYG{n}{SpeciesName}\PYG{p}{(}\PYG{n}{Spec}\PYG{p}{)}\PYG{p}{)}\PYG{o}{+}\PYG{l+s}{'}\PYG{l+s+se}{\PYGZbs{}t}\PYG{l+s}{'}\PYG{o}{+}\PYG{n+nb}{str}\PYG{p}{(}\PYG{n}{Solution}\PYG{p}{[}\PYG{l+m+mi}{0}\PYG{p}{]}\PYG{p}{)}\PYG{o}{+}\PYG{l+s}{'}\PYG{l+s+se}{\PYGZbs{}t}\PYG{l+s}{'}\PYG{o}{+}\PYG{n+nb}{str}\PYG{p}{(}\PYG{n}{Solution}\PYG{p}{[}\PYG{l+m+mi}{1}\PYG{p}{]}\PYG{p}{)}\PYG{o}{+}\PYG{l+s}{'}\PYG{l+s+se}{\PYGZbs{}n}\PYG{l+s}{'}\PYG{p}{)}
            \PYG{c}{\PYGZsh{}for the comparison of the species sum with (1-Solid)}
            \PYG{k}{if} \PYG{n}{CPDFit}\PYG{o}{.}\PYG{n}{SpeciesName}\PYG{p}{(}\PYG{n}{Spec}\PYG{p}{)}\PYG{o}{==}\PYG{l+s}{'}\PYG{l+s}{Solid}\PYG{l+s}{'}\PYG{p}{:}
                \PYG{n}{SolidYieldsCalc}\PYG{o}{+}\PYG{o}{=}\PYG{n}{CRCPD}\PYG{o}{.}\PYG{n}{calcMass}\PYG{p}{(}\PYG{n}{CPDFit}\PYG{p}{,}\PYG{n}{CPDFit}\PYG{o}{.}\PYG{n}{Time}\PYG{p}{(}\PYG{p}{)}\PYG{p}{,}\PYG{n}{CPDFit}\PYG{o}{.}\PYG{n}{Interpolate}\PYG{p}{(}\PYG{l+s}{'}\PYG{l+s}{Temp}\PYG{l+s}{'}\PYG{p}{)}\PYG{p}{,}\PYG{n}{Spec}\PYG{p}{)}
                \PYG{n}{SolidYieldsCPD}\PYG{o}{+}\PYG{o}{=}\PYG{n}{CPDFit}\PYG{o}{.}\PYG{n}{Yield}\PYG{p}{(}\PYG{n}{Spec}\PYG{p}{)}
            \PYG{k}{elif} \PYG{n}{CPDFit}\PYG{o}{.}\PYG{n}{SpeciesName}\PYG{p}{(}\PYG{n}{Spec}\PYG{p}{)}\PYG{o}{!=}\PYG{l+s}{'}\PYG{l+s}{Solid}\PYG{l+s}{'} \PYG{o+ow}{and} \PYG{n}{CPDFit}\PYG{o}{.}\PYG{n}{SpeciesName}\PYG{p}{(}\PYG{n}{Spec}\PYG{p}{)}\PYG{o}{!=}\PYG{l+s}{'}\PYG{l+s}{Temp}\PYG{l+s}{'} \PYG{o+ow}{and} \PYG{n}{CPDFit}\PYG{o}{.}\PYG{n}{SpeciesName}\PYG{p}{(}\PYG{n}{Spec}\PYG{p}{)}\PYG{o}{!=}\PYG{l+s}{'}\PYG{l+s}{Time}\PYG{l+s}{'} \PYG{o+ow}{and} \PYG{n}{CPDFit}\PYG{o}{.}\PYG{n}{SpeciesName}\PYG{p}{(}\PYG{n}{Spec}\PYG{p}{)}\PYG{o}{!=}\PYG{l+s}{'}\PYG{l+s}{Gas}\PYG{l+s}{'} \PYG{o+ow}{and} \PYG{n}{CPDFit}\PYG{o}{.}\PYG{n}{SpeciesName}\PYG{p}{(}\PYG{n}{Spec}\PYG{p}{)}\PYG{o}{!=}\PYG{l+s}{'}\PYG{l+s}{Total}\PYG{l+s}{'}\PYG{p}{:}
                \PYG{n}{SumSingleYieldsCalc}\PYG{o}{+}\PYG{o}{=}\PYG{n}{CRCPD}\PYG{o}{.}\PYG{n}{calcMass}\PYG{p}{(}\PYG{n}{CPDFit}\PYG{p}{,}\PYG{n}{CPDFit}\PYG{o}{.}\PYG{n}{Time}\PYG{p}{(}\PYG{p}{)}\PYG{p}{,}\PYG{n}{CPDFit}\PYG{o}{.}\PYG{n}{Interpolate}\PYG{p}{(}\PYG{l+s}{'}\PYG{l+s}{Temp}\PYG{l+s}{'}\PYG{p}{)}\PYG{p}{,}\PYG{n}{Spec}\PYG{p}{)}
                \PYG{n}{SumSingleYieldsCPD}\PYG{o}{+}\PYG{o}{=}\PYG{n}{CPDFit}\PYG{o}{.}\PYG{n}{Yield}\PYG{p}{(}\PYG{n}{Spec}\PYG{p}{)}
        \PYG{n}{outfile}\PYG{o}{.}\PYG{n}{close}\PYG{p}{(}\PYG{p}{)}
        \PYG{n}{CPDFit}\PYG{o}{.}\PYG{n}{plt\PYGZus{}InputVectors}\PYG{p}{(}\PYG{n}{CPDFit}\PYG{o}{.}\PYG{n}{Time}\PYG{p}{(}\PYG{p}{)}\PYG{p}{,}\PYG{l+m+mf}{1.}\PYG{o}{-}\PYG{n}{SolidYieldsCalc}\PYG{p}{,}\PYG{l+m+mf}{1.}\PYG{o}{-}\PYG{n}{SolidYieldsCPD}\PYG{p}{,}\PYG{n}{SumSingleYieldsCalc}\PYG{p}{,}\PYG{n}{SumSingleYieldsCPD}\PYG{p}{,}\PYG{l+s}{'}\PYG{l+s}{1-Solid; fitted}\PYG{l+s}{'}\PYG{p}{,}\PYG{l+s}{'}\PYG{l+s}{1-Solid; CPD output}\PYG{l+s}{'}\PYG{p}{,}\PYG{l+s}{'}\PYG{l+s}{Sum Yields; fitted}\PYG{l+s}{'}\PYG{p}{,}\PYG{l+s}{'}\PYG{l+s}{Sum Yields; CPD output}\PYG{l+s}{'}\PYG{p}{)}
    \PYG{c}{\PYGZsh{}\PYGZsh{}ARRHENIUS RATE}
    \PYG{k}{if} \PYG{n}{CPD\PYGZus{}FittingKineticParameter\PYGZus{}Select}\PYG{o}{==}\PYG{l+s}{'}\PYG{l+s}{Arrhenius}\PYG{l+s}{'}\PYG{p}{:} \PYG{c}{\PYGZsh{}Arr means Arrhenius}
        \PYG{n}{PredictionV0}\PYG{o}{=}\PYG{p}{[}\PYG{l+m+mf}{0.86e15}\PYG{p}{,}\PYG{l+m+mi}{0}\PYG{p}{,}\PYG{l+m+mi}{27700}\PYG{p}{]}  \PYG{c}{\PYGZsh{}for Standard Arrhenius}
        \PYG{n}{PredictionV1}\PYG{o}{=}\PYG{p}{[}\PYG{l+m+mf}{10.}\PYG{p}{,}\PYG{o}{-}\PYG{l+m+mf}{20.}\PYG{p}{]}         \PYG{c}{\PYGZsh{}for Arrhenius notation \PYGZsh{}1}
        \PYG{n}{PredictionV2}\PYG{o}{=}\PYG{p}{[}\PYG{l+m+mi}{10}\PYG{p}{,}\PYG{o}{-}\PYG{l+m+mi}{18}\PYG{p}{]}           \PYG{c}{\PYGZsh{}for Arrhenius notation \PYGZsh{}2}
        \PYG{n}{LSCPD}\PYG{o}{=}\PYG{n}{Fit\PYGZus{}one\PYGZus{}run}\PYG{o}{.}\PYG{n}{LeastSquarsEstimator}\PYG{p}{(}\PYG{p}{)}
        \PYG{n}{LSCPD}\PYG{o}{.}\PYG{n}{setOptimizer}\PYG{p}{(}\PYG{l+s}{'}\PYG{l+s}{fmin}\PYG{l+s}{'}\PYG{p}{)}\PYG{c}{\PYGZsh{}('leastsq')   \PYGZsh{} 'leastsq' often faster, but if this does not work: 'fmin' is more reliable}
        \PYG{n}{LSCPD}\PYG{o}{.}\PYG{n}{setTolerance}\PYG{p}{(}\PYG{l+m+mf}{1.e-10}\PYG{p}{)}
        \PYG{n}{LSCPD}\PYG{o}{.}\PYG{n}{setWeights}\PYG{p}{(}\PYG{n}{WeightY}\PYG{p}{,}\PYG{n}{WeightR}\PYG{p}{)}
        \PYG{n}{outfile} \PYG{o}{=} \PYG{n+nb}{open}\PYG{p}{(}\PYG{l+s}{'}\PYG{l+s}{CPD-Results\PYGZus{}ArrheniusRate.txt}\PYG{l+s}{'}\PYG{p}{,} \PYG{l+s}{'}\PYG{l+s}{w}\PYG{l+s}{'}\PYG{p}{)}
        \PYG{n}{outfile}\PYG{o}{.}\PYG{n}{write}\PYG{p}{(}\PYG{l+s}{"}\PYG{l+s}{Species}\PYG{l+s+se}{\PYGZbs{}t}\PYG{l+s+se}{\PYGZbs{}t}\PYG{l+s}{A [1/s]}\PYG{l+s+se}{\PYGZbs{}t}\PYG{l+s+se}{\PYGZbs{}t}\PYG{l+s}{b}\PYG{l+s+se}{\PYGZbs{}t}\PYG{l+s+se}{\PYGZbs{}t}\PYG{l+s}{E\PYGZus{}a [K]}\PYG{l+s+se}{\PYGZbs{}n}\PYG{l+s+se}{\PYGZbs{}n}\PYG{l+s}{"}\PYG{p}{)}
        \PYG{c}{\PYGZsh{}select one of the follwoing notations: }
        \PYG{c}{\PYGZsh{}Arr=Fit\PYGZus{}one\PYGZus{}run.ArrheniusModel(PredictionV0)}
        \PYG{c}{\PYGZsh{}Arr=Fit\PYGZus{}one\PYGZus{}run.ArrheniusModelAlternativeNotation1(PredictionV1)}
        \PYG{n}{ArrCPD}\PYG{o}{=}\PYG{n}{Fit\PYGZus{}one\PYGZus{}run}\PYG{o}{.}\PYG{n}{ArrheniusModelAlternativeNotation2}\PYG{p}{(}\PYG{n}{CPDFit}\PYG{p}{,}\PYG{n}{PredictionV2}\PYG{p}{)}
        \PYG{c}{\PYGZsh{}\PYGZsh{}\PYGZsh{}\PYGZsh{}\PYGZsh{}\PYGZsh{}\PYGZsh{}}
        \PYG{c}{\PYGZsh{}uses a separate Arrhenius model to plot, to ensure that the result converted into standart notation (!) output vector is right}
        \PYG{n}{ArrPCPD}\PYG{o}{=}\PYG{n}{Fit\PYGZus{}one\PYGZus{}run}\PYG{o}{.}\PYG{n}{ArrheniusModel}\PYG{p}{(}\PYG{p}{[}\PYG{l+m+mi}{0}\PYG{p}{,}\PYG{l+m+mi}{0}\PYG{p}{,}\PYG{l+m+mi}{0}\PYG{p}{]}\PYG{p}{)}
        \PYG{c}{\PYGZsh{}\PYGZsh{}The single species:}
        \PYG{k}{for} \PYG{n}{Species} \PYG{o+ow}{in} \PYG{n+nb}{range}\PYG{p}{(}\PYG{l+m+mi}{2}\PYG{p}{,}\PYG{n+nb}{len}\PYG{p}{(}\PYG{n}{CPDFit}\PYG{o}{.}\PYG{n}{SpeciesNames}\PYG{p}{(}\PYG{p}{)}\PYG{p}{)}\PYG{p}{,}\PYG{l+m+mi}{1}\PYG{p}{)}\PYG{p}{:}
            \PYG{k}{print} \PYG{n}{CPDFit}\PYG{o}{.}\PYG{n}{SpeciesName}\PYG{p}{(}\PYG{n}{Species}\PYG{p}{)}
            \PYG{n}{ArrCPD}\PYG{o}{.}\PYG{n}{setParamVector}\PYG{p}{(}\PYG{n}{LSCPD}\PYG{o}{.}\PYG{n}{estimate\PYGZus{}T}\PYG{p}{(}\PYG{n}{CPDFit}\PYG{p}{,}\PYG{n}{ArrCPD}\PYG{p}{,}\PYG{n}{ArrCPD}\PYG{o}{.}\PYG{n}{ParamVector}\PYG{p}{(}\PYG{p}{)}\PYG{p}{,}\PYG{n}{Species}\PYG{p}{)}\PYG{p}{)}
            \PYG{n}{Solution}\PYG{o}{=}\PYG{n}{ArrCPD}\PYG{o}{.}\PYG{n}{ConvertKinFactors}\PYG{p}{(}\PYG{n}{ArrCPD}\PYG{o}{.}\PYG{n}{ParamVector}\PYG{p}{(}\PYG{p}{)}\PYG{p}{)}
            \PYG{c}{\PYGZsh{}Solution=Arr.ParamVector()}
            \PYG{n}{ArrPCPD}\PYG{o}{.}\PYG{n}{setParamVector}\PYG{p}{(}\PYG{n}{Solution}\PYG{p}{)}
            \PYG{n}{ArrPCPD}\PYG{o}{.}\PYG{n}{plot}\PYG{p}{(}\PYG{n}{CPDFit}\PYG{p}{,}\PYG{n}{Species}\PYG{p}{)}
            \PYG{n}{outfile}\PYG{o}{.}\PYG{n}{write}\PYG{p}{(}\PYG{n+nb}{str}\PYG{p}{(}\PYG{n}{CPDFit}\PYG{o}{.}\PYG{n}{SpeciesName}\PYG{p}{(}\PYG{n}{Species}\PYG{p}{)}\PYG{p}{)}\PYG{o}{+}\PYG{l+s}{'}\PYG{l+s+se}{\PYGZbs{}t}\PYG{l+s}{'}\PYG{o}{+}\PYG{n+nb}{str}\PYG{p}{(}\PYG{n}{Solution}\PYG{p}{[}\PYG{l+m+mi}{0}\PYG{p}{]}\PYG{p}{)}\PYG{o}{+}\PYG{l+s}{'}\PYG{l+s+se}{\PYGZbs{}t}\PYG{l+s}{'}\PYG{o}{+}\PYG{n+nb}{str}\PYG{p}{(}\PYG{n}{Solution}\PYG{p}{[}\PYG{l+m+mi}{1}\PYG{p}{]}\PYG{p}{)}\PYG{o}{+}\PYG{l+s}{'}\PYG{l+s+se}{\PYGZbs{}t}\PYG{l+s}{'}\PYG{o}{+}\PYG{n+nb}{str}\PYG{p}{(}\PYG{n}{Solution}\PYG{p}{[}\PYG{l+m+mi}{2}\PYG{p}{]}\PYG{p}{)}\PYG{o}{+}\PYG{l+s}{'}\PYG{l+s+se}{\PYGZbs{}n}\PYG{l+s}{'}\PYG{p}{)}
            \PYG{c}{\PYGZsh{}for the comparison of the species sum with (1-Solid)}
            \PYG{k}{if} \PYG{n}{CPDFit}\PYG{o}{.}\PYG{n}{SpeciesName}\PYG{p}{(}\PYG{n}{Species}\PYG{p}{)}\PYG{o}{==}\PYG{l+s}{'}\PYG{l+s}{Solid}\PYG{l+s}{'}\PYG{p}{:}
                \PYG{n}{SolidYieldsCalc}\PYG{o}{+}\PYG{o}{=}\PYG{n}{ArrPCPD}\PYG{o}{.}\PYG{n}{calcMass}\PYG{p}{(}\PYG{n}{CPDFit}\PYG{p}{,}\PYG{n}{CPDFit}\PYG{o}{.}\PYG{n}{Time}\PYG{p}{(}\PYG{p}{)}\PYG{p}{,}\PYG{n}{CPDFit}\PYG{o}{.}\PYG{n}{Interpolate}\PYG{p}{(}\PYG{l+s}{'}\PYG{l+s}{Temp}\PYG{l+s}{'}\PYG{p}{)}\PYG{p}{,}\PYG{n}{Species}\PYG{p}{)}
                \PYG{n}{SolidYieldsCPD}\PYG{o}{+}\PYG{o}{=}\PYG{n}{CPDFit}\PYG{o}{.}\PYG{n}{Yield}\PYG{p}{(}\PYG{n}{Species}\PYG{p}{)}
            \PYG{k}{elif} \PYG{n}{CPDFit}\PYG{o}{.}\PYG{n}{SpeciesName}\PYG{p}{(}\PYG{n}{Species}\PYG{p}{)}\PYG{o}{!=}\PYG{l+s}{'}\PYG{l+s}{Solid}\PYG{l+s}{'} \PYG{o+ow}{and} \PYG{n}{CPDFit}\PYG{o}{.}\PYG{n}{SpeciesName}\PYG{p}{(}\PYG{n}{Species}\PYG{p}{)}\PYG{o}{!=}\PYG{l+s}{'}\PYG{l+s}{Temp}\PYG{l+s}{'} \PYG{o+ow}{and} \PYG{n}{CPDFit}\PYG{o}{.}\PYG{n}{SpeciesName}\PYG{p}{(}\PYG{n}{Species}\PYG{p}{)}\PYG{o}{!=}\PYG{l+s}{'}\PYG{l+s}{Time}\PYG{l+s}{'} \PYG{o+ow}{and} \PYG{n}{CPDFit}\PYG{o}{.}\PYG{n}{SpeciesName}\PYG{p}{(}\PYG{n}{Species}\PYG{p}{)}\PYG{o}{!=}\PYG{l+s}{'}\PYG{l+s}{Gas}\PYG{l+s}{'} \PYG{o+ow}{and} \PYG{n}{CPDFit}\PYG{o}{.}\PYG{n}{SpeciesName}\PYG{p}{(}\PYG{n}{Species}\PYG{p}{)}\PYG{o}{!=}\PYG{l+s}{'}\PYG{l+s}{Total}\PYG{l+s}{'}\PYG{p}{:}
                \PYG{n}{SumSingleYieldsCalc}\PYG{o}{+}\PYG{o}{=}\PYG{n}{ArrPCPD}\PYG{o}{.}\PYG{n}{calcMass}\PYG{p}{(}\PYG{n}{CPDFit}\PYG{p}{,}\PYG{n}{CPDFit}\PYG{o}{.}\PYG{n}{Time}\PYG{p}{(}\PYG{p}{)}\PYG{p}{,}\PYG{n}{CPDFit}\PYG{o}{.}\PYG{n}{Interpolate}\PYG{p}{(}\PYG{l+s}{'}\PYG{l+s}{Temp}\PYG{l+s}{'}\PYG{p}{)}\PYG{p}{,}\PYG{n}{Species}\PYG{p}{)}
                \PYG{n}{SumSingleYieldsCPD}\PYG{o}{+}\PYG{o}{=}\PYG{n}{CPDFit}\PYG{o}{.}\PYG{n}{Yield}\PYG{p}{(}\PYG{n}{Species}\PYG{p}{)}
        \PYG{n}{outfile}\PYG{o}{.}\PYG{n}{close}\PYG{p}{(}\PYG{p}{)}
        \PYG{n}{CPDFit}\PYG{o}{.}\PYG{n}{plt\PYGZus{}InputVectors}\PYG{p}{(}\PYG{n}{CPDFit}\PYG{o}{.}\PYG{n}{Time}\PYG{p}{(}\PYG{p}{)}\PYG{p}{,}\PYG{l+m+mf}{1.}\PYG{o}{-}\PYG{n}{SolidYieldsCalc}\PYG{p}{,}\PYG{l+m+mf}{1.}\PYG{o}{-}\PYG{n}{SolidYieldsCPD}\PYG{p}{,}\PYG{n}{SumSingleYieldsCalc}\PYG{p}{,}\PYG{n}{SumSingleYieldsCPD}\PYG{p}{,}\PYG{l+s}{'}\PYG{l+s}{1-Solid; fitted}\PYG{l+s}{'}\PYG{p}{,}\PYG{l+s}{'}\PYG{l+s}{1-Solid; CPD output}\PYG{l+s}{'}\PYG{p}{,}\PYG{l+s}{'}\PYG{l+s}{Sum Yields; fitted}\PYG{l+s}{'}\PYG{p}{,}\PYG{l+s}{'}\PYG{l+s}{Sum Yields; CPD output}\PYG{l+s}{'}\PYG{p}{)}
    \PYG{c}{\PYGZsh{}\PYGZsh{}KOBAYASHI RATE}
    \PYG{k}{if} \PYG{n}{CPD\PYGZus{}FittingKineticParameter\PYGZus{}Select}\PYG{o}{==}\PYG{l+s}{'}\PYG{l+s}{Kobayashi}\PYG{l+s}{'}\PYG{p}{:} \PYG{c}{\PYGZsh{}Kob means Kobayashi}
        \PYG{n}{PredictionVKob2}\PYG{o}{=}\PYG{p}{[}\PYG{l+m+mi}{10}\PYG{p}{,}\PYG{o}{-}\PYG{l+m+mi}{16}\PYG{p}{,}\PYG{l+m+mi}{8}\PYG{p}{,}\PYG{o}{-}\PYG{l+m+mi}{20}\PYG{p}{,}\PYG{l+m+mf}{0.5}\PYG{p}{,}\PYG{l+m+mf}{1.0}\PYG{p}{]}           \PYG{c}{\PYGZsh{}for Arrhenius notation \PYGZsh{}2 [b11,b21,b12,b22] with the second indice as the reaction}
        \PYG{n}{LSCPD}\PYG{o}{=}\PYG{n}{Fit\PYGZus{}one\PYGZus{}run}\PYG{o}{.}\PYG{n}{LeastSquarsEstimator}\PYG{p}{(}\PYG{p}{)}
        \PYG{n}{LSCPD}\PYG{o}{.}\PYG{n}{setOptimizer}\PYG{p}{(}\PYG{l+s}{'}\PYG{l+s}{fmin}\PYG{l+s}{'}\PYG{p}{)}\PYG{c}{\PYGZsh{}('leastsq')   \PYGZsh{} 'leastsq' often faster, but if this does not work: 'fmin' is more reliable}
        \PYG{n}{LSCPD}\PYG{o}{.}\PYG{n}{setTolerance}\PYG{p}{(}\PYG{l+m+mf}{1.e-7}\PYG{p}{)}
        \PYG{n}{LSCPD}\PYG{o}{.}\PYG{n}{setWeights}\PYG{p}{(}\PYG{n}{WeightY}\PYG{p}{,}\PYG{n}{WeightR}\PYG{p}{)}
        \PYG{n}{outfile} \PYG{o}{=} \PYG{n+nb}{open}\PYG{p}{(}\PYG{l+s}{'}\PYG{l+s}{CPD-Results\PYGZus{}KobayashiRate.txt}\PYG{l+s}{'}\PYG{p}{,} \PYG{l+s}{'}\PYG{l+s}{w}\PYG{l+s}{'}\PYG{p}{)}
        \PYG{n}{outfile}\PYG{o}{.}\PYG{n}{write}\PYG{p}{(}\PYG{l+s}{"}\PYG{l+s}{Species}\PYG{l+s+se}{\PYGZbs{}t}\PYG{l+s+se}{\PYGZbs{}t}\PYG{l+s+se}{\PYGZbs{}t}\PYG{l+s}{A1 [1/s]}\PYG{l+s+se}{\PYGZbs{}t}\PYG{l+s+se}{\PYGZbs{}t}\PYG{l+s+se}{\PYGZbs{}t}\PYG{l+s+se}{\PYGZbs{}t}\PYG{l+s}{E\PYGZus{}a1 [K]}\PYG{l+s+se}{\PYGZbs{}t}\PYG{l+s+se}{\PYGZbs{}t}\PYG{l+s}{A2 [1/s]}\PYG{l+s+se}{\PYGZbs{}t}\PYG{l+s+se}{\PYGZbs{}t}\PYG{l+s+se}{\PYGZbs{}t}\PYG{l+s+se}{\PYGZbs{}t}\PYG{l+s}{E\PYGZus{}a2 [K]}\PYG{l+s+se}{\PYGZbs{}t}\PYG{l+s+se}{\PYGZbs{}t}\PYG{l+s+se}{\PYGZbs{}t}\PYG{l+s+se}{\PYGZbs{}t}\PYG{l+s}{alpha1 }\PYG{l+s+se}{\PYGZbs{}t}\PYG{l+s+se}{\PYGZbs{}t}\PYG{l+s+se}{\PYGZbs{}t}\PYG{l+s}{alpha2 }\PYG{l+s+se}{\PYGZbs{}n}\PYG{l+s+se}{\PYGZbs{}n}\PYG{l+s}{"}\PYG{p}{)}
        \PYG{n}{KobCPD}\PYG{o}{=}\PYG{n}{Fit\PYGZus{}one\PYGZus{}run}\PYG{o}{.}\PYG{n}{KobayashiA2}\PYG{p}{(}\PYG{n}{CPDFit}\PYG{p}{,}\PYG{n}{PredictionVKob2}\PYG{p}{)}
        \PYG{c}{\PYGZsh{}\PYGZsh{}\PYGZsh{}\PYGZsh{}\PYGZsh{}\PYGZsh{}\PYGZsh{}}
        \PYG{c}{\PYGZsh{}uses a separate Arrhenius model to plot, to ensure that the result converted into standart notation (!) output vector is right}
        \PYG{n}{KobPCPD}\PYG{o}{=}\PYG{n}{Fit\PYGZus{}one\PYGZus{}run}\PYG{o}{.}\PYG{n}{Kobayashi}\PYG{p}{(}\PYG{n}{CPDFit}\PYG{p}{,}\PYG{p}{[}\PYG{l+m+mi}{0}\PYG{p}{,}\PYG{l+m+mi}{0}\PYG{p}{,}\PYG{l+m+mi}{0}\PYG{p}{,}\PYG{l+m+mi}{0}\PYG{p}{,}\PYG{l+m+mi}{0}\PYG{p}{,}\PYG{l+m+mi}{0}\PYG{p}{]}\PYG{p}{)}
        \PYG{c}{\PYGZsh{}\PYGZsh{}The single species:}
        \PYG{k}{for} \PYG{n}{Species} \PYG{o+ow}{in} \PYG{n+nb}{range}\PYG{p}{(}\PYG{l+m+mi}{2}\PYG{p}{,}\PYG{n+nb}{len}\PYG{p}{(}\PYG{n}{CPDFit}\PYG{o}{.}\PYG{n}{SpeciesNames}\PYG{p}{(}\PYG{p}{)}\PYG{p}{)}\PYG{p}{,}\PYG{l+m+mi}{1}\PYG{p}{)}\PYG{p}{:}
            \PYG{k}{print} \PYG{n}{CPDFit}\PYG{o}{.}\PYG{n}{SpeciesName}\PYG{p}{(}\PYG{n}{Species}\PYG{p}{)}
            \PYG{n}{KobCPD}\PYG{o}{.}\PYG{n}{setParamVector}\PYG{p}{(}\PYG{n}{LSCPD}\PYG{o}{.}\PYG{n}{estimate\PYGZus{}T}\PYG{p}{(}\PYG{n}{CPDFit}\PYG{p}{,}\PYG{n}{KobCPD}\PYG{p}{,}\PYG{n}{KobCPD}\PYG{o}{.}\PYG{n}{ParamVector}\PYG{p}{(}\PYG{p}{)}\PYG{p}{,}\PYG{n}{Species}\PYG{p}{)}\PYG{p}{)}
            \PYG{n}{Solution}\PYG{o}{=}\PYG{n}{KobCPD}\PYG{o}{.}\PYG{n}{ConvertKinFactors}\PYG{p}{(}\PYG{n}{KobCPD}\PYG{o}{.}\PYG{n}{ParamVector}\PYG{p}{(}\PYG{p}{)}\PYG{p}{)}
            \PYG{c}{\PYGZsh{}Solution=Arr.ParamVector()}
            \PYG{n}{KobPCPD}\PYG{o}{.}\PYG{n}{setParamVector}\PYG{p}{(}\PYG{n}{Solution}\PYG{p}{)}
            \PYG{n}{KobPCPD}\PYG{o}{.}\PYG{n}{plot}\PYG{p}{(}\PYG{n}{CPDFit}\PYG{p}{,}\PYG{n}{Species}\PYG{p}{)}
            \PYG{n}{outfile}\PYG{o}{.}\PYG{n}{write}\PYG{p}{(}\PYG{n+nb}{str}\PYG{p}{(}\PYG{n}{CPDFit}\PYG{o}{.}\PYG{n}{SpeciesName}\PYG{p}{(}\PYG{n}{Species}\PYG{p}{)}\PYG{p}{)}\PYG{o}{+}\PYG{l+s}{'}\PYG{l+s+se}{\PYGZbs{}t}\PYG{l+s+se}{\PYGZbs{}t}\PYG{l+s}{'}\PYG{o}{+}\PYG{n+nb}{str}\PYG{p}{(}\PYG{n}{Solution}\PYG{p}{[}\PYG{l+m+mi}{0}\PYG{p}{]}\PYG{p}{)}\PYG{o}{+}\PYG{l+s}{'}\PYG{l+s+se}{\PYGZbs{}t}\PYG{l+s+se}{\PYGZbs{}t}\PYG{l+s}{'}\PYG{o}{+}\PYG{n+nb}{str}\PYG{p}{(}\PYG{n}{Solution}\PYG{p}{[}\PYG{l+m+mi}{1}\PYG{p}{]}\PYG{p}{)}\PYG{o}{+}\PYG{l+s}{'}\PYG{l+s+se}{\PYGZbs{}t}\PYG{l+s+se}{\PYGZbs{}t}\PYG{l+s}{'}\PYG{o}{+}\PYG{n+nb}{str}\PYG{p}{(}\PYG{n}{Solution}\PYG{p}{[}\PYG{l+m+mi}{2}\PYG{p}{]}\PYG{p}{)}\PYG{o}{+}\PYG{l+s}{'}\PYG{l+s+se}{\PYGZbs{}t}\PYG{l+s+se}{\PYGZbs{}t}\PYG{l+s}{'}\PYG{o}{+}\PYG{n+nb}{str}\PYG{p}{(}\PYG{n}{Solution}\PYG{p}{[}\PYG{l+m+mi}{3}\PYG{p}{]}\PYG{p}{)}\PYG{o}{+}\PYG{l+s}{'}\PYG{l+s+se}{\PYGZbs{}t}\PYG{l+s+se}{\PYGZbs{}t}\PYG{l+s}{'}\PYG{o}{+}\PYG{n+nb}{str}\PYG{p}{(}\PYG{n}{Solution}\PYG{p}{[}\PYG{l+m+mi}{4}\PYG{p}{]}\PYG{p}{)}\PYG{o}{+}\PYG{l+s}{'}\PYG{l+s+se}{\PYGZbs{}t}\PYG{l+s+se}{\PYGZbs{}t}\PYG{l+s}{'}\PYG{o}{+}\PYG{n+nb}{str}\PYG{p}{(}\PYG{n}{Solution}\PYG{p}{[}\PYG{l+m+mi}{5}\PYG{p}{]}\PYG{p}{)}\PYG{o}{+}\PYG{l+s}{'}\PYG{l+s+se}{\PYGZbs{}n}\PYG{l+s}{'}\PYG{p}{)}
            \PYG{c}{\PYGZsh{}for the comparison of the species sum with (1-Solid)}
            \PYG{k}{if} \PYG{n}{CPDFit}\PYG{o}{.}\PYG{n}{SpeciesName}\PYG{p}{(}\PYG{n}{Species}\PYG{p}{)}\PYG{o}{==}\PYG{l+s}{'}\PYG{l+s}{Solid}\PYG{l+s}{'}\PYG{p}{:}
                \PYG{n}{SolidYieldsCalc}\PYG{o}{+}\PYG{o}{=}\PYG{n}{KobPCPD}\PYG{o}{.}\PYG{n}{calcMass}\PYG{p}{(}\PYG{n}{CPDFit}\PYG{p}{,}\PYG{n}{CPDFit}\PYG{o}{.}\PYG{n}{Time}\PYG{p}{(}\PYG{p}{)}\PYG{p}{,}\PYG{n}{CPDFit}\PYG{o}{.}\PYG{n}{Interpolate}\PYG{p}{(}\PYG{l+s}{'}\PYG{l+s}{Temp}\PYG{l+s}{'}\PYG{p}{)}\PYG{p}{,}\PYG{n}{Species}\PYG{p}{)}
                \PYG{n}{SolidYieldsCPD}\PYG{o}{+}\PYG{o}{=}\PYG{n}{CPDFit}\PYG{o}{.}\PYG{n}{Yield}\PYG{p}{(}\PYG{n}{Species}\PYG{p}{)}
            \PYG{k}{elif} \PYG{n}{CPDFit}\PYG{o}{.}\PYG{n}{SpeciesName}\PYG{p}{(}\PYG{n}{Species}\PYG{p}{)}\PYG{o}{!=}\PYG{l+s}{'}\PYG{l+s}{Solid}\PYG{l+s}{'} \PYG{o+ow}{and} \PYG{n}{CPDFit}\PYG{o}{.}\PYG{n}{SpeciesName}\PYG{p}{(}\PYG{n}{Species}\PYG{p}{)}\PYG{o}{!=}\PYG{l+s}{'}\PYG{l+s}{Temp}\PYG{l+s}{'} \PYG{o+ow}{and} \PYG{n}{CPDFit}\PYG{o}{.}\PYG{n}{SpeciesName}\PYG{p}{(}\PYG{n}{Species}\PYG{p}{)}\PYG{o}{!=}\PYG{l+s}{'}\PYG{l+s}{Time}\PYG{l+s}{'} \PYG{o+ow}{and} \PYG{n}{CPDFit}\PYG{o}{.}\PYG{n}{SpeciesName}\PYG{p}{(}\PYG{n}{Species}\PYG{p}{)}\PYG{o}{!=}\PYG{l+s}{'}\PYG{l+s}{Gas}\PYG{l+s}{'} \PYG{o+ow}{and} \PYG{n}{CPDFit}\PYG{o}{.}\PYG{n}{SpeciesName}\PYG{p}{(}\PYG{n}{Species}\PYG{p}{)}\PYG{o}{!=}\PYG{l+s}{'}\PYG{l+s}{Total}\PYG{l+s}{'}\PYG{p}{:}
                \PYG{n}{SumSingleYieldsCalc}\PYG{o}{+}\PYG{o}{=}\PYG{n}{KobPCPD}\PYG{o}{.}\PYG{n}{calcMass}\PYG{p}{(}\PYG{n}{CPDFit}\PYG{p}{,}\PYG{n}{CPDFit}\PYG{o}{.}\PYG{n}{Time}\PYG{p}{(}\PYG{p}{)}\PYG{p}{,}\PYG{n}{CPDFit}\PYG{o}{.}\PYG{n}{Interpolate}\PYG{p}{(}\PYG{l+s}{'}\PYG{l+s}{Temp}\PYG{l+s}{'}\PYG{p}{)}\PYG{p}{,}\PYG{n}{Species}\PYG{p}{)}
                \PYG{n}{SumSingleYieldsCPD}\PYG{o}{+}\PYG{o}{=}\PYG{n}{CPDFit}\PYG{o}{.}\PYG{n}{Yield}\PYG{p}{(}\PYG{n}{Species}\PYG{p}{)}
        \PYG{n}{outfile}\PYG{o}{.}\PYG{n}{close}\PYG{p}{(}\PYG{p}{)}
        \PYG{n}{CPDFit}\PYG{o}{.}\PYG{n}{plt\PYGZus{}InputVectors}\PYG{p}{(}\PYG{n}{CPDFit}\PYG{o}{.}\PYG{n}{Time}\PYG{p}{(}\PYG{p}{)}\PYG{p}{,}\PYG{l+m+mf}{1.}\PYG{o}{-}\PYG{n}{SolidYieldsCalc}\PYG{p}{,}\PYG{l+m+mf}{1.}\PYG{o}{-}\PYG{n}{SolidYieldsCPD}\PYG{p}{,}\PYG{n}{SumSingleYieldsCalc}\PYG{p}{,}\PYG{n}{SumSingleYieldsCPD}\PYG{p}{,}\PYG{l+s}{'}\PYG{l+s}{1-Solid; fitted}\PYG{l+s}{'}\PYG{p}{,}\PYG{l+s}{'}\PYG{l+s}{1-Solid; CPD output}\PYG{l+s}{'}\PYG{p}{,}\PYG{l+s}{'}\PYG{l+s}{Sum Yields; fitted}\PYG{l+s}{'}\PYG{p}{,}\PYG{l+s}{'}\PYG{l+s}{Sum Yields; CPD output}\PYG{l+s}{'}\PYG{p}{)}
    \PYG{c}{\PYGZsh{}\PYGZsh{}SPECIES AND ENERGY BALANCE:}
    \PYG{n}{SpecCPD}\PYG{o}{=}\PYG{n}{Compos\PYGZus{}and\PYGZus{}Energy}\PYG{o}{.}\PYG{n}{CPD\PYGZus{}SpeciesBalance}\PYG{p}{(}\PYG{n}{CPDFile}\PYG{p}{,}\PYG{n}{UAC}\PYG{p}{,}\PYG{n}{UAH}\PYG{p}{,}\PYG{n}{UAN}\PYG{p}{,}\PYG{n}{UAO}\PYG{p}{,}\PYG{n}{PAVM\PYGZus{}asrec}\PYG{p}{,}\PYG{n}{PAFC\PYGZus{}asrec}\PYG{p}{,}\PYG{n}{HHV}\PYG{p}{,}\PYG{n}{MTar}\PYG{p}{)}
\PYG{c}{\PYGZsh{}}
\PYG{c}{\PYGZsh{}}
\PYG{c}{\PYGZsh{}}
\PYG{c}{\PYGZsh{}\PYGZsh{}\PYGZsh{}\PYGZsh{}FG-DVC\PYGZsh{}\PYGZsh{}\PYGZsh{}\PYGZsh{}}
\PYG{k}{if} \PYG{n}{FG\PYGZus{}select}\PYG{o}{==}\PYG{n+nb+bp}{True}\PYG{p}{:}
    \PYG{c}{\PYGZsh{}writes Time-Temperature file}
    \PYG{n}{FG\PYGZus{}TimeTemp}\PYG{o}{=}\PYG{n}{CPD\PYGZus{}TimeTemp}
    \PYG{n}{FG\PYGZus{}TimeTemp}\PYG{p}{[}\PYG{p}{:}\PYG{p}{,}\PYG{l+m+mi}{0}\PYG{p}{]}\PYG{o}{=}\PYG{n}{CPD\PYGZus{}TimeTemp}\PYG{p}{[}\PYG{p}{:}\PYG{p}{,}\PYG{l+m+mi}{0}\PYG{p}{]}\PYG{o}{*}\PYG{l+m+mf}{1.e-3}
    \PYG{n}{OpCondInp}\PYG{o}{.}\PYG{n}{writeFGDVCtTHist}\PYG{p}{(}\PYG{n}{FG\PYGZus{}TimeTemp}\PYG{p}{,}\PYG{n}{FG\PYGZus{}dt}\PYG{p}{,}\PYG{n}{FG\PYGZus{}T\PYGZus{}t\PYGZus{}History}\PYG{p}{)}
    \PYG{c}{\PYGZsh{}initialize the launching object}
    \PYG{n}{FGDVC}\PYG{o}{=}\PYG{n}{FGDVC\PYGZus{}Fit\PYGZus{}lin\PYGZus{}regr}\PYG{o}{.}\PYG{n}{SetterAndLauncher}\PYG{p}{(}\PYG{p}{)}
    \PYG{c}{\PYGZsh{}set and writes Coal Files:}
    \PYG{k}{if} \PYG{n}{FG\PYGZus{}CoalSelection}\PYG{o}{==}\PYG{l+m+mi}{0}\PYG{p}{:}
        \PYG{c}{\PYGZsh{}deletes old generated file}
        \PYG{n}{os}\PYG{o}{.}\PYG{n}{system}\PYG{p}{(}\PYG{l+s}{'}\PYG{l+s}{cd }\PYG{l+s}{'}\PYG{o}{+}\PYG{n}{FG\PYGZus{}MainDir}\PYG{o}{+}\PYG{n}{FG\PYGZus{}GenCoalDir}\PYG{o}{+}\PYG{l+s}{'}\PYG{l+s}{ \& del }\PYG{l+s}{'}\PYG{o}{+}\PYG{n}{FG\PYGZus{}CoalName}\PYG{o}{+}\PYG{l+s}{'}\PYG{l+s}{\PYGZus{}com.dat, }\PYG{l+s}{'}\PYG{o}{+}\PYG{n}{FG\PYGZus{}CoalName}\PYG{o}{+}\PYG{l+s}{'}\PYG{l+s}{\PYGZus{}kin.dat, }\PYG{l+s}{'}\PYG{o}{+}\PYG{n}{FG\PYGZus{}CoalName}\PYG{o}{+}\PYG{l+s}{'}\PYG{l+s}{\PYGZus{}pol.dat}\PYG{l+s}{'}\PYG{p}{)}
        \PYG{c}{\PYGZsh{}generates coalsd.exe input file}
        \PYG{n}{MakeCoalGenFile}\PYG{o}{=}\PYG{n}{ReadInputFiles}\PYG{o}{.}\PYG{n}{WriteFGDVCCoalFile}\PYG{p}{(}\PYG{n}{FG\PYGZus{}CoalGenFileName}\PYG{p}{)}
        \PYG{n}{MakeCoalGenFile}\PYG{o}{.}\PYG{n}{setCoalComp}\PYG{p}{(}\PYG{n}{UAC}\PYG{p}{,}\PYG{n}{UAH}\PYG{p}{,}\PYG{n}{UAO}\PYG{p}{,}\PYG{n}{UAN}\PYG{p}{,}\PYG{p}{(}\PYG{l+m+mf}{100.}\PYG{o}{-}\PYG{n}{UAC}\PYG{o}{-}\PYG{n}{UAH}\PYG{o}{-}\PYG{n}{UAO}\PYG{o}{-}\PYG{n}{UAN}\PYG{p}{)}\PYG{p}{,}\PYG{l+m+mi}{0}\PYG{p}{)}
        \PYG{n}{MakeCoalGenFile}\PYG{o}{.}\PYG{n}{write}\PYG{p}{(}\PYG{n}{FG\PYGZus{}MainDir}\PYG{o}{+}\PYG{n}{FG\PYGZus{}GenCoalDir}\PYG{o}{+}\PYG{l+s}{'}\PYG{l+s+se}{\PYGZbs{}\PYGZbs{}}\PYG{l+s}{'}\PYG{p}{,}\PYG{n}{FG\PYGZus{}CoalName}\PYG{p}{)}
        \PYG{c}{\PYGZsh{}makes new file}
        \PYG{n}{os}\PYG{o}{.}\PYG{n}{system}\PYG{p}{(}\PYG{l+s}{'}\PYG{l+s}{cd }\PYG{l+s}{'}\PYG{o}{+}\PYG{n}{FG\PYGZus{}MainDir}\PYG{o}{+}\PYG{n}{FG\PYGZus{}GenCoalDir}\PYG{o}{+}\PYG{l+s}{'}\PYG{l+s}{ \& }\PYG{l+s}{'}\PYG{o}{+}\PYG{l+s}{'}\PYG{l+s}{coalsd.exe \textless{} }\PYG{l+s}{'}\PYG{o}{+}\PYG{n}{FG\PYGZus{}CoalGenFileName}\PYG{p}{)}
        \PYG{c}{\PYGZsh{}tests weather the coal file was genearated:}
        \PYG{k}{if} \PYG{n}{os}\PYG{o}{.}\PYG{n}{path}\PYG{o}{.}\PYG{n}{exists}\PYG{p}{(}\PYG{n}{FG\PYGZus{}MainDir}\PYG{o}{+}\PYG{l+s}{'}\PYG{l+s+se}{\PYGZbs{}\PYGZbs{}}\PYG{l+s}{'}\PYG{o}{+}\PYG{n}{FG\PYGZus{}GenCoalDir}\PYG{o}{+}\PYG{l+s}{'}\PYG{l+s+se}{\PYGZbs{}\PYGZbs{}}\PYG{l+s}{'}\PYG{o}{+}\PYG{n}{FG\PYGZus{}CoalName}\PYG{o}{+}\PYG{l+s}{'}\PYG{l+s}{\PYGZus{}com.dat}\PYG{l+s}{'}\PYG{p}{)}\PYG{o}{==}\PYG{n+nb+bp}{False}\PYG{p}{:}
            \PYG{k}{print} \PYG{l+m+mi}{30}\PYG{o}{*}\PYG{l+s}{'}\PYG{l+s}{*}\PYG{l+s}{'}\PYG{p}{,}\PYG{l+s}{'}\PYG{l+s+se}{\PYGZbs{}n}\PYG{l+s}{'}\PYG{p}{,}\PYG{l+s}{'}\PYG{l+s}{The coal is may outside the libraries coals. Select manually the closest library coal.}\PYG{l+s}{'}\PYG{p}{,}\PYG{l+m+mi}{30}\PYG{o}{*}\PYG{l+s}{'}\PYG{l+s}{*}\PYG{l+s}{'}\PYG{p}{,}\PYG{l+s}{'}\PYG{l+s+se}{\PYGZbs{}n}\PYG{l+s}{'}
        \PYG{c}{\PYGZsh{}sets generated file for instruct.ini}
        \PYG{n}{FGDVC}\PYG{o}{.}\PYG{n}{set1CoalLocation}\PYG{p}{(}\PYG{n}{FG\PYGZus{}MainDir}\PYG{o}{+}\PYG{n}{FG\PYGZus{}GenCoalDir}\PYG{o}{+}\PYG{l+s}{'}\PYG{l+s+se}{\PYGZbs{}\PYGZbs{}}\PYG{l+s}{'}\PYG{o}{+}\PYG{n}{FG\PYGZus{}CoalName}\PYG{o}{+}\PYG{l+s}{'}\PYG{l+s}{\PYGZus{}com.dat}\PYG{l+s}{'}\PYG{p}{)}
        \PYG{n}{FGDVC}\PYG{o}{.}\PYG{n}{set2KinLocation}\PYG{p}{(}\PYG{n}{FG\PYGZus{}MainDir}\PYG{o}{+}\PYG{n}{FG\PYGZus{}GenCoalDir}\PYG{o}{+}\PYG{l+s}{'}\PYG{l+s+se}{\PYGZbs{}\PYGZbs{}}\PYG{l+s}{'}\PYG{o}{+}\PYG{n}{FG\PYGZus{}CoalName}\PYG{o}{+}\PYG{l+s}{'}\PYG{l+s}{\PYGZus{}kin.dat}\PYG{l+s}{'}\PYG{p}{)}
        \PYG{n}{FGDVC}\PYG{o}{.}\PYG{n}{set3PolyLocation}\PYG{p}{(}\PYG{n}{FG\PYGZus{}MainDir}\PYG{o}{+}\PYG{n}{FG\PYGZus{}GenCoalDir}\PYG{o}{+}\PYG{l+s}{'}\PYG{l+s+se}{\PYGZbs{}\PYGZbs{}}\PYG{l+s}{'}\PYG{o}{+}\PYG{n}{FG\PYGZus{}CoalName}\PYG{o}{+}\PYG{l+s}{'}\PYG{l+s}{\PYGZus{}pol.dat}\PYG{l+s}{'}\PYG{p}{)}
    \PYG{k}{elif} \PYG{n}{FG\PYGZus{}CoalSelection}\PYG{o}{\textgreater{}}\PYG{l+m+mi}{0} \PYG{o+ow}{and} \PYG{n}{FG\PYGZus{}CoalSelection}\PYG{o}{\textless{}}\PYG{l+m+mi}{9}\PYG{p}{:}
        \PYG{c}{\PYGZsh{}sets library file for instruct.ini}
        \PYG{n}{FGDVC}\PYG{o}{.}\PYG{n}{set1CoalLocation}\PYG{p}{(}\PYG{n}{FG\PYGZus{}MainDir}\PYG{o}{+}\PYG{n}{FG\PYGZus{}LibCoalDir}\PYG{o}{+}\PYG{l+s}{'}\PYG{l+s+se}{\PYGZbs{}\PYGZbs{}}\PYG{l+s}{coal.ar}\PYG{l+s}{'}\PYG{o}{+}\PYG{n+nb}{str}\PYG{p}{(}\PYG{n}{FG\PYGZus{}CoalSelection}\PYG{p}{)}\PYG{p}{)}
        \PYG{n}{FGDVC}\PYG{o}{.}\PYG{n}{set2KinLocation}\PYG{p}{(}\PYG{n}{FG\PYGZus{}MainDir}\PYG{o}{+}\PYG{n}{FG\PYGZus{}LibCoalDir}\PYG{o}{+}\PYG{l+s}{'}\PYG{l+s+se}{\PYGZbs{}\PYGZbs{}}\PYG{l+s}{kin.ar}\PYG{l+s}{'}\PYG{o}{+}\PYG{n+nb}{str}\PYG{p}{(}\PYG{n}{FG\PYGZus{}CoalSelection}\PYG{p}{)}\PYG{p}{)}
        \PYG{n}{FGDVC}\PYG{o}{.}\PYG{n}{set3PolyLocation}\PYG{p}{(}\PYG{n}{FG\PYGZus{}MainDir}\PYG{o}{+}\PYG{n}{FG\PYGZus{}LibCoalDir}\PYG{o}{+}\PYG{l+s}{'}\PYG{l+s+se}{\PYGZbs{}\PYGZbs{}}\PYG{l+s}{polymr.ar}\PYG{l+s}{'}\PYG{o}{+}\PYG{n+nb}{str}\PYG{p}{(}\PYG{n}{FG\PYGZus{}CoalSelection}\PYG{p}{)}\PYG{p}{)}
    \PYG{k}{else}\PYG{p}{:}
        \PYG{k}{print} \PYG{l+s}{"}\PYG{l+s}{select Choose Coal: 0 interpolate between library coals and generate own coal. Set 1 to 8 for a library coal.}\PYG{l+s}{'}\PYG{l+s}{ in FGDVC.inp equal a value between 0 and 8}\PYG{l+s}{"}
    \PYG{c}{\PYGZsh{}sets FG-DVC instruct.ini parameter}
    \PYG{n}{FGDVC}\PYG{o}{.}\PYG{n}{set5Pressure}\PYG{p}{(}\PYG{n}{FG\PYGZus{}pressure}\PYG{p}{)}
    \PYG{k}{if} \PYG{n}{FG\PYGZus{}TarCacking}\PYG{o}{==}\PYG{l+m+mf}{0.0}\PYG{p}{:}            \PYG{c}{\PYGZsh{}case: no tar cracking}
        \PYG{n}{FGDVC}\PYG{o}{.}\PYG{n}{set6Theorie}\PYG{p}{(}\PYG{l+m+mi}{13}\PYG{p}{,}\PYG{l+m+mf}{0.0}\PYG{p}{)}
    \PYG{k}{elif} \PYG{n}{FG\PYGZus{}TarCacking}\PYG{o}{\textless{}}\PYG{l+m+mf}{0.0}\PYG{p}{:}           \PYG{c}{\PYGZsh{}case: full tar cracking}
        \PYG{n}{FGDVC}\PYG{o}{.}\PYG{n}{set6Theorie}\PYG{p}{(}\PYG{l+m+mi}{15}\PYG{p}{,}\PYG{l+m+mf}{0.0}\PYG{p}{)}
    \PYG{k}{else}\PYG{p}{:}                             \PYG{c}{\PYGZsh{}case: partial tar cracking}
        \PYG{n}{FGDVC}\PYG{o}{.}\PYG{n}{set6Theorie}\PYG{p}{(}\PYG{l+m+mi}{13}\PYG{p}{,}\PYG{n+nb}{float}\PYG{p}{(}\PYG{n}{FG\PYGZus{}TarCacking}\PYG{p}{)}\PYG{p}{)}
    \PYG{n}{FGDVC}\PYG{o}{.}\PYG{n}{set7File}\PYG{p}{(}\PYG{n}{FG\PYGZus{}T\PYGZus{}t\PYGZus{}History}\PYG{p}{)}
    \PYG{n}{FGDVC}\PYG{o}{.}\PYG{n}{set9AshMoisture}\PYG{p}{(}\PYG{l+m+mf}{0.0}\PYG{p}{,}\PYG{l+m+mf}{0.0}\PYG{p}{)}
    \PYG{n}{FGDVC}\PYG{o}{.}\PYG{n}{setTRamp\PYGZus{}or\PYGZus{}TFile}\PYG{p}{(}\PYG{l+s}{'}\PYG{l+s}{File}\PYG{l+s}{'}\PYG{p}{)} \PYG{c}{\PYGZsh{}case: models temperature history with the file}
    \PYG{c}{\PYGZsh{}writes the instruct.ini and launches FG-DVC (no graphical user interface, only main file fgdvcd.exe)}
    \PYG{n}{FGDVC}\PYG{o}{.}\PYG{n}{writeInstructFile}\PYG{p}{(}\PYG{n}{FG\PYGZus{}MainDir}\PYG{o}{+}\PYG{l+s}{'}\PYG{l+s+se}{\PYGZbs{}\PYGZbs{}}\PYG{l+s}{'}\PYG{o}{+}\PYG{n}{FG\PYGZus{}ExeCoalDir}\PYG{o}{+}\PYG{l+s}{'}\PYG{l+s+se}{\PYGZbs{}\PYGZbs{}}\PYG{l+s}{'}\PYG{p}{)}
    \PYG{n}{FGDVC}\PYG{o}{.}\PYG{n}{Run}\PYG{p}{(}\PYG{l+s}{'}\PYG{l+s}{cd }\PYG{l+s}{'}\PYG{o}{+}\PYG{n}{FG\PYGZus{}MainDir}\PYG{o}{+}\PYG{n}{FG\PYGZus{}ExeCoalDir}\PYG{o}{+}\PYG{l+s}{'}\PYG{l+s}{ \& }\PYG{l+s}{'}\PYG{o}{+}\PYG{l+s}{'}\PYG{l+s}{fgdvcd.exe}\PYG{l+s}{'}\PYG{p}{)}
    \PYG{c}{\PYGZsh{}}
    \PYG{c}{\PYGZsh{}\PYGZsh{}\PYGZsh{}calibrate kinetic parameter:}
    \PYG{c}{\PYGZsh{}read result:}
    \PYG{n}{FGFile}\PYG{o}{=}\PYG{n}{Fit\PYGZus{}one\PYGZus{}run}\PYG{o}{.}\PYG{n}{FGDVC\PYGZus{}Result}\PYG{p}{(}\PYG{n}{FG\PYGZus{}DirOut}\PYG{p}{)}
    \PYG{c}{\PYGZsh{} creates object, required for fitting procedures}
    \PYG{n}{FGFit}\PYG{o}{=}\PYG{n}{Fit\PYGZus{}one\PYGZus{}run}\PYG{o}{.}\PYG{n}{Fit\PYGZus{}one\PYGZus{}run}\PYG{p}{(}\PYG{n}{FGFile}\PYG{p}{)}
    \PYG{c}{\PYGZsh{}Array for the comparison of the sum of the individual species(t) with the (1-solid(t))}
    \PYG{n}{SumSingleYieldsCalc}\PYG{o}{=}\PYG{n}{np}\PYG{o}{.}\PYG{n}{zeros}\PYG{p}{(}\PYG{n+nb}{len}\PYG{p}{(}\PYG{n}{FGFit}\PYG{o}{.}\PYG{n}{Yield}\PYG{p}{(}\PYG{l+s}{'}\PYG{l+s}{Time}\PYG{l+s}{'}\PYG{p}{)}\PYG{p}{)}\PYG{p}{)} \PYG{c}{\PYGZsh{}Array initialized for the sum of the single yields (calculated)}
    \PYG{n}{SumSingleYieldsFG}\PYG{o}{=}\PYG{n}{np}\PYG{o}{.}\PYG{n}{zeros}\PYG{p}{(}\PYG{n+nb}{len}\PYG{p}{(}\PYG{n}{FGFit}\PYG{o}{.}\PYG{n}{Yield}\PYG{p}{(}\PYG{l+s}{'}\PYG{l+s}{Time}\PYG{l+s}{'}\PYG{p}{)}\PYG{p}{)}\PYG{p}{)}   \PYG{c}{\PYGZsh{}Array initialized for the sum of the single yields (CPD output)}
    \PYG{n}{SolidYieldsCalc}\PYG{o}{=}\PYG{n}{np}\PYG{o}{.}\PYG{n}{zeros}\PYG{p}{(}\PYG{n+nb}{len}\PYG{p}{(}\PYG{n}{FGFit}\PYG{o}{.}\PYG{n}{Yield}\PYG{p}{(}\PYG{l+s}{'}\PYG{l+s}{Time}\PYG{l+s}{'}\PYG{p}{)}\PYG{p}{)}\PYG{p}{)}     \PYG{c}{\PYGZsh{}Array initialized for the yields of the Solids (calculated)}
    \PYG{n}{SolidYieldsFG}\PYG{o}{=}\PYG{n}{np}\PYG{o}{.}\PYG{n}{zeros}\PYG{p}{(}\PYG{n+nb}{len}\PYG{p}{(}\PYG{n}{FGFit}\PYG{o}{.}\PYG{n}{Yield}\PYG{p}{(}\PYG{l+s}{'}\PYG{l+s}{Time}\PYG{l+s}{'}\PYG{p}{)}\PYG{p}{)}\PYG{p}{)}       \PYG{c}{\PYGZsh{}Array initialized for the yields of the Solids (CPD output)}
    \PYG{c}{\PYGZsh{}\PYGZsh{}CONSTANT RATE}
    \PYG{k}{if} \PYG{n}{FG\PYGZus{}FittingKineticParameter\PYGZus{}Select}\PYG{o}{==}\PYG{l+s}{'}\PYG{l+s}{constantRate}\PYG{l+s}{'}\PYG{p}{:}
        \PYG{n}{PredictionVector}\PYG{o}{=}\PYG{p}{[}\PYG{l+m+mi}{50}\PYG{p}{,}\PYG{l+m+mf}{0.01}\PYG{p}{]} \PYG{c}{\PYGZsh{}first argument is k, second is t\PYGZus{}start}
        \PYG{n}{LSFG}\PYG{o}{=}\PYG{n}{Fit\PYGZus{}one\PYGZus{}run}\PYG{o}{.}\PYG{n}{LeastSquarsEstimator}\PYG{p}{(}\PYG{p}{)}
        \PYG{n}{LSFG}\PYG{o}{.}\PYG{n}{setOptimizer}\PYG{p}{(}\PYG{l+s}{'}\PYG{l+s}{fmin}\PYG{l+s}{'}\PYG{p}{)}
        \PYG{n}{LSFG}\PYG{o}{.}\PYG{n}{setTolerance}\PYG{p}{(}\PYG{l+m+mf}{1.e-18}\PYG{p}{)}
        \PYG{n}{LSFG}\PYG{o}{.}\PYG{n}{setWeights}\PYG{p}{(}\PYG{n}{WeightY}\PYG{p}{,}\PYG{n}{WeightR}\PYG{p}{)}
        \PYG{n}{CRFG}\PYG{o}{=}\PYG{n}{Fit\PYGZus{}one\PYGZus{}run}\PYG{o}{.}\PYG{n}{ConstantRateModel}\PYG{p}{(}\PYG{n}{PredictionVector}\PYG{p}{)}
        \PYG{n}{outfile} \PYG{o}{=} \PYG{n+nb}{open}\PYG{p}{(}\PYG{l+s}{'}\PYG{l+s}{FGDVC-Results\PYGZus{}const\PYGZus{}rate.txt}\PYG{l+s}{'}\PYG{p}{,} \PYG{l+s}{'}\PYG{l+s}{w}\PYG{l+s}{'}\PYG{p}{)}
        \PYG{n}{outfile}\PYG{o}{.}\PYG{n}{write}\PYG{p}{(}\PYG{l+s}{"}\PYG{l+s}{Species}\PYG{l+s+se}{\PYGZbs{}t}\PYG{l+s+se}{\PYGZbs{}t}\PYG{l+s}{k [1/s]}\PYG{l+s+se}{\PYGZbs{}t}\PYG{l+s+se}{\PYGZbs{}t}\PYG{l+s}{t\PYGZus{}start [s]}\PYG{l+s+se}{\PYGZbs{}n}\PYG{l+s+se}{\PYGZbs{}n}\PYG{l+s}{"}\PYG{p}{)}
        \PYG{c}{\PYGZsh{}}
        \PYG{k}{for} \PYG{n}{Spec} \PYG{o+ow}{in} \PYG{n+nb}{range}\PYG{p}{(}\PYG{l+m+mi}{2}\PYG{p}{,}\PYG{n+nb}{len}\PYG{p}{(}\PYG{n}{FGFit}\PYG{o}{.}\PYG{n}{SpeciesNames}\PYG{p}{(}\PYG{p}{)}\PYG{p}{)}\PYG{p}{,}\PYG{l+m+mi}{1}\PYG{p}{)}\PYG{p}{:}
            \PYG{n}{CRFG}\PYG{o}{.}\PYG{n}{setParamVector}\PYG{p}{(}\PYG{n}{PredictionVector}\PYG{p}{)}
            \PYG{n}{CRFG}\PYG{o}{.}\PYG{n}{setParamVector}\PYG{p}{(}\PYG{n}{LSFG}\PYG{o}{.}\PYG{n}{estimate\PYGZus{}T}\PYG{p}{(}\PYG{n}{FGFit}\PYG{p}{,}\PYG{n}{CRFG}\PYG{p}{,}\PYG{n}{PredictionVector}\PYG{p}{,}\PYG{n}{Spec}\PYG{p}{)}\PYG{p}{)}
            \PYG{n}{CRFG}\PYG{o}{.}\PYG{n}{plot}\PYG{p}{(}\PYG{n}{FGFit}\PYG{p}{,}\PYG{n}{Spec}\PYG{p}{)}
            \PYG{n}{Solution}\PYG{o}{=}\PYG{n}{CRFG}\PYG{o}{.}\PYG{n}{ParamVector}\PYG{p}{(}\PYG{p}{)}
            \PYG{n}{outfile}\PYG{o}{.}\PYG{n}{write}\PYG{p}{(}\PYG{n+nb}{str}\PYG{p}{(}\PYG{n}{FGFit}\PYG{o}{.}\PYG{n}{SpeciesName}\PYG{p}{(}\PYG{n}{Spec}\PYG{p}{)}\PYG{p}{)}\PYG{o}{+}\PYG{l+s}{'}\PYG{l+s+se}{\PYGZbs{}t}\PYG{l+s}{'}\PYG{o}{+}\PYG{n+nb}{str}\PYG{p}{(}\PYG{n}{Solution}\PYG{p}{[}\PYG{l+m+mi}{0}\PYG{p}{]}\PYG{p}{)}\PYG{o}{+}\PYG{l+s}{'}\PYG{l+s+se}{\PYGZbs{}t}\PYG{l+s}{'}\PYG{o}{+}\PYG{n+nb}{str}\PYG{p}{(}\PYG{n}{Solution}\PYG{p}{[}\PYG{l+m+mi}{1}\PYG{p}{]}\PYG{p}{)}\PYG{o}{+}\PYG{l+s}{'}\PYG{l+s+se}{\PYGZbs{}n}\PYG{l+s}{'}\PYG{p}{)}
            \PYG{c}{\PYGZsh{}for the comparison of the species sum with (1-Solid)}
            \PYG{k}{if} \PYG{n}{FGFit}\PYG{o}{.}\PYG{n}{SpeciesName}\PYG{p}{(}\PYG{n}{Spec}\PYG{p}{)}\PYG{o}{==}\PYG{l+s}{'}\PYG{l+s}{Solid}\PYG{l+s}{'}\PYG{p}{:}
                \PYG{n}{SolidYieldsCalc}\PYG{o}{+}\PYG{o}{=}\PYG{n}{CRFG}\PYG{o}{.}\PYG{n}{calcMass}\PYG{p}{(}\PYG{n}{FGFit}\PYG{p}{,}\PYG{n}{FGFit}\PYG{o}{.}\PYG{n}{Time}\PYG{p}{(}\PYG{p}{)}\PYG{p}{,}\PYG{n}{FGFit}\PYG{o}{.}\PYG{n}{Interpolate}\PYG{p}{(}\PYG{l+s}{'}\PYG{l+s}{Temp}\PYG{l+s}{'}\PYG{p}{)}\PYG{p}{,}\PYG{n}{Spec}\PYG{p}{)}
                \PYG{n}{SolidYieldsFG}\PYG{o}{+}\PYG{o}{=}\PYG{n}{FGFit}\PYG{o}{.}\PYG{n}{Yield}\PYG{p}{(}\PYG{n}{Spec}\PYG{p}{)}
            \PYG{k}{elif} \PYG{n}{FGFit}\PYG{o}{.}\PYG{n}{SpeciesName}\PYG{p}{(}\PYG{n}{Spec}\PYG{p}{)}\PYG{o}{!=}\PYG{l+s}{'}\PYG{l+s}{Solid}\PYG{l+s}{'} \PYG{o+ow}{and} \PYG{n}{FGFit}\PYG{o}{.}\PYG{n}{SpeciesName}\PYG{p}{(}\PYG{n}{Spec}\PYG{p}{)}\PYG{o}{!=}\PYG{l+s}{'}\PYG{l+s}{Temp}\PYG{l+s}{'} \PYG{o+ow}{and} \PYG{n}{FGFit}\PYG{o}{.}\PYG{n}{SpeciesName}\PYG{p}{(}\PYG{n}{Spec}\PYG{p}{)}\PYG{o}{!=}\PYG{l+s}{'}\PYG{l+s}{Time}\PYG{l+s}{'}\PYG{p}{:}
                \PYG{n}{SumSingleYieldsCalc}\PYG{o}{+}\PYG{o}{=}\PYG{n}{CRFG}\PYG{o}{.}\PYG{n}{calcMass}\PYG{p}{(}\PYG{n}{FGFit}\PYG{p}{,}\PYG{n}{FGFit}\PYG{o}{.}\PYG{n}{Time}\PYG{p}{(}\PYG{p}{)}\PYG{p}{,}\PYG{n}{FGFit}\PYG{o}{.}\PYG{n}{Interpolate}\PYG{p}{(}\PYG{l+s}{'}\PYG{l+s}{Temp}\PYG{l+s}{'}\PYG{p}{)}\PYG{p}{,}\PYG{n}{Spec}\PYG{p}{)}
                \PYG{n}{SumSingleYieldsFG}\PYG{o}{+}\PYG{o}{=}\PYG{n}{FGFit}\PYG{o}{.}\PYG{n}{Yield}\PYG{p}{(}\PYG{n}{Spec}\PYG{p}{)}
        \PYG{n}{outfile}\PYG{o}{.}\PYG{n}{close}\PYG{p}{(}\PYG{p}{)}
        \PYG{n}{FGFit}\PYG{o}{.}\PYG{n}{plt\PYGZus{}InputVectors}\PYG{p}{(}\PYG{n}{FGFit}\PYG{o}{.}\PYG{n}{Time}\PYG{p}{(}\PYG{p}{)}\PYG{p}{,}\PYG{l+m+mf}{100.}\PYG{o}{-}\PYG{n}{SolidYieldsCalc}\PYG{p}{,}\PYG{l+m+mf}{100.}\PYG{o}{-}\PYG{n}{SolidYieldsFG}\PYG{p}{,}\PYG{n}{SumSingleYieldsCalc}\PYG{p}{,}\PYG{n}{SumSingleYieldsFG}\PYG{p}{,}\PYG{l+s}{'}\PYG{l+s}{1-Solid; fitted}\PYG{l+s}{'}\PYG{p}{,}\PYG{l+s}{'}\PYG{l+s}{1-Solid; FG-DVC output}\PYG{l+s}{'}\PYG{p}{,}\PYG{l+s}{'}\PYG{l+s}{Sum Yields; fitted}\PYG{l+s}{'}\PYG{p}{,}\PYG{l+s}{'}\PYG{l+s}{Sum Yields; FG-DVC output}\PYG{l+s}{'}\PYG{p}{)}
    \PYG{c}{\PYGZsh{}\PYGZsh{}ARRHENIUS RATE}
    \PYG{k}{if} \PYG{n}{FG\PYGZus{}FittingKineticParameter\PYGZus{}Select}\PYG{o}{==}\PYG{l+s}{'}\PYG{l+s}{Arrhenius}\PYG{l+s}{'}\PYG{p}{:}
        \PYG{n}{PredictionV0}\PYG{o}{=}\PYG{p}{[}\PYG{l+m+mf}{0.86e15}\PYG{p}{,}\PYG{l+m+mi}{0}\PYG{p}{,}\PYG{l+m+mi}{27700}\PYG{p}{]}  \PYG{c}{\PYGZsh{}for Standard Arrhenius}
        \PYG{n}{PredictionV1}\PYG{o}{=}\PYG{p}{[}\PYG{l+m+mf}{10.}\PYG{p}{,}\PYG{o}{-}\PYG{l+m+mf}{20.}\PYG{p}{]}         \PYG{c}{\PYGZsh{}for Arrhenius notation \PYGZsh{}1}
        \PYG{n}{PredictionV2}\PYG{o}{=}\PYG{p}{[}\PYG{l+m+mi}{10}\PYG{p}{,}\PYG{o}{-}\PYG{l+m+mi}{18}\PYG{p}{]}           \PYG{c}{\PYGZsh{}for Arrhenius notation \PYGZsh{}2}
        \PYG{n}{LSFG}\PYG{o}{=}\PYG{n}{Fit\PYGZus{}one\PYGZus{}run}\PYG{o}{.}\PYG{n}{LeastSquarsEstimator}\PYG{p}{(}\PYG{p}{)}
        \PYG{n}{LSFG}\PYG{o}{.}\PYG{n}{setOptimizer}\PYG{p}{(}\PYG{l+s}{'}\PYG{l+s}{fmin}\PYG{l+s}{'}\PYG{p}{)}\PYG{c}{\PYGZsh{}('leastsq')   \PYGZsh{} 'leastsq' often faster, but if this does not work: 'fmin' is more reliable}
        \PYG{n}{LSFG}\PYG{o}{.}\PYG{n}{setTolerance}\PYG{p}{(}\PYG{l+m+mf}{1.e-10}\PYG{p}{)}
        \PYG{n}{LSFG}\PYG{o}{.}\PYG{n}{setWeights}\PYG{p}{(}\PYG{n}{WeightY}\PYG{p}{,}\PYG{n}{WeightR}\PYG{p}{)}
        \PYG{n}{outfile} \PYG{o}{=} \PYG{n+nb}{open}\PYG{p}{(}\PYG{l+s}{'}\PYG{l+s}{FGDVC-Results\PYGZus{}ArrheniusRate.txt}\PYG{l+s}{'}\PYG{p}{,} \PYG{l+s}{'}\PYG{l+s}{w}\PYG{l+s}{'}\PYG{p}{)}
        \PYG{n}{outfile}\PYG{o}{.}\PYG{n}{write}\PYG{p}{(}\PYG{l+s}{"}\PYG{l+s}{Species}\PYG{l+s+se}{\PYGZbs{}t}\PYG{l+s+se}{\PYGZbs{}t}\PYG{l+s}{A [1/s]}\PYG{l+s+se}{\PYGZbs{}t}\PYG{l+s+se}{\PYGZbs{}t}\PYG{l+s}{b}\PYG{l+s+se}{\PYGZbs{}t}\PYG{l+s+se}{\PYGZbs{}t}\PYG{l+s}{E\PYGZus{}a [K]}\PYG{l+s+se}{\PYGZbs{}n}\PYG{l+s+se}{\PYGZbs{}n}\PYG{l+s}{"}\PYG{p}{)}
        \PYG{c}{\PYGZsh{}select one of the follwoing notations: }
        \PYG{c}{\PYGZsh{}Arr=Fit\PYGZus{}one\PYGZus{}run.ArrheniusModel(PredictionV0)}
        \PYG{c}{\PYGZsh{}Arr=Fit\PYGZus{}one\PYGZus{}run.ArrheniusModelAlternativeNotation1(PredictionV1)}
        \PYG{n}{ArrFG}\PYG{o}{=}\PYG{n}{Fit\PYGZus{}one\PYGZus{}run}\PYG{o}{.}\PYG{n}{ArrheniusModelAlternativeNotation2}\PYG{p}{(}\PYG{n}{FGFit}\PYG{p}{,}\PYG{n}{PredictionV2}\PYG{p}{)}
        \PYG{c}{\PYGZsh{}\PYGZsh{}\PYGZsh{}\PYGZsh{}\PYGZsh{}\PYGZsh{}\PYGZsh{}}
        \PYG{c}{\PYGZsh{}uses a separate Arrhenius model to plot, to ensure that the converted (!) output vector is right}
        \PYG{n}{ArrPFG}\PYG{o}{=}\PYG{n}{Fit\PYGZus{}one\PYGZus{}run}\PYG{o}{.}\PYG{n}{ArrheniusModel}\PYG{p}{(}\PYG{p}{[}\PYG{l+m+mi}{0}\PYG{p}{,}\PYG{l+m+mi}{0}\PYG{p}{,}\PYG{l+m+mi}{0}\PYG{p}{]}\PYG{p}{)}
        \PYG{c}{\PYGZsh{}\PYGZsh{}The single species:}
        \PYG{k}{for} \PYG{n}{Species} \PYG{o+ow}{in} \PYG{n+nb}{range}\PYG{p}{(}\PYG{l+m+mi}{2}\PYG{p}{,}\PYG{n+nb}{len}\PYG{p}{(}\PYG{n}{FGFit}\PYG{o}{.}\PYG{n}{SpeciesNames}\PYG{p}{(}\PYG{p}{)}\PYG{p}{)}\PYG{p}{,}\PYG{l+m+mi}{1}\PYG{p}{)}\PYG{p}{:}
            \PYG{k}{print} \PYG{n}{FGFit}\PYG{o}{.}\PYG{n}{SpeciesName}\PYG{p}{(}\PYG{n}{Species}\PYG{p}{)}
            \PYG{n}{ArrFG}\PYG{o}{.}\PYG{n}{setParamVector}\PYG{p}{(}\PYG{n}{LSFG}\PYG{o}{.}\PYG{n}{estimate\PYGZus{}T}\PYG{p}{(}\PYG{n}{FGFit}\PYG{p}{,}\PYG{n}{ArrFG}\PYG{p}{,}\PYG{n}{ArrFG}\PYG{o}{.}\PYG{n}{ParamVector}\PYG{p}{(}\PYG{p}{)}\PYG{p}{,}\PYG{n}{Species}\PYG{p}{)}\PYG{p}{)}
            \PYG{n}{Solution}\PYG{o}{=}\PYG{n}{ArrFG}\PYG{o}{.}\PYG{n}{ConvertKinFactors}\PYG{p}{(}\PYG{n}{ArrFG}\PYG{o}{.}\PYG{n}{ParamVector}\PYG{p}{(}\PYG{p}{)}\PYG{p}{)}
            \PYG{c}{\PYGZsh{}Solution=Arr.ParamVector()}
            \PYG{n}{ArrPFG}\PYG{o}{.}\PYG{n}{setParamVector}\PYG{p}{(}\PYG{n}{Solution}\PYG{p}{)}
            \PYG{n}{ArrPFG}\PYG{o}{.}\PYG{n}{plot}\PYG{p}{(}\PYG{n}{FGFit}\PYG{p}{,}\PYG{n}{Species}\PYG{p}{)}
            \PYG{n}{outfile}\PYG{o}{.}\PYG{n}{write}\PYG{p}{(}\PYG{n+nb}{str}\PYG{p}{(}\PYG{n}{FGFit}\PYG{o}{.}\PYG{n}{SpeciesName}\PYG{p}{(}\PYG{n}{Species}\PYG{p}{)}\PYG{p}{)}\PYG{o}{+}\PYG{l+s}{'}\PYG{l+s+se}{\PYGZbs{}t}\PYG{l+s}{'}\PYG{o}{+}\PYG{n+nb}{str}\PYG{p}{(}\PYG{n}{Solution}\PYG{p}{[}\PYG{l+m+mi}{0}\PYG{p}{]}\PYG{p}{)}\PYG{o}{+}\PYG{l+s}{'}\PYG{l+s+se}{\PYGZbs{}t}\PYG{l+s}{'}\PYG{o}{+}\PYG{n+nb}{str}\PYG{p}{(}\PYG{n}{Solution}\PYG{p}{[}\PYG{l+m+mi}{1}\PYG{p}{]}\PYG{p}{)}\PYG{o}{+}\PYG{l+s}{'}\PYG{l+s+se}{\PYGZbs{}t}\PYG{l+s}{'}\PYG{o}{+}\PYG{n+nb}{str}\PYG{p}{(}\PYG{n}{Solution}\PYG{p}{[}\PYG{l+m+mi}{2}\PYG{p}{]}\PYG{p}{)}\PYG{o}{+}\PYG{l+s}{'}\PYG{l+s+se}{\PYGZbs{}n}\PYG{l+s}{'}\PYG{p}{)}
            \PYG{c}{\PYGZsh{}for the comparison of the species sum with (1-Solid)}
            \PYG{k}{if} \PYG{n}{FGFit}\PYG{o}{.}\PYG{n}{SpeciesName}\PYG{p}{(}\PYG{n}{Species}\PYG{p}{)}\PYG{o}{==}\PYG{l+s}{'}\PYG{l+s}{Solid}\PYG{l+s}{'}\PYG{p}{:}
                \PYG{n}{SolidYieldsCalc}\PYG{o}{+}\PYG{o}{=}\PYG{n}{ArrPFG}\PYG{o}{.}\PYG{n}{calcMass}\PYG{p}{(}\PYG{n}{FGFit}\PYG{p}{,}\PYG{n}{FGFit}\PYG{o}{.}\PYG{n}{Time}\PYG{p}{(}\PYG{p}{)}\PYG{p}{,}\PYG{n}{FGFit}\PYG{o}{.}\PYG{n}{Interpolate}\PYG{p}{(}\PYG{l+s}{'}\PYG{l+s}{Temp}\PYG{l+s}{'}\PYG{p}{)}\PYG{p}{,}\PYG{n}{Species}\PYG{p}{)}
                \PYG{n}{SolidYieldsFG}\PYG{o}{+}\PYG{o}{=}\PYG{n}{FGFit}\PYG{o}{.}\PYG{n}{Yield}\PYG{p}{(}\PYG{n}{Species}\PYG{p}{)}
            \PYG{k}{elif} \PYG{n}{FGFit}\PYG{o}{.}\PYG{n}{SpeciesName}\PYG{p}{(}\PYG{n}{Species}\PYG{p}{)}\PYG{o}{!=}\PYG{l+s}{'}\PYG{l+s}{Solid}\PYG{l+s}{'} \PYG{o+ow}{and} \PYG{n}{FGFit}\PYG{o}{.}\PYG{n}{SpeciesName}\PYG{p}{(}\PYG{n}{Species}\PYG{p}{)}\PYG{o}{!=}\PYG{l+s}{'}\PYG{l+s}{Temp}\PYG{l+s}{'} \PYG{o+ow}{and} \PYG{n}{FGFit}\PYG{o}{.}\PYG{n}{SpeciesName}\PYG{p}{(}\PYG{n}{Species}\PYG{p}{)}\PYG{o}{!=}\PYG{l+s}{'}\PYG{l+s}{Time}\PYG{l+s}{'}\PYG{p}{:}
                \PYG{n}{SumSingleYieldsCalc}\PYG{o}{+}\PYG{o}{=}\PYG{n}{ArrPFG}\PYG{o}{.}\PYG{n}{calcMass}\PYG{p}{(}\PYG{n}{FGFit}\PYG{p}{,}\PYG{n}{FGFit}\PYG{o}{.}\PYG{n}{Time}\PYG{p}{(}\PYG{p}{)}\PYG{p}{,}\PYG{n}{FGFit}\PYG{o}{.}\PYG{n}{Interpolate}\PYG{p}{(}\PYG{l+s}{'}\PYG{l+s}{Temp}\PYG{l+s}{'}\PYG{p}{)}\PYG{p}{,}\PYG{n}{Species}\PYG{p}{)}
                \PYG{n}{SumSingleYieldsFG}\PYG{o}{+}\PYG{o}{=}\PYG{n}{FGFit}\PYG{o}{.}\PYG{n}{Yield}\PYG{p}{(}\PYG{n}{Species}\PYG{p}{)}
        \PYG{n}{outfile}\PYG{o}{.}\PYG{n}{close}\PYG{p}{(}\PYG{p}{)}
        \PYG{n}{FGFit}\PYG{o}{.}\PYG{n}{plt\PYGZus{}InputVectors}\PYG{p}{(}\PYG{n}{FGFit}\PYG{o}{.}\PYG{n}{Time}\PYG{p}{(}\PYG{p}{)}\PYG{p}{,}\PYG{l+m+mf}{100.}\PYG{o}{-}\PYG{n}{SolidYieldsCalc}\PYG{p}{,}\PYG{l+m+mf}{100.}\PYG{o}{-}\PYG{n}{SolidYieldsFG}\PYG{p}{,}\PYG{n}{SumSingleYieldsCalc}\PYG{p}{,}\PYG{n}{SumSingleYieldsFG}\PYG{p}{,}\PYG{l+s}{'}\PYG{l+s}{1-Solid; fitted}\PYG{l+s}{'}\PYG{p}{,}\PYG{l+s}{'}\PYG{l+s}{1-Solid; FG-DVC output}\PYG{l+s}{'}\PYG{p}{,}\PYG{l+s}{'}\PYG{l+s}{Sum Yields; fitted}\PYG{l+s}{'}\PYG{p}{,}\PYG{l+s}{'}\PYG{l+s}{Sum Yields; FG-DVC output}\PYG{l+s}{'}\PYG{p}{)}
    \PYG{c}{\PYGZsh{}\PYGZsh{}KOBAYASHI RATE}
    \PYG{k}{if} \PYG{n}{FG\PYGZus{}FittingKineticParameter\PYGZus{}Select}\PYG{o}{==}\PYG{l+s}{'}\PYG{l+s}{Kobayashi}\PYG{l+s}{'}\PYG{p}{:} \PYG{c}{\PYGZsh{}Kob means Kobayashi}
        \PYG{n}{PredictionVKob2}\PYG{o}{=}\PYG{p}{[}\PYG{l+m+mf}{2e5}\PYG{p}{,}\PYG{l+m+mf}{1.046e8}\PYG{o}{/}\PYG{l+m+mf}{8314.33}\PYG{p}{,}\PYG{l+m+mf}{1.3e7}\PYG{p}{,}\PYG{l+m+mf}{1.674e8}\PYG{o}{/}\PYG{l+m+mf}{8314.33}\PYG{p}{,}\PYG{l+m+mi}{0}\PYG{p}{,}\PYG{l+m+mi}{0}\PYG{p}{]}           \PYG{c}{\PYGZsh{}for Arrhenius notation \PYGZsh{}2 [b11,b21,b12,b22] with the second indice as the reaction}
        \PYG{n}{LSFG}\PYG{o}{=}\PYG{n}{Fit\PYGZus{}one\PYGZus{}run}\PYG{o}{.}\PYG{n}{LeastSquarsEstimator}\PYG{p}{(}\PYG{p}{)}
        \PYG{n}{LSFG}\PYG{o}{.}\PYG{n}{setOptimizer}\PYG{p}{(}\PYG{l+s}{'}\PYG{l+s}{fmin}\PYG{l+s}{'}\PYG{p}{)}\PYG{c}{\PYGZsh{}('leastsq')   \PYGZsh{} 'leastsq' often faster, but if this does not work: 'fmin' is more reliable}
        \PYG{n}{LSFG}\PYG{o}{.}\PYG{n}{setTolerance}\PYG{p}{(}\PYG{l+m+mf}{1.e-7}\PYG{p}{)}
        \PYG{n}{LSFG}\PYG{o}{.}\PYG{n}{setWeights}\PYG{p}{(}\PYG{n}{WeightY}\PYG{p}{,}\PYG{n}{WeightR}\PYG{p}{)}
        \PYG{n}{outfile} \PYG{o}{=} \PYG{n+nb}{open}\PYG{p}{(}\PYG{l+s}{'}\PYG{l+s}{FGDVC-Results\PYGZus{}KobayashiRate.txt}\PYG{l+s}{'}\PYG{p}{,} \PYG{l+s}{'}\PYG{l+s}{w}\PYG{l+s}{'}\PYG{p}{)}
        \PYG{n}{outfile}\PYG{o}{.}\PYG{n}{write}\PYG{p}{(}\PYG{l+s}{"}\PYG{l+s}{Species}\PYG{l+s+se}{\PYGZbs{}t}\PYG{l+s+se}{\PYGZbs{}t}\PYG{l+s+se}{\PYGZbs{}t}\PYG{l+s}{A1 [1/s]}\PYG{l+s+se}{\PYGZbs{}t}\PYG{l+s+se}{\PYGZbs{}t}\PYG{l+s+se}{\PYGZbs{}t}\PYG{l+s+se}{\PYGZbs{}t}\PYG{l+s}{E\PYGZus{}a1 [K]}\PYG{l+s+se}{\PYGZbs{}t}\PYG{l+s+se}{\PYGZbs{}t}\PYG{l+s}{A2 [1/s]}\PYG{l+s+se}{\PYGZbs{}t}\PYG{l+s+se}{\PYGZbs{}t}\PYG{l+s+se}{\PYGZbs{}t}\PYG{l+s+se}{\PYGZbs{}t}\PYG{l+s}{E\PYGZus{}a2 [K]}\PYG{l+s+se}{\PYGZbs{}t}\PYG{l+s+se}{\PYGZbs{}t}\PYG{l+s+se}{\PYGZbs{}t}\PYG{l+s+se}{\PYGZbs{}t}\PYG{l+s}{alpha1 }\PYG{l+s+se}{\PYGZbs{}t}\PYG{l+s+se}{\PYGZbs{}t}\PYG{l+s+se}{\PYGZbs{}t}\PYG{l+s}{alpha2 }\PYG{l+s+se}{\PYGZbs{}n}\PYG{l+s+se}{\PYGZbs{}n}\PYG{l+s}{"}\PYG{p}{)}
        \PYG{n}{KobFG}\PYG{o}{=}\PYG{n}{Fit\PYGZus{}one\PYGZus{}run}\PYG{o}{.}\PYG{n}{KobayashiA2}\PYG{p}{(}\PYG{n}{FGFit}\PYG{p}{,}\PYG{n}{PredictionVKob2}\PYG{p}{)}
        \PYG{c}{\PYGZsh{}\PYGZsh{}\PYGZsh{}\PYGZsh{}\PYGZsh{}\PYGZsh{}\PYGZsh{}}
        \PYG{c}{\PYGZsh{}uses a separate Arrhenius model to plot, to ensure that the result converted into standart notation (!) output vector is right}
        \PYG{n}{KobPFG}\PYG{o}{=}\PYG{n}{Fit\PYGZus{}one\PYGZus{}run}\PYG{o}{.}\PYG{n}{Kobayashi}\PYG{p}{(}\PYG{n}{FGFit}\PYG{p}{,}\PYG{p}{[}\PYG{l+m+mf}{2e5}\PYG{p}{,}\PYG{l+m+mf}{1.046e8}\PYG{o}{/}\PYG{l+m+mf}{8314.33}\PYG{p}{,}\PYG{l+m+mf}{1.3e7}\PYG{p}{,}\PYG{l+m+mf}{1.674e8}\PYG{o}{/}\PYG{l+m+mf}{8314.33}\PYG{p}{,}\PYG{l+m+mi}{0}\PYG{p}{,}\PYG{l+m+mi}{0}\PYG{p}{]}\PYG{p}{)}
        \PYG{c}{\PYGZsh{}\PYGZsh{}The single species:}
        \PYG{k}{for} \PYG{n}{Species} \PYG{o+ow}{in} \PYG{n+nb}{range}\PYG{p}{(}\PYG{l+m+mi}{2}\PYG{p}{,}\PYG{n+nb}{len}\PYG{p}{(}\PYG{n}{FGFit}\PYG{o}{.}\PYG{n}{SpeciesNames}\PYG{p}{(}\PYG{p}{)}\PYG{p}{)}\PYG{p}{,}\PYG{l+m+mi}{1}\PYG{p}{)}\PYG{p}{:}
            \PYG{k}{print} \PYG{n}{FGFit}\PYG{o}{.}\PYG{n}{SpeciesName}\PYG{p}{(}\PYG{n}{Species}\PYG{p}{)}
            \PYG{n}{KobFG}\PYG{o}{.}\PYG{n}{setParamVector}\PYG{p}{(}\PYG{n}{LSFG}\PYG{o}{.}\PYG{n}{estimate\PYGZus{}T}\PYG{p}{(}\PYG{n}{FGFit}\PYG{p}{,}\PYG{n}{KobFG}\PYG{p}{,}\PYG{n}{KobFG}\PYG{o}{.}\PYG{n}{ParamVector}\PYG{p}{(}\PYG{p}{)}\PYG{p}{,}\PYG{n}{Species}\PYG{p}{)}\PYG{p}{)}
            \PYG{n}{Solution}\PYG{o}{=}\PYG{n}{KobFG}\PYG{o}{.}\PYG{n}{ConvertKinFactors}\PYG{p}{(}\PYG{n}{KobFG}\PYG{o}{.}\PYG{n}{ParamVector}\PYG{p}{(}\PYG{p}{)}\PYG{p}{)}
            \PYG{c}{\PYGZsh{}Solution=Arr.ParamVector()}
            \PYG{n}{KobPFG}\PYG{o}{.}\PYG{n}{setParamVector}\PYG{p}{(}\PYG{n}{Solution}\PYG{p}{)}
            \PYG{n}{KobPFG}\PYG{o}{.}\PYG{n}{plot}\PYG{p}{(}\PYG{n}{FGFit}\PYG{p}{,}\PYG{n}{Species}\PYG{p}{)}
            \PYG{n}{outfile}\PYG{o}{.}\PYG{n}{write}\PYG{p}{(}\PYG{n+nb}{str}\PYG{p}{(}\PYG{n}{CPDFit}\PYG{o}{.}\PYG{n}{SpeciesName}\PYG{p}{(}\PYG{n}{Species}\PYG{p}{)}\PYG{p}{)}\PYG{o}{+}\PYG{l+s}{'}\PYG{l+s+se}{\PYGZbs{}t}\PYG{l+s+se}{\PYGZbs{}t}\PYG{l+s}{'}\PYG{o}{+}\PYG{n+nb}{str}\PYG{p}{(}\PYG{n}{Solution}\PYG{p}{[}\PYG{l+m+mi}{0}\PYG{p}{]}\PYG{p}{)}\PYG{o}{+}\PYG{l+s}{'}\PYG{l+s+se}{\PYGZbs{}t}\PYG{l+s+se}{\PYGZbs{}t}\PYG{l+s}{'}\PYG{o}{+}\PYG{n+nb}{str}\PYG{p}{(}\PYG{n}{Solution}\PYG{p}{[}\PYG{l+m+mi}{1}\PYG{p}{]}\PYG{p}{)}\PYG{o}{+}\PYG{l+s}{'}\PYG{l+s+se}{\PYGZbs{}t}\PYG{l+s+se}{\PYGZbs{}t}\PYG{l+s}{'}\PYG{o}{+}\PYG{n+nb}{str}\PYG{p}{(}\PYG{n}{Solution}\PYG{p}{[}\PYG{l+m+mi}{2}\PYG{p}{]}\PYG{p}{)}\PYG{o}{+}\PYG{l+s}{'}\PYG{l+s+se}{\PYGZbs{}t}\PYG{l+s+se}{\PYGZbs{}t}\PYG{l+s}{'}\PYG{o}{+}\PYG{n+nb}{str}\PYG{p}{(}\PYG{n}{Solution}\PYG{p}{[}\PYG{l+m+mi}{3}\PYG{p}{]}\PYG{p}{)}\PYG{o}{+}\PYG{l+s}{'}\PYG{l+s+se}{\PYGZbs{}t}\PYG{l+s+se}{\PYGZbs{}t}\PYG{l+s}{'}\PYG{o}{+}\PYG{n+nb}{str}\PYG{p}{(}\PYG{n}{Solution}\PYG{p}{[}\PYG{l+m+mi}{4}\PYG{p}{]}\PYG{p}{)}\PYG{o}{+}\PYG{l+s}{'}\PYG{l+s+se}{\PYGZbs{}t}\PYG{l+s+se}{\PYGZbs{}t}\PYG{l+s}{'}\PYG{o}{+}\PYG{n+nb}{str}\PYG{p}{(}\PYG{n}{Solution}\PYG{p}{[}\PYG{l+m+mi}{5}\PYG{p}{]}\PYG{p}{)}\PYG{o}{+}\PYG{l+s}{'}\PYG{l+s+se}{\PYGZbs{}n}\PYG{l+s}{'}\PYG{p}{)}
            \PYG{c}{\PYGZsh{}for the comparison of the species sum with (1-Solid)}
            \PYG{k}{if} \PYG{n}{FGFit}\PYG{o}{.}\PYG{n}{SpeciesName}\PYG{p}{(}\PYG{n}{Species}\PYG{p}{)}\PYG{o}{==}\PYG{l+s}{'}\PYG{l+s}{Solid}\PYG{l+s}{'}\PYG{p}{:}
                \PYG{n}{SolidYieldsCalc}\PYG{o}{+}\PYG{o}{=}\PYG{n}{KobPFG}\PYG{o}{.}\PYG{n}{calcMass}\PYG{p}{(}\PYG{n}{FGFit}\PYG{p}{,}\PYG{n}{FGFit}\PYG{o}{.}\PYG{n}{Time}\PYG{p}{(}\PYG{p}{)}\PYG{p}{,}\PYG{n}{FGFit}\PYG{o}{.}\PYG{n}{Interpolate}\PYG{p}{(}\PYG{l+s}{'}\PYG{l+s}{Temp}\PYG{l+s}{'}\PYG{p}{)}\PYG{p}{,}\PYG{n}{Species}\PYG{p}{)}
                \PYG{n}{SolidYieldsFG}\PYG{o}{+}\PYG{o}{=}\PYG{n}{FGFit}\PYG{o}{.}\PYG{n}{Yield}\PYG{p}{(}\PYG{n}{Species}\PYG{p}{)}
            \PYG{k}{elif} \PYG{n}{FGFit}\PYG{o}{.}\PYG{n}{SpeciesName}\PYG{p}{(}\PYG{n}{Species}\PYG{p}{)}\PYG{o}{!=}\PYG{l+s}{'}\PYG{l+s}{Solid}\PYG{l+s}{'} \PYG{o+ow}{and} \PYG{n}{FGFit}\PYG{o}{.}\PYG{n}{SpeciesName}\PYG{p}{(}\PYG{n}{Species}\PYG{p}{)}\PYG{o}{!=}\PYG{l+s}{'}\PYG{l+s}{Temp}\PYG{l+s}{'} \PYG{o+ow}{and} \PYG{n}{FGFit}\PYG{o}{.}\PYG{n}{SpeciesName}\PYG{p}{(}\PYG{n}{Species}\PYG{p}{)}\PYG{o}{!=}\PYG{l+s}{'}\PYG{l+s}{Time}\PYG{l+s}{'}\PYG{p}{:}
                \PYG{n}{SumSingleYieldsCalc}\PYG{o}{+}\PYG{o}{=}\PYG{n}{KobPFG}\PYG{o}{.}\PYG{n}{calcMass}\PYG{p}{(}\PYG{n}{FGFit}\PYG{p}{,}\PYG{n}{FGFit}\PYG{o}{.}\PYG{n}{Time}\PYG{p}{(}\PYG{p}{)}\PYG{p}{,}\PYG{n}{FGFit}\PYG{o}{.}\PYG{n}{Interpolate}\PYG{p}{(}\PYG{l+s}{'}\PYG{l+s}{Temp}\PYG{l+s}{'}\PYG{p}{)}\PYG{p}{,}\PYG{n}{Species}\PYG{p}{)}
                \PYG{n}{SumSingleYieldsFG}\PYG{o}{+}\PYG{o}{=}\PYG{n}{FGFit}\PYG{o}{.}\PYG{n}{Yield}\PYG{p}{(}\PYG{n}{Species}\PYG{p}{)}
        \PYG{n}{outfile}\PYG{o}{.}\PYG{n}{close}\PYG{p}{(}\PYG{p}{)}
        \PYG{n}{FGFit}\PYG{o}{.}\PYG{n}{plt\PYGZus{}InputVectors}\PYG{p}{(}\PYG{n}{FGFit}\PYG{o}{.}\PYG{n}{Time}\PYG{p}{(}\PYG{p}{)}\PYG{p}{,}\PYG{l+m+mf}{1.}\PYG{o}{-}\PYG{n}{SolidYieldsCalc}\PYG{p}{,}\PYG{l+m+mf}{1.}\PYG{o}{-}\PYG{n}{SolidYieldsFG}\PYG{p}{,}\PYG{n}{SumSingleYieldsCalc}\PYG{p}{,}\PYG{n}{SumSingleYieldsFG}\PYG{p}{,}\PYG{l+s}{'}\PYG{l+s}{1-Solid; fitted}\PYG{l+s}{'}\PYG{p}{,}\PYG{l+s}{'}\PYG{l+s}{1-Solid; FG-DVC output}\PYG{l+s}{'}\PYG{p}{,}\PYG{l+s}{'}\PYG{l+s}{Sum Yields; fitted}\PYG{l+s}{'}\PYG{p}{,}\PYG{l+s}{'}\PYG{l+s}{Sum Yields; FG-DVC output}\PYG{l+s}{'}\PYG{p}{)}
    \PYG{c}{\PYGZsh{}\PYGZsh{}\PYGZsh{}\PYGZsh{}\PYGZsh{}\PYGZsh{}\PYGZsh{}\PYGZsh{}\PYGZsh{}\PYGZsh{}\PYGZsh{}\PYGZsh{}\PYGZsh{}\PYGZsh{}\PYGZsh{}\PYGZsh{}\PYGZsh{}\PYGZsh{}\PYGZsh{}\PYGZsh{}\PYGZsh{}\PYGZsh{}\PYGZsh{}\PYGZsh{}\PYGZsh{}\PYGZsh{}\PYGZsh{}\PYGZsh{}\PYGZsh{}}
    \PYG{c}{\PYGZsh{}\PYGZsh{}SPECIES AND ENERGY BALANCE:}
    \PYG{n}{SpecFG}\PYG{o}{=}\PYG{n}{Compos\PYGZus{}and\PYGZus{}Energy}\PYG{o}{.}\PYG{n}{FGDVC\PYGZus{}SpeciesBalance}\PYG{p}{(}\PYG{n}{FGFile}\PYG{p}{,}\PYG{n}{UAC}\PYG{p}{,}\PYG{n}{UAH}\PYG{p}{,}\PYG{n}{UAN}\PYG{p}{,}\PYG{n}{UAO}\PYG{p}{,}\PYG{n}{PAVM\PYGZus{}asrec}\PYG{p}{,}\PYG{n}{PAFC\PYGZus{}asrec}\PYG{p}{,}\PYG{n}{HHV}\PYG{p}{,}\PYG{n}{MTar}\PYG{p}{)}
\PYG{c}{\PYGZsh{}}
\end{Verbatim}


\chapter{The Appendix}
\label{Appendix:the-appendix}\label{Appendix::doc}
As an alternative to the Least Square optimizaion, a linear regression method (see: `Holstein, A., Bassilakis, R., Wojtowicz, M.A., and Serio, M.A. “Kinetics of methane and tar
evolution during coal pyrolysis”. In: Proceedings of the Combustion Institute 30 (2005), pp. 2177–2185') was tested. But as this shows not that good results as the Least Square optimization, refering to computational time, the precision of the calculated parameter and the reliability (the results were quite dependent on the time step and the number and range of the runned heating rates), this idea and the realized classes are still in the Python-files but still not implemented in the main file, the `Pyrolysis.py'. Another problem was that the oszillating yield curve of CPD cannot be used to generate reasonable results.
Here an overview of the classes and their Methods:


\section{The linear Regression Class for FG-DVC}
\label{Appendix:the-linear-regression-class-for-fg-dvc}\index{Process (class in FGDVC\_Fit\_lin\_regr)}

\begin{fulllineitems}
\phantomsection\label{Appendix:FGDVC_Fit_lin_regr.Process}\pysiglinewithargsret{\strong{class }\code{FGDVC\_Fit\_lin\_regr.}\bfcode{Process}}{\emph{DirectoryFromFGDVCD\_EXE}}{}
Calculates the kinetic parameter using several FG-DVC runs with different heating rates.
\index{CompareResults() (FGDVC\_Fit\_lin\_regr.Process method)}

\begin{fulllineitems}
\phantomsection\label{Appendix:FGDVC_Fit_lin_regr.Process.CompareResults}\pysiglinewithargsret{\bfcode{CompareResults}}{\emph{genSetterAndLauncher}, \emph{HrWhereToCompare}}{}
Compares the FG-DVC output plot with the calculated plot using the generated Arrhenius parameter.

\end{fulllineitems}

\index{CpFile() (FGDVC\_Fit\_lin\_regr.Process method)}

\begin{fulllineitems}
\phantomsection\label{Appendix:FGDVC_Fit_lin_regr.Process.CpFile}\pysiglinewithargsret{\bfcode{CpFile}}{\emph{genSetterAndLauncher}, \emph{PathFromEXE}, \emph{HeatingR}}{}
Copies the current FG-DVC result file into the folder `runs' and renames it with their heating rate.

\end{fulllineitems}

\index{Derive() (FGDVC\_Fit\_lin\_regr.Process method)}

\begin{fulllineitems}
\phantomsection\label{Appendix:FGDVC_Fit_lin_regr.Process.Derive}\pysiglinewithargsret{\bfcode{Derive}}{\emph{u}}{}
Derive the input vector u using a CDS.

\end{fulllineitems}

\index{MaxRateTemp() (FGDVC\_Fit\_lin\_regr.Process method)}

\begin{fulllineitems}
\phantomsection\label{Appendix:FGDVC_Fit_lin_regr.Process.MaxRateTemp}\pysiglinewithargsret{\bfcode{MaxRateTemp}}{\emph{SpeciesIndice}}{}
Returns the temperature where the maximum rate occurs.

\end{fulllineitems}

\index{Rate() (FGDVC\_Fit\_lin\_regr.Process method)}

\begin{fulllineitems}
\phantomsection\label{Appendix:FGDVC_Fit_lin_regr.Process.Rate}\pysiglinewithargsret{\bfcode{Rate}}{\emph{speciesCol}}{}
Returns the yield list for the inputted species.

\end{fulllineitems}

\index{ReadRates() (FGDVC\_Fit\_lin\_regr.Process method)}

\begin{fulllineitems}
\phantomsection\label{Appendix:FGDVC_Fit_lin_regr.Process.ReadRates}\pysiglinewithargsret{\bfcode{ReadRates}}{\emph{DirectoryWhereFGDVCoutFilesAreLocated}}{}
Reads the current FG-DVC rates.

\end{fulllineitems}

\index{ReadYields() (FGDVC\_Fit\_lin\_regr.Process method)}

\begin{fulllineitems}
\phantomsection\label{Appendix:FGDVC_Fit_lin_regr.Process.ReadYields}\pysiglinewithargsret{\bfcode{ReadYields}}{\emph{DirectoryWhereFGDVCoutFilesAreLocated}}{}
Reads the current FG-DVC yield result file.

\end{fulllineitems}

\index{SpeciesName() (FGDVC\_Fit\_lin\_regr.Process method)}

\begin{fulllineitems}
\phantomsection\label{Appendix:FGDVC_Fit_lin_regr.Process.SpeciesName}\pysiglinewithargsret{\bfcode{SpeciesName}}{\emph{SpeciesIndice}}{}
Returns the Name of species with the Input column number.

\end{fulllineitems}

\index{Time() (FGDVC\_Fit\_lin\_regr.Process method)}

\begin{fulllineitems}
\phantomsection\label{Appendix:FGDVC_Fit_lin_regr.Process.Time}\pysiglinewithargsret{\bfcode{Time}}{}{}
Returns the time Array.

\end{fulllineitems}

\index{TtInterpol() (FGDVC\_Fit\_lin\_regr.Process method)}

\begin{fulllineitems}
\phantomsection\label{Appendix:FGDVC_Fit_lin_regr.Process.TtInterpol}\pysiglinewithargsret{\bfcode{TtInterpol}}{}{}
Outputs a Interpolation object for T(t).

\end{fulllineitems}

\index{Yield() (FGDVC\_Fit\_lin\_regr.Process method)}

\begin{fulllineitems}
\phantomsection\label{Appendix:FGDVC_Fit_lin_regr.Process.Yield}\pysiglinewithargsret{\bfcode{Yield}}{\emph{speciesCol}}{}
Returns the yield list for the inputted species.

\end{fulllineitems}

\index{calcAE() (FGDVC\_Fit\_lin\_regr.Process method)}

\begin{fulllineitems}
\phantomsection\label{Appendix:FGDVC_Fit_lin_regr.Process.calcAE}\pysiglinewithargsret{\bfcode{calcAE}}{\emph{genSetterAndLauncher}}{}
Calculates the Arrhenius rate parameter using the results generated with the method `GenerateRuns'.

\end{fulllineitems}

\index{calcYield() (FGDVC\_Fit\_lin\_regr.Process method)}

\begin{fulllineitems}
\phantomsection\label{Appendix:FGDVC_Fit_lin_regr.Process.calcYield}\pysiglinewithargsret{\bfcode{calcYield}}{\emph{genSetterAndLauncher}, \emph{Species}}{}
Calcules the yields for the input species.

\end{fulllineitems}

\index{generateRuns() (FGDVC\_Fit\_lin\_regr.Process method)}

\begin{fulllineitems}
\phantomsection\label{Appendix:FGDVC_Fit_lin_regr.Process.generateRuns}\pysiglinewithargsret{\bfcode{generateRuns}}{\emph{genSetterAndLauncher}, \emph{lowestHr}, \emph{highestHr}, \emph{NumberOfRuns}, \emph{pltHrCurves=True}, \emph{PltLegend=False}}{}
Runs the FG-DVC as often and with the heating rates defined before and generates an array containing the heating rate and the temperature where the maximum rate occurs.

\end{fulllineitems}

\index{getAE() (FGDVC\_Fit\_lin\_regr.Process method)}

\begin{fulllineitems}
\phantomsection\label{Appendix:FGDVC_Fit_lin_regr.Process.getAE}\pysiglinewithargsret{\bfcode{getAE}}{\emph{Species}}{}
Returns the kinetic parameter for the input species.

\end{fulllineitems}

\index{makeDt() (FGDVC\_Fit\_lin\_regr.Process method)}

\begin{fulllineitems}
\phantomsection\label{Appendix:FGDVC_Fit_lin_regr.Process.makeDt}\pysiglinewithargsret{\bfcode{makeDt}}{}{}
Generates a numpy 1D Array with the time steps.

\end{fulllineitems}

\index{setConditions() (FGDVC\_Fit\_lin\_regr.Process method)}

\begin{fulllineitems}
\phantomsection\label{Appendix:FGDVC_Fit_lin_regr.Process.setConditions}\pysiglinewithargsret{\bfcode{setConditions}}{\emph{pressureATM}, \emph{totaltime}, \emph{dt}, \emph{dT}, \emph{T\_End}, \emph{T\_Begin}}{}
Defines the OPerating conditions in FG-DVC.

\end{fulllineitems}

\index{setFiles() (FGDVC\_Fit\_lin\_regr.Process method)}

\begin{fulllineitems}
\phantomsection\label{Appendix:FGDVC_Fit_lin_regr.Process.setFiles}\pysiglinewithargsret{\bfcode{setFiles}}{\emph{CoalFile}, \emph{KinFile}, \emph{PolyFile}}{}
Sets the inputted coal files as class intern parameter.

\end{fulllineitems}


\end{fulllineitems}



\section{The linear Regression Class for CPD}
\label{Appendix:the-linear-regression-class-for-cpd}\index{ProcessCPD (class in CPD\_Fit\_lin\_regr)}

\begin{fulllineitems}
\phantomsection\label{Appendix:CPD_Fit_lin_regr.ProcessCPD}\pysiglinewithargsret{\strong{class }\code{CPD\_Fit\_lin\_regr.}\bfcode{ProcessCPD}}{\emph{DirectoryFromCPD\_EXE}}{}
Calculates the kinetic parameter using several CPD runs with different heating rates.
\index{CompareResults() (CPD\_Fit\_lin\_regr.ProcessCPD method)}

\begin{fulllineitems}
\phantomsection\label{Appendix:CPD_Fit_lin_regr.ProcessCPD.CompareResults}\pysiglinewithargsret{\bfcode{CompareResults}}{\emph{genSetterAndLauncher}, \emph{HrWhereToCompare}}{}
Plots the genearted curve and the CPD output in a Yield vs. time diagram, each for one species.

\end{fulllineitems}

\index{CpFile() (CPD\_Fit\_lin\_regr.ProcessCPD method)}

\begin{fulllineitems}
\phantomsection\label{Appendix:CPD_Fit_lin_regr.ProcessCPD.CpFile}\pysiglinewithargsret{\bfcode{CpFile}}{\emph{genSetterAndLauncher}, \emph{PathFromEXE}, \emph{HeatingR}}{}
Copies the current FG-DVC result file into the folder `runs' and renames it with their heating rate.

\end{fulllineitems}

\index{Derive() (CPD\_Fit\_lin\_regr.ProcessCPD method)}

\begin{fulllineitems}
\phantomsection\label{Appendix:CPD_Fit_lin_regr.ProcessCPD.Derive}\pysiglinewithargsret{\bfcode{Derive}}{\emph{u}}{}
Derives the input vector using a CDS.

\end{fulllineitems}

\index{MaxRateTemp() (CPD\_Fit\_lin\_regr.ProcessCPD method)}

\begin{fulllineitems}
\phantomsection\label{Appendix:CPD_Fit_lin_regr.ProcessCPD.MaxRateTemp}\pysiglinewithargsret{\bfcode{MaxRateTemp}}{\emph{SpeciesIndice}}{}
Returns the temperature where the maximum rate occurs.

\end{fulllineitems}

\index{ReadYields() (CPD\_Fit\_lin\_regr.ProcessCPD method)}

\begin{fulllineitems}
\phantomsection\label{Appendix:CPD_Fit_lin_regr.ProcessCPD.ReadYields}\pysiglinewithargsret{\bfcode{ReadYields}}{\emph{DirectoryWhereFGDVCoutFilesAreLocated}}{}
Reads the yields and rates of the current run.

\end{fulllineitems}

\index{SpeciesIndex() (CPD\_Fit\_lin\_regr.ProcessCPD method)}

\begin{fulllineitems}
\phantomsection\label{Appendix:CPD_Fit_lin_regr.ProcessCPD.SpeciesIndex}\pysiglinewithargsret{\bfcode{SpeciesIndex}}{\emph{SpeciesName}}{}
Returns the column number of the input species.

\end{fulllineitems}

\index{SpeciesName() (CPD\_Fit\_lin\_regr.ProcessCPD method)}

\begin{fulllineitems}
\phantomsection\label{Appendix:CPD_Fit_lin_regr.ProcessCPD.SpeciesName}\pysiglinewithargsret{\bfcode{SpeciesName}}{\emph{SpeciesIndice}}{}
Returns the Name of species with the Input column number.

\end{fulllineitems}

\index{Time() (CPD\_Fit\_lin\_regr.ProcessCPD method)}

\begin{fulllineitems}
\phantomsection\label{Appendix:CPD_Fit_lin_regr.ProcessCPD.Time}\pysiglinewithargsret{\bfcode{Time}}{}{}
Returns the time array.

\end{fulllineitems}

\index{TtInterpol() (CPD\_Fit\_lin\_regr.ProcessCPD method)}

\begin{fulllineitems}
\phantomsection\label{Appendix:CPD_Fit_lin_regr.ProcessCPD.TtInterpol}\pysiglinewithargsret{\bfcode{TtInterpol}}{}{}
Returns an interpolation object of T(t).

\end{fulllineitems}

\index{calcAE() (CPD\_Fit\_lin\_regr.ProcessCPD method)}

\begin{fulllineitems}
\phantomsection\label{Appendix:CPD_Fit_lin_regr.ProcessCPD.calcAE}\pysiglinewithargsret{\bfcode{calcAE}}{\emph{genSetterAndLauncher}}{}
Calculates the Arrhenius parameter using the array generated in `generateRuns'.

\end{fulllineitems}

\index{calcYield() (CPD\_Fit\_lin\_regr.ProcessCPD method)}

\begin{fulllineitems}
\phantomsection\label{Appendix:CPD_Fit_lin_regr.ProcessCPD.calcYield}\pysiglinewithargsret{\bfcode{calcYield}}{\emph{genSetterAndLauncher}, \emph{Species}}{}
Calculates the yields.

\end{fulllineitems}

\index{generateRuns() (CPD\_Fit\_lin\_regr.ProcessCPD method)}

\begin{fulllineitems}
\phantomsection\label{Appendix:CPD_Fit_lin_regr.ProcessCPD.generateRuns}\pysiglinewithargsret{\bfcode{generateRuns}}{\emph{genSetterAndLauncher}, \emph{tEnd}, \emph{Tstart}, \emph{TEnd}, \emph{lowestHr}, \emph{highestHr}, \emph{NumberOfRuns}, \emph{CPD\_exe}, \emph{Input\_File}, \emph{PltLegend=False}}{}
Runs CPD for the range of the heating rates and writes an Array containing the heating rates and the corresponding maximum rate temperatures.

\end{fulllineitems}

\index{getAE() (CPD\_Fit\_lin\_regr.ProcessCPD method)}

\begin{fulllineitems}
\phantomsection\label{Appendix:CPD_Fit_lin_regr.ProcessCPD.getAE}\pysiglinewithargsret{\bfcode{getAE}}{\emph{Species}}{}
returns the calculated kinetic parameter.

\end{fulllineitems}

\index{makeDt() (CPD\_Fit\_lin\_regr.ProcessCPD method)}

\begin{fulllineitems}
\phantomsection\label{Appendix:CPD_Fit_lin_regr.ProcessCPD.makeDt}\pysiglinewithargsret{\bfcode{makeDt}}{}{}
Generates a list of the time steps.

\end{fulllineitems}


\end{fulllineitems}



\chapter{Indices and tables}
\label{index:indices-and-tables}\begin{itemize}
\item {} 
\emph{genindex}

\item {} 
\emph{modindex}

\item {} 
\emph{search}

\end{itemize}



\renewcommand{\indexname}{Index}
\printindex
\end{document}
