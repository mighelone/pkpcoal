% Generated by Sphinx.
\def\sphinxdocclass{report}
\documentclass[letterpaper,10pt,english]{sphinxmanual}
\usepackage[utf8]{inputenc}
\DeclareUnicodeCharacter{00A0}{\nobreakspace}
\usepackage[T1]{fontenc}
\usepackage{babel}
\usepackage{times}
\usepackage[Bjarne]{fncychap}
\usepackage{longtable}
\usepackage{sphinx}
\usepackage{multirow}


\title{PKP Code Documentation}
\date{December 03, 2012}
\release{1.0.0}
\author{Martin Pollack}
\newcommand{\sphinxlogo}{}
\renewcommand{\releasename}{Release}
\makeindex

\makeatletter
\def\PYG@reset{\let\PYG@it=\relax \let\PYG@bf=\relax%
    \let\PYG@ul=\relax \let\PYG@tc=\relax%
    \let\PYG@bc=\relax \let\PYG@ff=\relax}
\def\PYG@tok#1{\csname PYG@tok@#1\endcsname}
\def\PYG@toks#1+{\ifx\relax#1\empty\else%
    \PYG@tok{#1}\expandafter\PYG@toks\fi}
\def\PYG@do#1{\PYG@bc{\PYG@tc{\PYG@ul{%
    \PYG@it{\PYG@bf{\PYG@ff{#1}}}}}}}
\def\PYG#1#2{\PYG@reset\PYG@toks#1+\relax+\PYG@do{#2}}

\def\PYG@tok@gd{\def\PYG@tc##1{\textcolor[rgb]{0.63,0.00,0.00}{##1}}}
\def\PYG@tok@gu{\let\PYG@bf=\textbf\def\PYG@tc##1{\textcolor[rgb]{0.50,0.00,0.50}{##1}}}
\def\PYG@tok@gt{\def\PYG@tc##1{\textcolor[rgb]{0.00,0.25,0.82}{##1}}}
\def\PYG@tok@gs{\let\PYG@bf=\textbf}
\def\PYG@tok@gr{\def\PYG@tc##1{\textcolor[rgb]{1.00,0.00,0.00}{##1}}}
\def\PYG@tok@cm{\let\PYG@it=\textit\def\PYG@tc##1{\textcolor[rgb]{0.25,0.50,0.56}{##1}}}
\def\PYG@tok@vg{\def\PYG@tc##1{\textcolor[rgb]{0.73,0.38,0.84}{##1}}}
\def\PYG@tok@m{\def\PYG@tc##1{\textcolor[rgb]{0.13,0.50,0.31}{##1}}}
\def\PYG@tok@mh{\def\PYG@tc##1{\textcolor[rgb]{0.13,0.50,0.31}{##1}}}
\def\PYG@tok@cs{\def\PYG@tc##1{\textcolor[rgb]{0.25,0.50,0.56}{##1}}\def\PYG@bc##1{\colorbox[rgb]{1.00,0.94,0.94}{##1}}}
\def\PYG@tok@ge{\let\PYG@it=\textit}
\def\PYG@tok@vc{\def\PYG@tc##1{\textcolor[rgb]{0.73,0.38,0.84}{##1}}}
\def\PYG@tok@il{\def\PYG@tc##1{\textcolor[rgb]{0.13,0.50,0.31}{##1}}}
\def\PYG@tok@go{\def\PYG@tc##1{\textcolor[rgb]{0.19,0.19,0.19}{##1}}}
\def\PYG@tok@cp{\def\PYG@tc##1{\textcolor[rgb]{0.00,0.44,0.13}{##1}}}
\def\PYG@tok@gi{\def\PYG@tc##1{\textcolor[rgb]{0.00,0.63,0.00}{##1}}}
\def\PYG@tok@gh{\let\PYG@bf=\textbf\def\PYG@tc##1{\textcolor[rgb]{0.00,0.00,0.50}{##1}}}
\def\PYG@tok@ni{\let\PYG@bf=\textbf\def\PYG@tc##1{\textcolor[rgb]{0.84,0.33,0.22}{##1}}}
\def\PYG@tok@nl{\let\PYG@bf=\textbf\def\PYG@tc##1{\textcolor[rgb]{0.00,0.13,0.44}{##1}}}
\def\PYG@tok@nn{\let\PYG@bf=\textbf\def\PYG@tc##1{\textcolor[rgb]{0.05,0.52,0.71}{##1}}}
\def\PYG@tok@no{\def\PYG@tc##1{\textcolor[rgb]{0.38,0.68,0.84}{##1}}}
\def\PYG@tok@na{\def\PYG@tc##1{\textcolor[rgb]{0.25,0.44,0.63}{##1}}}
\def\PYG@tok@nb{\def\PYG@tc##1{\textcolor[rgb]{0.00,0.44,0.13}{##1}}}
\def\PYG@tok@nc{\let\PYG@bf=\textbf\def\PYG@tc##1{\textcolor[rgb]{0.05,0.52,0.71}{##1}}}
\def\PYG@tok@nd{\let\PYG@bf=\textbf\def\PYG@tc##1{\textcolor[rgb]{0.33,0.33,0.33}{##1}}}
\def\PYG@tok@ne{\def\PYG@tc##1{\textcolor[rgb]{0.00,0.44,0.13}{##1}}}
\def\PYG@tok@nf{\def\PYG@tc##1{\textcolor[rgb]{0.02,0.16,0.49}{##1}}}
\def\PYG@tok@si{\let\PYG@it=\textit\def\PYG@tc##1{\textcolor[rgb]{0.44,0.63,0.82}{##1}}}
\def\PYG@tok@s2{\def\PYG@tc##1{\textcolor[rgb]{0.25,0.44,0.63}{##1}}}
\def\PYG@tok@vi{\def\PYG@tc##1{\textcolor[rgb]{0.73,0.38,0.84}{##1}}}
\def\PYG@tok@nt{\let\PYG@bf=\textbf\def\PYG@tc##1{\textcolor[rgb]{0.02,0.16,0.45}{##1}}}
\def\PYG@tok@nv{\def\PYG@tc##1{\textcolor[rgb]{0.73,0.38,0.84}{##1}}}
\def\PYG@tok@s1{\def\PYG@tc##1{\textcolor[rgb]{0.25,0.44,0.63}{##1}}}
\def\PYG@tok@gp{\let\PYG@bf=\textbf\def\PYG@tc##1{\textcolor[rgb]{0.78,0.36,0.04}{##1}}}
\def\PYG@tok@sh{\def\PYG@tc##1{\textcolor[rgb]{0.25,0.44,0.63}{##1}}}
\def\PYG@tok@ow{\let\PYG@bf=\textbf\def\PYG@tc##1{\textcolor[rgb]{0.00,0.44,0.13}{##1}}}
\def\PYG@tok@sx{\def\PYG@tc##1{\textcolor[rgb]{0.78,0.36,0.04}{##1}}}
\def\PYG@tok@bp{\def\PYG@tc##1{\textcolor[rgb]{0.00,0.44,0.13}{##1}}}
\def\PYG@tok@c1{\let\PYG@it=\textit\def\PYG@tc##1{\textcolor[rgb]{0.25,0.50,0.56}{##1}}}
\def\PYG@tok@kc{\let\PYG@bf=\textbf\def\PYG@tc##1{\textcolor[rgb]{0.00,0.44,0.13}{##1}}}
\def\PYG@tok@c{\let\PYG@it=\textit\def\PYG@tc##1{\textcolor[rgb]{0.25,0.50,0.56}{##1}}}
\def\PYG@tok@mf{\def\PYG@tc##1{\textcolor[rgb]{0.13,0.50,0.31}{##1}}}
\def\PYG@tok@err{\def\PYG@bc##1{\fcolorbox[rgb]{1.00,0.00,0.00}{1,1,1}{##1}}}
\def\PYG@tok@kd{\let\PYG@bf=\textbf\def\PYG@tc##1{\textcolor[rgb]{0.00,0.44,0.13}{##1}}}
\def\PYG@tok@ss{\def\PYG@tc##1{\textcolor[rgb]{0.32,0.47,0.09}{##1}}}
\def\PYG@tok@sr{\def\PYG@tc##1{\textcolor[rgb]{0.14,0.33,0.53}{##1}}}
\def\PYG@tok@mo{\def\PYG@tc##1{\textcolor[rgb]{0.13,0.50,0.31}{##1}}}
\def\PYG@tok@mi{\def\PYG@tc##1{\textcolor[rgb]{0.13,0.50,0.31}{##1}}}
\def\PYG@tok@kn{\let\PYG@bf=\textbf\def\PYG@tc##1{\textcolor[rgb]{0.00,0.44,0.13}{##1}}}
\def\PYG@tok@o{\def\PYG@tc##1{\textcolor[rgb]{0.40,0.40,0.40}{##1}}}
\def\PYG@tok@kr{\let\PYG@bf=\textbf\def\PYG@tc##1{\textcolor[rgb]{0.00,0.44,0.13}{##1}}}
\def\PYG@tok@s{\def\PYG@tc##1{\textcolor[rgb]{0.25,0.44,0.63}{##1}}}
\def\PYG@tok@kp{\def\PYG@tc##1{\textcolor[rgb]{0.00,0.44,0.13}{##1}}}
\def\PYG@tok@w{\def\PYG@tc##1{\textcolor[rgb]{0.73,0.73,0.73}{##1}}}
\def\PYG@tok@kt{\def\PYG@tc##1{\textcolor[rgb]{0.56,0.13,0.00}{##1}}}
\def\PYG@tok@sc{\def\PYG@tc##1{\textcolor[rgb]{0.25,0.44,0.63}{##1}}}
\def\PYG@tok@sb{\def\PYG@tc##1{\textcolor[rgb]{0.25,0.44,0.63}{##1}}}
\def\PYG@tok@k{\let\PYG@bf=\textbf\def\PYG@tc##1{\textcolor[rgb]{0.00,0.44,0.13}{##1}}}
\def\PYG@tok@se{\let\PYG@bf=\textbf\def\PYG@tc##1{\textcolor[rgb]{0.25,0.44,0.63}{##1}}}
\def\PYG@tok@sd{\let\PYG@it=\textit\def\PYG@tc##1{\textcolor[rgb]{0.25,0.44,0.63}{##1}}}

\def\PYGZbs{\char`\\}
\def\PYGZus{\char`\_}
\def\PYGZob{\char`\{}
\def\PYGZcb{\char`\}}
\def\PYGZca{\char`\^}
\def\PYGZsh{\char`\#}
\def\PYGZpc{\char`\%}
\def\PYGZdl{\char`\$}
\def\PYGZti{\char`\~}
% for compatibility with earlier versions
\def\PYGZat{@}
\def\PYGZlb{[}
\def\PYGZrb{]}
\makeatother

\begin{document}

\maketitle
\tableofcontents
\phantomsection\label{index::doc}


This is the code documentation of the PKP program. The general structur of the classes is the following:
\begin{itemize}
\item {} 
There are pyrolysis program (like CPD and FG-DVC) specific ones to write the configuration files, launch their program and read the specific output files and support the other classes with the pyrolysis program specific data. Every of the programs has also it's own class calculating the species and energy balance.

\item {} 
The pyrolysis program independent Fitting classes.

\item {} 
The IO-classes reading the user input and the classes generaing the output, e.g. for the FG-DVC coal file.

\item {} 
The different user input classes reading the input files or the GUI input.

\end{itemize}

The manual is also structured the same way. Firstly the specific classes for CPD and FG-DVC are introduced, followed by the fitting classes.

Required additional python libraries for using PKP are:
\begin{itemize}
\item {} 
scipy

\item {} 
numpy

\item {} 
pylab

\item {} 
pyevolve

\item {} 
PyQt4

\end{itemize}


\chapter{The CPD specific Classes}
\label{CPDClasses::doc}\label{CPDClasses:the-pkp-code-documentation}\label{CPDClasses:the-cpd-specific-classes}
The CPD classes are contained in the files:
\begin{itemize}
\item {} 
CPD\_SetAndLaunch.py  (Writes the CPD input file and launches the program)

\item {} 
Compos\_and\_Energy.py  (The species and energy balance)

\item {} 
CPD\_Result.py (Reads the specific output format of CPD)

\end{itemize}
\phantomsection\label{CPDClasses:ss-readgen}

\section{The Class writing the CPD instruction File and launching CPD}
\label{CPDClasses:ss-readgen}\label{CPDClasses:the-class-writing-the-cpd-instruction-file-and-launching-cpd}\label{CPDClasses:my-reference-label}\index{SetterAndLauncher (class in CPD\_SetAndLaunch)}

\begin{fulllineitems}
\phantomsection\label{CPDClasses:CPD_SetAndLaunch.SetterAndLauncher}\pysigline{\strong{class }\code{CPD\_SetAndLaunch.}\bfcode{SetterAndLauncher}}
This class is able to write the CPD input file and launch the CPD program. Before writing the CPD input file (method `writeInstructFile') set all parameter using the corresponding methods (SetCoalParameter, SetOperateCond, SetNumericalParam, CalcCoalParam). After writing the instruct file, the .Run method can be used.
\index{CalcCoalParam() (CPD\_SetAndLaunch.SetterAndLauncher method)}

\begin{fulllineitems}
\phantomsection\label{CPDClasses:CPD_SetAndLaunch.SetterAndLauncher.CalcCoalParam}\pysiglinewithargsret{\bfcode{CalcCoalParam}}{}{}
Calculates the CPD coal parameter mdel, mw, p0, sig and sets the as an attribute of the class.

\end{fulllineitems}

\index{Run() (CPD\_SetAndLaunch.SetterAndLauncher method)}

\begin{fulllineitems}
\phantomsection\label{CPDClasses:CPD_SetAndLaunch.SetterAndLauncher.Run}\pysiglinewithargsret{\bfcode{Run}}{\emph{CPD\_exe}, \emph{Input\_File}, \emph{LogFile}}{}
Launches CPD\_exe and inputs Input\_File. If the CPD executable is in another directory than the Python script enter the whole path for CPD\_exe.

\end{fulllineitems}

\index{SetCoalParameter() (CPD\_SetAndLaunch.SetterAndLauncher method)}

\begin{fulllineitems}
\phantomsection\label{CPDClasses:CPD_SetAndLaunch.SetterAndLauncher.SetCoalParameter}\pysiglinewithargsret{\bfcode{SetCoalParameter}}{\emph{fcar}, \emph{fhyd}, \emph{fnit}, \emph{foxy}, \emph{VMdaf}}{}
Set the mass fraction of carbon (UA), hydrogen  (UA), nitrogen (UA), oxygen (UA) and the daf fraction of volatile matter (PA).

\end{fulllineitems}

\index{SetNumericalParam() (CPD\_SetAndLaunch.SetterAndLauncher method)}

\begin{fulllineitems}
\phantomsection\label{CPDClasses:CPD_SetAndLaunch.SetterAndLauncher.SetNumericalParam}\pysiglinewithargsret{\bfcode{SetNumericalParam}}{\emph{dt}, \emph{timax}}{}
Sets the numerical parameter and the maximum simulation time. dt is a vector with the following information: {[}dt\_initial,print-increment,dt\_max{]}

\end{fulllineitems}

\index{SetOperateCond() (CPD\_SetAndLaunch.SetterAndLauncher method)}

\begin{fulllineitems}
\phantomsection\label{CPDClasses:CPD_SetAndLaunch.SetterAndLauncher.SetOperateCond}\pysiglinewithargsret{\bfcode{SetOperateCond}}{\emph{pressure}, \emph{TimeTemp}}{}
Stes the operating condtions pressure and the time-temperature array. TimeTemp is an 2D-Array. One line of this array is {[}time\_i,temp\_i{]}.

\end{fulllineitems}

\index{writeInstructFile() (CPD\_SetAndLaunch.SetterAndLauncher method)}

\begin{fulllineitems}
\phantomsection\label{CPDClasses:CPD_SetAndLaunch.SetterAndLauncher.writeInstructFile}\pysiglinewithargsret{\bfcode{writeInstructFile}}{\emph{Dirpath}}{}
Writes the File `CPD\_input.dat' into the directory Dirpath.

\end{fulllineitems}


\end{fulllineitems}



\section{The Class making the Species and the Energy Balance}
\label{CPDClasses:the-class-making-the-species-and-the-energy-balance}\index{SpeciesBalance (class in Compos\_and\_Energy)}

\begin{fulllineitems}
\phantomsection\label{CPDClasses:Compos_and_Energy.SpeciesBalance}\pysigline{\strong{class }\code{Compos\_and\_Energy.}\bfcode{SpeciesBalance}}
This is the parent Class for the CPD and FG-DVC specific Species Balances, containing general methods like the Dulong formular.
\index{Dulong() (Compos\_and\_Energy.SpeciesBalance method)}

\begin{fulllineitems}
\phantomsection\label{CPDClasses:Compos_and_Energy.SpeciesBalance.Dulong}\pysiglinewithargsret{\bfcode{Dulong}}{}{}
Uses the Dulong formular to calculate the Higher heating value. The output is in J/kg.

\end{fulllineitems}

\index{SpeciesIndex() (Compos\_and\_Energy.SpeciesBalance method)}

\begin{fulllineitems}
\phantomsection\label{CPDClasses:Compos_and_Energy.SpeciesBalance.SpeciesIndex}\pysiglinewithargsret{\bfcode{SpeciesIndex}}{\emph{species}}{}
Returns the column number of the input species.

\end{fulllineitems}

\index{correctUA() (Compos\_and\_Energy.SpeciesBalance method)}

\begin{fulllineitems}
\phantomsection\label{CPDClasses:Compos_and_Energy.SpeciesBalance.correctUA}\pysiglinewithargsret{\bfcode{correctUA}}{}{}
Scale Ultimate Analysis to have sum=1

\end{fulllineitems}


\end{fulllineitems}

\index{CPD\_SpeciesBalance (class in Compos\_and\_Energy)}

\begin{fulllineitems}
\phantomsection\label{CPDClasses:Compos_and_Energy.CPD_SpeciesBalance}\pysiglinewithargsret{\strong{class }\code{Compos\_and\_Energy.}\bfcode{CPD\_SpeciesBalance}}{\emph{CPD\_ResultObject}, \emph{UAC}, \emph{UAH}, \emph{UAN}, \emph{UAO}, \emph{UAS}, \emph{PAVM}, \emph{PAFC}, \emph{PAmoist}, \emph{PAash}, \emph{HHV}, \emph{MTar}, \emph{RunNr}}{}
This class calculates the Species and the Energy balance for CPD. See the manual for the formulas and more details.
\index{\_CPD\_SpeciesBalance\_\_CheckOthers() (Compos\_and\_Energy.CPD\_SpeciesBalance method)}

\begin{fulllineitems}
\phantomsection\label{CPDClasses:Compos_and_Energy.CPD_SpeciesBalance._CPD_SpeciesBalance__CheckOthers}\pysiglinewithargsret{\bfcode{\_CPD\_SpeciesBalance\_\_CheckOthers}}{}{}
If the yield of nitrogen (is equal the UA of nitrogen) is lower than the species `Other' the difference is set equal Methane.

\end{fulllineitems}

\index{\_CPD\_SpeciesBalance\_\_CheckOxygen() (Compos\_and\_Energy.CPD\_SpeciesBalance method)}

\begin{fulllineitems}
\phantomsection\label{CPDClasses:Compos_and_Energy.CPD_SpeciesBalance._CPD_SpeciesBalance__CheckOxygen}\pysiglinewithargsret{\bfcode{\_CPD\_SpeciesBalance\_\_CheckOxygen}}{}{}
Checks weather the amount of Oxygen in the light gases is lower than in the Ultimate Analysis. If not, the amount of these species decreases. If yes, the tar contains oxygen.

\end{fulllineitems}

\index{\_CPD\_SpeciesBalance\_\_QPyro() (Compos\_and\_Energy.CPD\_SpeciesBalance method)}

\begin{fulllineitems}
\phantomsection\label{CPDClasses:Compos_and_Energy.CPD_SpeciesBalance._CPD_SpeciesBalance__QPyro}\pysiglinewithargsret{\bfcode{\_CPD\_SpeciesBalance\_\_QPyro}}{}{}
Calculates the heat of the pyrolysis process, assuming heat of tar formation is equal zero.

\end{fulllineitems}

\index{\_CPD\_SpeciesBalance\_\_Q\_React() (Compos\_and\_Energy.CPD\_SpeciesBalance method)}

\begin{fulllineitems}
\phantomsection\label{CPDClasses:Compos_and_Energy.CPD_SpeciesBalance._CPD_SpeciesBalance__Q_React}\pysiglinewithargsret{\bfcode{\_CPD\_SpeciesBalance\_\_Q\_React}}{}{}
Calculates the Heat of Reaction of the coal cumbustion.

\end{fulllineitems}

\index{\_CPD\_SpeciesBalance\_\_TarComp() (Compos\_and\_Energy.CPD\_SpeciesBalance method)}

\begin{fulllineitems}
\phantomsection\label{CPDClasses:Compos_and_Energy.CPD_SpeciesBalance._CPD_SpeciesBalance__TarComp}\pysiglinewithargsret{\bfcode{\_CPD\_SpeciesBalance\_\_TarComp}}{}{}
Calculates and returns the tar composition.

\end{fulllineitems}

\index{\_CPD\_SpeciesBalance\_\_closeResultFile() (Compos\_and\_Energy.CPD\_SpeciesBalance method)}

\begin{fulllineitems}
\phantomsection\label{CPDClasses:Compos_and_Energy.CPD_SpeciesBalance._CPD_SpeciesBalance__closeResultFile}\pysiglinewithargsret{\bfcode{\_CPD\_SpeciesBalance\_\_closeResultFile}}{}{}
Closes the CPD Composition file.

\end{fulllineitems}

\index{\_CPD\_SpeciesBalance\_\_correctYields() (Compos\_and\_Energy.CPD\_SpeciesBalance method)}

\begin{fulllineitems}
\phantomsection\label{CPDClasses:Compos_and_Energy.CPD_SpeciesBalance._CPD_SpeciesBalance__correctYields}\pysiglinewithargsret{\bfcode{\_CPD\_SpeciesBalance\_\_correctYields}}{}{}
Correct the amount of the yields `Other'.

\end{fulllineitems}

\index{\_CPD\_SpeciesBalance\_\_hfRaw() (Compos\_and\_Energy.CPD\_SpeciesBalance method)}

\begin{fulllineitems}
\phantomsection\label{CPDClasses:Compos_and_Energy.CPD_SpeciesBalance._CPD_SpeciesBalance__hfRaw}\pysiglinewithargsret{\bfcode{\_CPD\_SpeciesBalance\_\_hfRaw}}{}{}
Calculates the heat of formation of the coal molecule and writes it into the output file.

\end{fulllineitems}

\index{\_CPD\_SpeciesBalance\_\_hfTar() (Compos\_and\_Energy.CPD\_SpeciesBalance method)}

\begin{fulllineitems}
\phantomsection\label{CPDClasses:Compos_and_Energy.CPD_SpeciesBalance._CPD_SpeciesBalance__hfTar}\pysiglinewithargsret{\bfcode{\_CPD\_SpeciesBalance\_\_hfTar}}{}{}
Calculates the heat of formation of tar and writes it into the CPD Composition file.

\end{fulllineitems}

\index{\_CPD\_SpeciesBalance\_\_nysEq15() (Compos\_and\_Energy.CPD\_SpeciesBalance method)}

\begin{fulllineitems}
\phantomsection\label{CPDClasses:Compos_and_Energy.CPD_SpeciesBalance._CPD_SpeciesBalance__nysEq15}\pysiglinewithargsret{\bfcode{\_CPD\_SpeciesBalance\_\_nysEq15}}{}{}
Calculates the stoichiometric coefficients of the devolatilization reaction.

\end{fulllineitems}


\end{fulllineitems}



\section{The Reading Class containg the CPD specific output information}
\label{CPDClasses:the-reading-class-containg-the-cpd-specific-output-information}\index{CPD\_Result (class in CPD\_Result)}

\begin{fulllineitems}
\phantomsection\label{CPDClasses:CPD_Result.CPD_Result}\pysiglinewithargsret{\strong{class }\code{CPD\_Result.}\bfcode{CPD\_Result}}{\emph{FilePath}}{}
Reads the CPD input and saves the values in one array. The results include the yields and the rates. The rates were calculated using a CDS. This class also contains the dictionaries for the columns in the array - the name of the species. These dictionaries are CPD-Version dependent and the only thing which has to be changed for the case of a new release of CPD with a new order of species in the result files.
\index{DictCols2Yields() (CPD\_Result.CPD\_Result method)}

\begin{fulllineitems}
\phantomsection\label{CPDClasses:CPD_Result.CPD_Result.DictCols2Yields}\pysiglinewithargsret{\bfcode{DictCols2Yields}}{}{}
Returns the whole Dictionary Columns of the matrix to Yield names

\end{fulllineitems}

\index{DictYields2Cols() (CPD\_Result.CPD\_Result method)}

\begin{fulllineitems}
\phantomsection\label{CPDClasses:CPD_Result.CPD_Result.DictYields2Cols}\pysiglinewithargsret{\bfcode{DictYields2Cols}}{}{}
Returns the whole Dictionary Yield names to Columns of the matrix

\end{fulllineitems}

\index{FinalYields() (CPD\_Result.CPD\_Result method)}

\begin{fulllineitems}
\phantomsection\label{CPDClasses:CPD_Result.CPD_Result.FinalYields}\pysiglinewithargsret{\bfcode{FinalYields}}{}{}
Returns the last line of the Array, containing the yields at the time=time\_End

\end{fulllineitems}

\index{Name() (CPD\_Result.CPD\_Result method)}

\begin{fulllineitems}
\phantomsection\label{CPDClasses:CPD_Result.CPD_Result.Name}\pysiglinewithargsret{\bfcode{Name}}{}{}
returns `CPD' as the name of the Program

\end{fulllineitems}

\index{Rates\_all() (CPD\_Result.CPD\_Result method)}

\begin{fulllineitems}
\phantomsection\label{CPDClasses:CPD_Result.CPD_Result.Rates_all}\pysiglinewithargsret{\bfcode{Rates\_all}}{}{}
Returns the whole result matrix of the Rates.

\end{fulllineitems}

\index{Yields\_all() (CPD\_Result.CPD\_Result method)}

\begin{fulllineitems}
\phantomsection\label{CPDClasses:CPD_Result.CPD_Result.Yields_all}\pysiglinewithargsret{\bfcode{Yields\_all}}{}{}
Returns the whole result matrix of the yields.

\end{fulllineitems}


\end{fulllineitems}



\chapter{The FG-DVC specific Classes}
\label{FGDVCClasses::doc}\label{FGDVCClasses:the-fg-dvc-specific-classes}
The FG-FVC classes are contained in the files:
\begin{itemize}
\item {} 
FGDVC\_SetAndLaunch.py  (Writes the FG-DVC `instruct.ini' and launches FG-DVC)

\item {} 
Compos\_and\_Energy.py  (The species and energy balance)

\item {} 
FGDVC\_Result.py  (`FGDVC\_Result' reads the FG-DVC output and contains the output specific information.)

\end{itemize}
\phantomsection\label{FGDVCClasses:ss-readgen}

\section{The Class writing the FG-DVC instruction file and launches FG-DVC}
\label{FGDVCClasses:ss-readgen}\label{FGDVCClasses:the-class-writing-the-fg-dvc-instruction-file-and-launches-fg-dvc}\label{FGDVCClasses:my-reference-label}\index{SetterAndLauncher (class in FGDVC\_SetAndLaunch)}

\begin{fulllineitems}
\phantomsection\label{FGDVCClasses:FGDVC_SetAndLaunch.SetterAndLauncher}\pysigline{\strong{class }\code{FGDVC\_SetAndLaunch.}\bfcode{SetterAndLauncher}}
This class is able to write the `instruct.ini' and launch the `fgdvcd.exe'. Before writing the instruct.ini (method `writeInstructFile') set all parameter using the corresponding methods (at least necessary: set1CoalLocation, set2KinLocation, set3PolyLocation, set5Pressure, set7Ramp). After writing the instruct file, the .Run method can be used.
\index{Run() (FGDVC\_SetAndLaunch.SetterAndLauncher method)}

\begin{fulllineitems}
\phantomsection\label{FGDVCClasses:FGDVC_SetAndLaunch.SetterAndLauncher.Run}\pysiglinewithargsret{\bfcode{Run}}{\emph{PathFromEXE}}{}
Lauchnes fgdvcd.exe. The `PathFromEXE' should not include the .exe and end with a backslash.

\end{fulllineitems}

\index{set1CoalLocation() (FGDVC\_SetAndLaunch.SetterAndLauncher method)}

\begin{fulllineitems}
\phantomsection\label{FGDVCClasses:FGDVC_SetAndLaunch.SetterAndLauncher.set1CoalLocation}\pysiglinewithargsret{\bfcode{set1CoalLocation}}{\emph{PathCoalFile}}{}
Sets the coal composition file directory.

\end{fulllineitems}

\index{set2KinLocation() (FGDVC\_SetAndLaunch.SetterAndLauncher method)}

\begin{fulllineitems}
\phantomsection\label{FGDVCClasses:FGDVC_SetAndLaunch.SetterAndLauncher.set2KinLocation}\pysiglinewithargsret{\bfcode{set2KinLocation}}{\emph{PathKinFile}}{}
Sets the coal kinetic file directory.

\end{fulllineitems}

\index{set3PolyLocation() (FGDVC\_SetAndLaunch.SetterAndLauncher method)}

\begin{fulllineitems}
\phantomsection\label{FGDVCClasses:FGDVC_SetAndLaunch.SetterAndLauncher.set3PolyLocation}\pysiglinewithargsret{\bfcode{set3PolyLocation}}{\emph{PathPolyFile}}{}
Sets the coal polymer file directory.

\end{fulllineitems}

\index{set4RunID() (FGDVC\_SetAndLaunch.SetterAndLauncher method)}

\begin{fulllineitems}
\phantomsection\label{FGDVCClasses:FGDVC_SetAndLaunch.SetterAndLauncher.set4RunID}\pysiglinewithargsret{\bfcode{set4RunID}}{\emph{pyrolysisORgeology}}{}
Sets weather the pyrolysis process or the geolegy process shall be modeled. For more information see FG-DVC manual.

\end{fulllineitems}

\index{set5Pressure() (FGDVC\_SetAndLaunch.SetterAndLauncher method)}

\begin{fulllineitems}
\phantomsection\label{FGDVCClasses:FGDVC_SetAndLaunch.SetterAndLauncher.set5Pressure}\pysiglinewithargsret{\bfcode{set5Pressure}}{\emph{PressureIn\_atm}}{}
Sets the operating pressure (float).

\end{fulllineitems}

\index{set6Theorie() (FGDVC\_SetAndLaunch.SetterAndLauncher method)}

\begin{fulllineitems}
\phantomsection\label{FGDVCClasses:FGDVC_SetAndLaunch.SetterAndLauncher.set6Theorie}\pysiglinewithargsret{\bfcode{set6Theorie}}{\emph{Theorie}, \emph{ResidenceTime}}{}
Sets the theory: 13 for no or partial tar cracking, 15 for full tar cracking. The residence input should be 0.0 for no tar cracking or a time greater zero for partial tar pressure. Full tar pressure also requires 0.0 as input, as it is the characteristic input of FG-DVC (although writing here 0.0, the full tar cracking is activated).

\end{fulllineitems}

\index{set7File() (FGDVC\_SetAndLaunch.SetterAndLauncher method)}

\begin{fulllineitems}
\phantomsection\label{FGDVCClasses:FGDVC_SetAndLaunch.SetterAndLauncher.set7File}\pysiglinewithargsret{\bfcode{set7File}}{\emph{THistoryFileLocation}}{}
For the case a temperature history shall be imported, this method should be used. `THistoryFileLocation' is the directory of the .txt file containing two columns: first column the time in seconds, the second the temperature indegree Celsius-

\end{fulllineitems}

\index{set7Ramp() (FGDVC\_SetAndLaunch.SetterAndLauncher method)}

\begin{fulllineitems}
\phantomsection\label{FGDVCClasses:FGDVC_SetAndLaunch.SetterAndLauncher.set7Ramp}\pysiglinewithargsret{\bfcode{set7Ramp}}{\emph{timeTotal}, \emph{dt}, \emph{dT}, \emph{finalPyrolysisTemp}, \emph{initialT}, \emph{HeatingRate}}{}
Sets the following time relevant and temperature history relevant parameter: the total simulation time `timeTotal', the constant numerical time step `dt', the maximum temperture step `dT', the final pyrolysis temperature `finalPyrolysisTemp', the temperature at t=0 `initialT', and the constant heating rate `HeatingRate'. All these parameter are required for the case a linear heating rate should be modeled.

\end{fulllineitems}

\index{set9AshMoisture() (FGDVC\_SetAndLaunch.SetterAndLauncher method)}

\begin{fulllineitems}
\phantomsection\label{FGDVCClasses:FGDVC_SetAndLaunch.SetterAndLauncher.set9AshMoisture}\pysiglinewithargsret{\bfcode{set9AshMoisture}}{\emph{AshContent}, \emph{MoistureContent}}{}
Sets the amount of ash and moisture in the coal. By initializing the SetterAndLauncher object, both of these values are setted equal zero.

\end{fulllineitems}

\index{setTRamp\_or\_TFile() (FGDVC\_SetAndLaunch.SetterAndLauncher method)}

\begin{fulllineitems}
\phantomsection\label{FGDVCClasses:FGDVC_SetAndLaunch.SetterAndLauncher.setTRamp_or_TFile}\pysiglinewithargsret{\bfcode{setTRamp\_or\_TFile}}{\emph{selectRamp\_or\_File}}{}
Select weather the time should be defined using a linear ramp (`selectRamp\_or\_File'='Ramp') or with a input file (`selectRamp\_or\_File'='File').

\end{fulllineitems}

\index{writeInstructFile() (FGDVC\_SetAndLaunch.SetterAndLauncher method)}

\begin{fulllineitems}
\phantomsection\label{FGDVCClasses:FGDVC_SetAndLaunch.SetterAndLauncher.writeInstructFile}\pysiglinewithargsret{\bfcode{writeInstructFile}}{\emph{Filepath}}{}
Writes the File `instruct.ini' into the directory `Filepath', which should end with FGDVC\_8-2-3/FGDVC .

\end{fulllineitems}


\end{fulllineitems}



\subsection{The Class generating the Coal File}
\label{FGDVCClasses:the-class-generating-the-coal-file}\index{WriteFGDVCCoalFile (class in InformationFiles)}

\begin{fulllineitems}
\phantomsection\label{FGDVCClasses:InformationFiles.WriteFGDVCCoalFile}\pysiglinewithargsret{\strong{class }\code{InformationFiles.}\bfcode{WriteFGDVCCoalFile}}{\emph{CoalGenFile}}{}
writes the file, which will be inputted into the FG-DVC coal generator
\index{setCoalComp() (InformationFiles.WriteFGDVCCoalFile method)}

\begin{fulllineitems}
\phantomsection\label{FGDVCClasses:InformationFiles.WriteFGDVCCoalFile.setCoalComp}\pysiglinewithargsret{\bfcode{setCoalComp}}{\emph{Carbon}, \emph{Hydrogen}, \emph{Oxygen}, \emph{Nitrogen}, \emph{Sulfur}, \emph{SulfurPyrite}}{}
Enter the coal composition with values in percent which have to sum up to 100

\end{fulllineitems}

\index{write() (InformationFiles.WriteFGDVCCoalFile method)}

\begin{fulllineitems}
\phantomsection\label{FGDVCClasses:InformationFiles.WriteFGDVCCoalFile.write}\pysiglinewithargsret{\bfcode{write}}{\emph{CoalsDirectory}, \emph{CoalResultFileName}}{}
writes the FG-DVC coal generator input file

\end{fulllineitems}


\end{fulllineitems}



\section{The Class making the Species and the Energy Balance}
\label{FGDVCClasses:the-class-making-the-species-and-the-energy-balance}\index{SpeciesBalance (class in Compos\_and\_Energy)}

\begin{fulllineitems}
\phantomsection\label{FGDVCClasses:Compos_and_Energy.SpeciesBalance}\pysigline{\strong{class }\code{Compos\_and\_Energy.}\bfcode{SpeciesBalance}}
This is the parent Class for the CPD and FG-DVC specific Species Balances, containing general methods like the Dulong formular.
\index{Dulong() (Compos\_and\_Energy.SpeciesBalance method)}

\begin{fulllineitems}
\phantomsection\label{FGDVCClasses:Compos_and_Energy.SpeciesBalance.Dulong}\pysiglinewithargsret{\bfcode{Dulong}}{}{}
Uses the Dulong formular to calculate the Higher heating value. The output is in J/kg.

\end{fulllineitems}

\index{SpeciesIndex() (Compos\_and\_Energy.SpeciesBalance method)}

\begin{fulllineitems}
\phantomsection\label{FGDVCClasses:Compos_and_Energy.SpeciesBalance.SpeciesIndex}\pysiglinewithargsret{\bfcode{SpeciesIndex}}{\emph{species}}{}
Returns the column number of the input species.

\end{fulllineitems}

\index{correctUA() (Compos\_and\_Energy.SpeciesBalance method)}

\begin{fulllineitems}
\phantomsection\label{FGDVCClasses:Compos_and_Energy.SpeciesBalance.correctUA}\pysiglinewithargsret{\bfcode{correctUA}}{}{}
Scale Ultimate Analysis to have sum=1

\end{fulllineitems}


\end{fulllineitems}

\index{FGDVC\_SpeciesBalance (class in Compos\_and\_Energy)}

\begin{fulllineitems}
\phantomsection\label{FGDVCClasses:Compos_and_Energy.FGDVC_SpeciesBalance}\pysiglinewithargsret{\strong{class }\code{Compos\_and\_Energy.}\bfcode{FGDVC\_SpeciesBalance}}{\emph{FGDVC\_ResultObject}, \emph{UAC}, \emph{UAH}, \emph{UAN}, \emph{UAO}, \emph{UAS}, \emph{PAVM}, \emph{PAFC}, \emph{PAmoist}, \emph{PAash}, \emph{HHV}, \emph{MTar}, \emph{RunNr}}{}
This class calculates the Species and the Energy balance for FG-DVC. See the manual for the formulars and more details.
\index{\_FGDVC\_SpeciesBalance\_\_EnergyBalance() (Compos\_and\_Energy.FGDVC\_SpeciesBalance method)}

\begin{fulllineitems}
\phantomsection\label{FGDVCClasses:Compos_and_Energy.FGDVC_SpeciesBalance._FGDVC_SpeciesBalance__EnergyBalance}\pysiglinewithargsret{\bfcode{\_FGDVC\_SpeciesBalance\_\_EnergyBalance}}{}{}
Calculates the heat of formation of tar.

\end{fulllineitems}

\index{\_FGDVC\_SpeciesBalance\_\_TarComp() (Compos\_and\_Energy.FGDVC\_SpeciesBalance method)}

\begin{fulllineitems}
\phantomsection\label{FGDVCClasses:Compos_and_Energy.FGDVC_SpeciesBalance._FGDVC_SpeciesBalance__TarComp}\pysiglinewithargsret{\bfcode{\_FGDVC\_SpeciesBalance\_\_TarComp}}{}{}
Calculates the Tar composition using analyisis of the Ultimate Analysis.

\end{fulllineitems}

\index{\_FGDVC\_SpeciesBalance\_\_closeFile() (Compos\_and\_Energy.FGDVC\_SpeciesBalance method)}

\begin{fulllineitems}
\phantomsection\label{FGDVCClasses:Compos_and_Energy.FGDVC_SpeciesBalance._FGDVC_SpeciesBalance__closeFile}\pysiglinewithargsret{\bfcode{\_FGDVC\_SpeciesBalance\_\_closeFile}}{}{}
Closes the FG-DVC Composition file.

\end{fulllineitems}

\index{\_FGDVC\_SpeciesBalance\_\_correctYields() (Compos\_and\_Energy.FGDVC\_SpeciesBalance method)}

\begin{fulllineitems}
\phantomsection\label{FGDVCClasses:Compos_and_Energy.FGDVC_SpeciesBalance._FGDVC_SpeciesBalance__correctYields}\pysiglinewithargsret{\bfcode{\_FGDVC\_SpeciesBalance\_\_correctYields}}{}{}
Further, only the yields of char, tar, CO, CO2, H2O, CH4, N2, H2, O2 are considered. Modifies the yields, merge species like Olefins parafins, HCN to tar.

\end{fulllineitems}

\index{\_FGDVC\_SpeciesBalance\_\_writeEnergyResults() (Compos\_and\_Energy.FGDVC\_SpeciesBalance method)}

\begin{fulllineitems}
\phantomsection\label{FGDVCClasses:Compos_and_Energy.FGDVC_SpeciesBalance._FGDVC_SpeciesBalance__writeEnergyResults}\pysiglinewithargsret{\bfcode{\_FGDVC\_SpeciesBalance\_\_writeEnergyResults}}{}{}
Writes the Energy results into the result file.

\end{fulllineitems}

\index{\_FGDVC\_SpeciesBalance\_\_writeSpeciesResults() (Compos\_and\_Energy.FGDVC\_SpeciesBalance method)}

\begin{fulllineitems}
\phantomsection\label{FGDVCClasses:Compos_and_Energy.FGDVC_SpeciesBalance._FGDVC_SpeciesBalance__writeSpeciesResults}\pysiglinewithargsret{\bfcode{\_FGDVC\_SpeciesBalance\_\_writeSpeciesResults}}{}{}
Writes the Species Balance results into the result file.

\end{fulllineitems}


\end{fulllineitems}



\section{The Reading Class containg the CPD specific output information}
\label{FGDVCClasses:the-reading-class-containg-the-cpd-specific-output-information}\index{FGDVC\_Result (class in FGDVC\_Result)}

\begin{fulllineitems}
\phantomsection\label{FGDVCClasses:FGDVC_Result.FGDVC_Result}\pysiglinewithargsret{\strong{class }\code{FGDVC\_Result.}\bfcode{FGDVC\_Result}}{\emph{FilePath}}{}
Reads the FG-DVC input and saves the values in one array. The results include the yields (from `gasyields.txt') and the rates. The rates for all species except the solids (here a CDS is used) are imported from `gasrates.txt'. The H\_2 yields were calculated by subtract all other species except parafins and olefins from the total yields (see FG-DVC manual). This H\_2-yield curve was smoothed and derived using a CDS to generate the H\_2 rates. The parafins and olefins are added into the tar. This class also contains the dictionaries for the columns in the array - the name of the species. These dictionaries are FG-DVC-Version dependent and the only thing which has to be changed for the case of a new release of FG-DVC with a new order of species in the result files (this was made for Versions 8.2.2. and 8.2.3.).
\index{DictCols2Yields() (FGDVC\_Result.FGDVC\_Result method)}

\begin{fulllineitems}
\phantomsection\label{FGDVCClasses:FGDVC_Result.FGDVC_Result.DictCols2Yields}\pysiglinewithargsret{\bfcode{DictCols2Yields}}{}{}
Returns the whole Dictionary Columns of the matrix to Yield names

\end{fulllineitems}

\index{DictYields2Cols() (FGDVC\_Result.FGDVC\_Result method)}

\begin{fulllineitems}
\phantomsection\label{FGDVCClasses:FGDVC_Result.FGDVC_Result.DictYields2Cols}\pysiglinewithargsret{\bfcode{DictYields2Cols}}{}{}
Returns the whole Dictionary Yield names to Columns of the matrix

\end{fulllineitems}

\index{FilePath() (FGDVC\_Result.FGDVC\_Result method)}

\begin{fulllineitems}
\phantomsection\label{FGDVCClasses:FGDVC_Result.FGDVC_Result.FilePath}\pysiglinewithargsret{\bfcode{FilePath}}{}{}
Returns the FG-DVC File path

\end{fulllineitems}

\index{FinalYields() (FGDVC\_Result.FGDVC\_Result method)}

\begin{fulllineitems}
\phantomsection\label{FGDVCClasses:FGDVC_Result.FGDVC_Result.FinalYields}\pysiglinewithargsret{\bfcode{FinalYields}}{}{}
Returns the last line of the Array, containing the yields at the time=time\_End

\end{fulllineitems}

\index{Name() (FGDVC\_Result.FGDVC\_Result method)}

\begin{fulllineitems}
\phantomsection\label{FGDVCClasses:FGDVC_Result.FGDVC_Result.Name}\pysiglinewithargsret{\bfcode{Name}}{}{}
returns `FG-DVC' as the name of the Program

\end{fulllineitems}

\index{Rates\_all() (FGDVC\_Result.FGDVC\_Result method)}

\begin{fulllineitems}
\phantomsection\label{FGDVCClasses:FGDVC_Result.FGDVC_Result.Rates_all}\pysiglinewithargsret{\bfcode{Rates\_all}}{}{}
Returns the whole result matrix of the Rates.

\end{fulllineitems}

\index{Yields\_all() (FGDVC\_Result.FGDVC\_Result method)}

\begin{fulllineitems}
\phantomsection\label{FGDVCClasses:FGDVC_Result.FGDVC_Result.Yields_all}\pysiglinewithargsret{\bfcode{Yields\_all}}{}{}
Returns the whole result matrix of the yields.

\end{fulllineitems}


\end{fulllineitems}



\chapter{The Fitting Classes}
\label{FittingClasses:the-fitting-classes}\label{FittingClasses::doc}

\section{The General Fitting Support Class}
\label{FittingClasses:the-general-fitting-support-class}\index{Fit\_one\_run (class in FitInfo)}

\begin{fulllineitems}
\phantomsection\label{FittingClasses:FitInfo.Fit_one_run}\pysiglinewithargsret{\strong{class }\code{FitInfo.}\bfcode{Fit\_one\_run}}{\emph{ResultObject}}{}
Imports from the Result objects the arrays. It provides the fitting objects with the yields and rates over time for the specific species. This class futher offers the option to plot the generated fitting results.
\index{Dt() (FitInfo.Fit\_one\_run method)}

\begin{fulllineitems}
\phantomsection\label{FittingClasses:FitInfo.Fit_one_run.Dt}\pysiglinewithargsret{\bfcode{Dt}}{}{}
Returns the vector with the time steps dt.

\end{fulllineitems}

\index{DtC() (FitInfo.Fit\_one\_run method)}

\begin{fulllineitems}
\phantomsection\label{FittingClasses:FitInfo.Fit_one_run.DtC}\pysiglinewithargsret{\bfcode{DtC}}{}{}
Returns the vector with the time steps dt\_C. This time steps are for points between the original points, so the lenght of this vector is the lenght of the time vector minus one.

\end{fulllineitems}

\index{Interpolate() (FitInfo.Fit\_one\_run method)}

\begin{fulllineitems}
\phantomsection\label{FittingClasses:FitInfo.Fit_one_run.Interpolate}\pysiglinewithargsret{\bfcode{Interpolate}}{\emph{Species}}{}
Outputs the interpolation object (e.g.Species(time)).

\end{fulllineitems}

\index{LineNumberMaxRate() (FitInfo.Fit\_one\_run method)}

\begin{fulllineitems}
\phantomsection\label{FittingClasses:FitInfo.Fit_one_run.LineNumberMaxRate}\pysiglinewithargsret{\bfcode{LineNumberMaxRate}}{\emph{Species}}{}
Returns the line with the maximum Rate of the inputted species.

\end{fulllineitems}

\index{MassCoal() (FitInfo.Fit\_one\_run method)}

\begin{fulllineitems}
\phantomsection\label{FittingClasses:FitInfo.Fit_one_run.MassCoal}\pysiglinewithargsret{\bfcode{MassCoal}}{}{}
returns the Vector with the solid mass(t)

\end{fulllineitems}

\index{MassVM\_s() (FitInfo.Fit\_one\_run method)}

\begin{fulllineitems}
\phantomsection\label{FittingClasses:FitInfo.Fit_one_run.MassVM_s}\pysiglinewithargsret{\bfcode{MassVM\_s}}{}{}
Returns the Vector with the mass of the volatile Matter over time

\end{fulllineitems}

\index{NPoints() (FitInfo.Fit\_one\_run method)}

\begin{fulllineitems}
\phantomsection\label{FittingClasses:FitInfo.Fit_one_run.NPoints}\pysiglinewithargsret{\bfcode{NPoints}}{}{}
returns number of Points for each species over time. Is equal the number of time points.

\end{fulllineitems}

\index{Name() (FitInfo.Fit\_one\_run method)}

\begin{fulllineitems}
\phantomsection\label{FittingClasses:FitInfo.Fit_one_run.Name}\pysiglinewithargsret{\bfcode{Name}}{}{}
Returns the Name of the imported Result object (e.g. `CPD')

\end{fulllineitems}

\index{Rate() (FitInfo.Fit\_one\_run method)}

\begin{fulllineitems}
\phantomsection\label{FittingClasses:FitInfo.Fit_one_run.Rate}\pysiglinewithargsret{\bfcode{Rate}}{\emph{species}}{}
Returns the Vector of the species rate(t). The species can be inputted with the Column number (integer) or the name corresponding to the dictionary saved in the result class (string).

\end{fulllineitems}

\index{RateSingleSpec() (FitInfo.Fit\_one\_run method)}

\begin{fulllineitems}
\phantomsection\label{FittingClasses:FitInfo.Fit_one_run.RateSingleSpec}\pysiglinewithargsret{\bfcode{RateSingleSpec}}{\emph{NameSpecies}}{}
Returns the Rate of the species (inputted as string) by calculate it from the yields by using a CDS

\end{fulllineitems}

\index{SpeciesIndex() (FitInfo.Fit\_one\_run method)}

\begin{fulllineitems}
\phantomsection\label{FittingClasses:FitInfo.Fit_one_run.SpeciesIndex}\pysiglinewithargsret{\bfcode{SpeciesIndex}}{\emph{species}}{}
Returns the species column number (integer) of the recieved species name (string)

\end{fulllineitems}

\index{SpeciesName() (FitInfo.Fit\_one\_run method)}

\begin{fulllineitems}
\phantomsection\label{FittingClasses:FitInfo.Fit_one_run.SpeciesName}\pysiglinewithargsret{\bfcode{SpeciesName}}{\emph{ColumnNumber}}{}
Returns the species name (string) of the recieved column number (integer)

\end{fulllineitems}

\index{SpeciesNames() (FitInfo.Fit\_one\_run method)}

\begin{fulllineitems}
\phantomsection\label{FittingClasses:FitInfo.Fit_one_run.SpeciesNames}\pysiglinewithargsret{\bfcode{SpeciesNames}}{}{}
Returns a list with all species names (including time and temperature).

\end{fulllineitems}

\index{Time() (FitInfo.Fit\_one\_run method)}

\begin{fulllineitems}
\phantomsection\label{FittingClasses:FitInfo.Fit_one_run.Time}\pysiglinewithargsret{\bfcode{Time}}{}{}
Returns the time vector

\end{fulllineitems}

\index{Yield() (FitInfo.Fit\_one\_run method)}

\begin{fulllineitems}
\phantomsection\label{FittingClasses:FitInfo.Fit_one_run.Yield}\pysiglinewithargsret{\bfcode{Yield}}{\emph{species}}{}
Returns the Vector of the species yield(t). The species can be inputted with the Column number (integer) or the name corresponding to the dictionary saved in the result class (string).

\end{fulllineitems}

\index{plt\_InputVectors() (FitInfo.Fit\_one\_run method)}

\begin{fulllineitems}
\phantomsection\label{FittingClasses:FitInfo.Fit_one_run.plt_InputVectors}\pysiglinewithargsret{\bfcode{plt\_InputVectors}}{\emph{xVector}, \emph{y1Vector}, \emph{y2Vector}, \emph{y3Vector}, \emph{y4Vector}, \emph{y1Name}, \emph{y2Name}, \emph{y3Name}, \emph{y4Name}}{}
Plots the y input Vectors vs. the x input vector.

\end{fulllineitems}

\index{plt\_RateVsTime() (FitInfo.Fit\_one\_run method)}

\begin{fulllineitems}
\phantomsection\label{FittingClasses:FitInfo.Fit_one_run.plt_RateVsTime}\pysiglinewithargsret{\bfcode{plt\_RateVsTime}}{\emph{ColumnNumber}}{}
Plots the original rates output of the pyrolysis program (as e.g. CPD) of the species marke with the columns number

\end{fulllineitems}

\index{plt\_YieldVsTime() (FitInfo.Fit\_one\_run method)}

\begin{fulllineitems}
\phantomsection\label{FittingClasses:FitInfo.Fit_one_run.plt_YieldVsTime}\pysiglinewithargsret{\bfcode{plt\_YieldVsTime}}{\emph{ColumnNumber}}{}
Plots the original yield output of the pyrolysis program (as e.g. CPD) of the species marke with the columns number

\end{fulllineitems}


\end{fulllineitems}



\section{A simple two point estimator for constant rate}
\label{FittingClasses:a-simple-two-point-estimator-for-constant-rate}\index{TwoPointEstimator (class in Fitter)}

\begin{fulllineitems}
\phantomsection\label{FittingClasses:Fitter.TwoPointEstimator}\pysigline{\strong{class }\code{Fitter.}\bfcode{TwoPointEstimator}}
Solves the devolatilization reaction analytically using two arbitrary selected points and the constant rate model. Unprecise. Should only be used for tests.

\end{fulllineitems}



\section{The Least Square Optimization Class}
\label{FittingClasses:the-least-square-optimization-class}\index{LeastSquarsEstimator (class in Fitter)}

\begin{fulllineitems}
\phantomsection\label{FittingClasses:Fitter.LeastSquarsEstimator}\pysigline{\strong{class }\code{Fitter.}\bfcode{LeastSquarsEstimator}}
Optimizes the Fitting curve using the Least Squares for the Yields and the Rates.
\index{Deviation() (Fitter.LeastSquarsEstimator method)}

\begin{fulllineitems}
\phantomsection\label{FittingClasses:Fitter.LeastSquarsEstimator.Deviation}\pysiglinewithargsret{\bfcode{Deviation}}{}{}
Returns the Deviation after the optimization procedure.

\end{fulllineitems}

\index{estimate\_T() (Fitter.LeastSquarsEstimator method)}

\begin{fulllineitems}
\phantomsection\label{FittingClasses:Fitter.LeastSquarsEstimator.estimate_T}\pysiglinewithargsret{\bfcode{estimate\_T}}{\emph{fgdvc\_list}, \emph{model}, \emph{Parameter\_Vector}, \emph{Name}, \emph{preLoopNumber=0}}{}
The main optimization method. Optimizes the Fitting curve using the Least Squares for the weighted Yields and the weighted Rates considering the temperatur history. Requires at input: The corresponding Fit\_one\_run object, the Model object, the kinetic parameter list, a name (e.g. the species). preLoopNumber is the number of running the  improve\_E and improve\_a routines. So the standard setting of preLoopNumber is equal zero. It may be used if there is only a very bad convergence when optimize all three parameter.

\end{fulllineitems}

\index{improve\_E() (Fitter.LeastSquarsEstimator method)}

\begin{fulllineitems}
\phantomsection\label{FittingClasses:Fitter.LeastSquarsEstimator.improve_E}\pysiglinewithargsret{\bfcode{improve\_E}}{\emph{fgdvc}, \emph{model}, \emph{t}, \emph{T}, \emph{Parameter\_Vector}, \emph{Name}}{}
Additional option: Only the Activation Energy in the Arrhenius Equation is optimized. Actual not necessary.

\end{fulllineitems}

\index{improve\_a() (Fitter.LeastSquarsEstimator method)}

\begin{fulllineitems}
\phantomsection\label{FittingClasses:Fitter.LeastSquarsEstimator.improve_a}\pysiglinewithargsret{\bfcode{improve\_a}}{\emph{fgdvc}, \emph{model}, \emph{t}, \emph{T}, \emph{Parameter\_Vector}, \emph{Name}}{}
Additional option: Only the preexponential factor in the Arrhenius Equation is optimized. Actual not necessary.

\end{fulllineitems}

\index{minLengthOfVectors() (Fitter.LeastSquarsEstimator method)}

\begin{fulllineitems}
\phantomsection\label{FittingClasses:Fitter.LeastSquarsEstimator.minLengthOfVectors}\pysiglinewithargsret{\bfcode{minLengthOfVectors}}{\emph{fgdvc\_list}}{}
Returns the minimum lenght of a all vectors from the several runs.

\end{fulllineitems}

\index{setMaxIter() (Fitter.LeastSquarsEstimator method)}

\begin{fulllineitems}
\phantomsection\label{FittingClasses:Fitter.LeastSquarsEstimator.setMaxIter}\pysiglinewithargsret{\bfcode{setMaxIter}}{\emph{MaxiumNumberOfIterationInMainProcedure}}{}
Sets the maximum number of iteration oin the optimizer as a abort criterion for the fitting procedure.

\end{fulllineitems}

\index{setOptimizer() (Fitter.LeastSquarsEstimator method)}

\begin{fulllineitems}
\phantomsection\label{FittingClasses:Fitter.LeastSquarsEstimator.setOptimizer}\pysiglinewithargsret{\bfcode{setOptimizer}}{\emph{ChosenOptimizer}}{}
Select one optimizer of the scipy.optimizer library: `fmin','fmin\_cg','fmin\_bfgs','fmin\_ncg','fmin\_slsqp' or `leastsq'. According to experience `fmin' (or also `leastsq') generates at best the results.

\end{fulllineitems}

\index{setPreMaxIter() (Fitter.LeastSquarsEstimator method)}

\begin{fulllineitems}
\phantomsection\label{FittingClasses:Fitter.LeastSquarsEstimator.setPreMaxIter}\pysiglinewithargsret{\bfcode{setPreMaxIter}}{\emph{MaxiumNumberOfIterationInPreProcedure}}{}
Sets the maximum number of iteration oin the optimizer as a abort criterion for the prefitting procedure (if preLoopNumber in estimate\_T is not equal zero).

\end{fulllineitems}

\index{setPreTolerance() (Fitter.LeastSquarsEstimator method)}

\begin{fulllineitems}
\phantomsection\label{FittingClasses:Fitter.LeastSquarsEstimator.setPreTolerance}\pysiglinewithargsret{\bfcode{setPreTolerance}}{\emph{ToleranceForFminFunction}}{}
Sets the tolerance as a abort criterion for the prefitting procedure (if preLoopNumber in estimate\_T is not equal zero).

\end{fulllineitems}

\index{setTolerance() (Fitter.LeastSquarsEstimator method)}

\begin{fulllineitems}
\phantomsection\label{FittingClasses:Fitter.LeastSquarsEstimator.setTolerance}\pysiglinewithargsret{\bfcode{setTolerance}}{\emph{ToleranceForFminFunction}}{}
Sets the tolerance as a abort criterion for the fitting procedure.

\end{fulllineitems}

\index{setWeights() (Fitter.LeastSquarsEstimator method)}

\begin{fulllineitems}
\phantomsection\label{FittingClasses:Fitter.LeastSquarsEstimator.setWeights}\pysiglinewithargsret{\bfcode{setWeights}}{\emph{WeightMass}, \emph{WeightRates}}{}
Sets the weights for the yields and the rates for the fitting procedure. See manual for equation.

\end{fulllineitems}


\end{fulllineitems}



\section{The Global Optimizer Class}
\label{FittingClasses:the-global-optimizer-class}\index{GlobalOptimizer (class in Fitter)}

\begin{fulllineitems}
\phantomsection\label{FittingClasses:Fitter.GlobalOptimizer}\pysiglinewithargsret{\strong{class }\code{Fitter.}\bfcode{GlobalOptimizer}}{\emph{localOptimizer}, \emph{KineticModel}, \emph{Fit\_one\_runObj}}{}
Makes runs over a defined range to look for global optimum. Local Optimizer is an LeastSquarsEstimator object, KineticModel is e.g. an constantRate model object, Fit\_one\_runObj is the List containing the Objects supporting the local fitting procedure with data.
\index{GenerateOptima() (Fitter.GlobalOptimizer method)}

\begin{fulllineitems}
\phantomsection\label{FittingClasses:Fitter.GlobalOptimizer.GenerateOptima}\pysiglinewithargsret{\bfcode{GenerateOptima}}{\emph{Species}, \emph{IndexListofParameterToOptimize}, \emph{ArrayOfRanges}, \emph{ListNrRuns}}{}
This method makes a several number of runs and returns the Parameter having the lowest deviation of all local minima. The list contains 3 or 4 parameters. If e.g., the second and the third Parameter have to be optimized: IndexListofParameterToOptimize={[}1,2{]}. If e.g. the range of the second is 10000 to 12000 and for the third 1 to 5,  ArrayOfRanges={[}{[}10000,12000{]},{[}1,5{]}{]}. When ListNrRuns is e.g. {[}0,10,5,0{]} the the second Parameter will be optimizted eleven times between 10000 and 12000, the third six times between 1 and 5. Attention, the number of runs grows by NrRuns1*NrRuns2*NrRuns3*NrRuns4, which can lead to a very large number needing very much time!

\end{fulllineitems}

\index{ParamList() (Fitter.GlobalOptimizer method)}

\begin{fulllineitems}
\phantomsection\label{FittingClasses:Fitter.GlobalOptimizer.ParamList}\pysiglinewithargsret{\bfcode{ParamList}}{}{}
Returns the Parameter list.

\end{fulllineitems}

\index{setParamList() (Fitter.GlobalOptimizer method)}

\begin{fulllineitems}
\phantomsection\label{FittingClasses:Fitter.GlobalOptimizer.setParamList}\pysiglinewithargsret{\bfcode{setParamList}}{\emph{ParameterList}}{}
Sets the Parameter list.

\end{fulllineitems}


\end{fulllineitems}



\section{The Evolution algorithm  Class}
\label{FittingClasses:the-evolution-algorithm-class}\index{GenericOpt (class in Evolve)}

\begin{fulllineitems}
\phantomsection\label{FittingClasses:Evolve.GenericOpt}\pysiglinewithargsret{\strong{class }\code{Evolve.}\bfcode{GenericOpt}}{\emph{KineticModel}, \emph{Fit\_one\_runObj}, \emph{Species}}{}
Class which uses the pyevolve module to search for the global optimum.To initialize use the Kinetic model (e.g. Kobayashi) and the Fit one run object list, which supports the fitting process with the informations.
\index{\_GenericOpt\_\_UpdateParam() (Evolve.GenericOpt method)}

\begin{fulllineitems}
\phantomsection\label{FittingClasses:Evolve.GenericOpt._GenericOpt__UpdateParam}\pysiglinewithargsret{\bfcode{\_GenericOpt\_\_UpdateParam}}{}{}
Updates the non-scaled Parameter. Updates the non-scaled parameter vector with the values of the sclaed vector

\end{fulllineitems}

\index{mkResults() (Evolve.GenericOpt method)}

\begin{fulllineitems}
\phantomsection\label{FittingClasses:Evolve.GenericOpt.mkResults}\pysiglinewithargsret{\bfcode{mkResults}}{}{}
Generates the result.

\end{fulllineitems}

\index{setNrGenerations() (Evolve.GenericOpt method)}

\begin{fulllineitems}
\phantomsection\label{FittingClasses:Evolve.GenericOpt.setNrGenerations}\pysiglinewithargsret{\bfcode{setNrGenerations}}{\emph{NrOfGenerations}}{}
Defines the number of generations for the generic optimization.

\end{fulllineitems}

\index{setNrPopulation() (Evolve.GenericOpt method)}

\begin{fulllineitems}
\phantomsection\label{FittingClasses:Evolve.GenericOpt.setNrPopulation}\pysiglinewithargsret{\bfcode{setNrPopulation}}{\emph{NrOfPopulation}}{}
Defines the size of the population for the generic optimization.

\end{fulllineitems}

\index{setParamRanges() (Evolve.GenericOpt method)}

\begin{fulllineitems}
\phantomsection\label{FittingClasses:Evolve.GenericOpt.setParamRanges}\pysiglinewithargsret{\bfcode{setParamRanges}}{\emph{InitialGuess}, \emph{minimum}, \emph{maximum}}{}
Sets the range where to evolve.

\end{fulllineitems}

\index{setScaledParameter() (Evolve.GenericOpt method)}

\begin{fulllineitems}
\phantomsection\label{FittingClasses:Evolve.GenericOpt.setScaledParameter}\pysiglinewithargsret{\bfcode{setScaledParameter}}{\emph{Parameter}}{}
Sets Sclaed Parameter equal to the input parameter.

\end{fulllineitems}

\index{setWeights() (Evolve.GenericOpt method)}

\begin{fulllineitems}
\phantomsection\label{FittingClasses:Evolve.GenericOpt.setWeights}\pysiglinewithargsret{\bfcode{setWeights}}{\emph{WeightY}, \emph{WeightR}}{}
Sets the yield and the weight rate for the optimization equation.

\end{fulllineitems}


\end{fulllineitems}



\section{The Model parent Class}
\label{FittingClasses:the-model-parent-class}\index{Model (class in Models)}

\begin{fulllineitems}
\phantomsection\label{FittingClasses:Models.Model}\pysigline{\strong{class }\code{Models.}\bfcode{Model}}
Parent class of the children ConstantRateModel, the three Arrhenius Models (notations) and the Kobayashi models.
\index{ErrorRate() (Models.Model method)}

\begin{fulllineitems}
\phantomsection\label{FittingClasses:Models.Model.ErrorRate}\pysiglinewithargsret{\bfcode{ErrorRate}}{\emph{fgdvc}, \emph{Species}}{}
Returns the absolute deviation per point between the fitted and the original rate curve.

\end{fulllineitems}

\index{ErrorYield() (Models.Model method)}

\begin{fulllineitems}
\phantomsection\label{FittingClasses:Models.Model.ErrorYield}\pysiglinewithargsret{\bfcode{ErrorYield}}{\emph{fgdvc}, \emph{Species}}{}
Returns the absolute deviation per point between the fitted and the original yield curve.

\end{fulllineitems}

\index{ParamVector() (Models.Model method)}

\begin{fulllineitems}
\phantomsection\label{FittingClasses:Models.Model.ParamVector}\pysiglinewithargsret{\bfcode{ParamVector}}{}{}
Returns the Vector containing the kinetic parameter of the Model (refering to the child model).

\end{fulllineitems}

\index{calcRate() (Models.Model method)}

\begin{fulllineitems}
\phantomsection\label{FittingClasses:Models.Model.calcRate}\pysiglinewithargsret{\bfcode{calcRate}}{\emph{fgdvc}, \emph{time}, \emph{T}, \emph{Name}}{}
Generates the Rates using the yields vector and a CDS.

\end{fulllineitems}

\index{deriveC() (Models.Model method)}

\begin{fulllineitems}
\phantomsection\label{FittingClasses:Models.Model.deriveC}\pysiglinewithargsret{\bfcode{deriveC}}{\emph{fgdvc}, \emph{yVector}, \emph{maxVectorLenght=None}}{}
Returns a CDS of the inputted yVector.

\end{fulllineitems}

\index{minLengthOfVectors() (Models.Model method)}

\begin{fulllineitems}
\phantomsection\label{FittingClasses:Models.Model.minLengthOfVectors}\pysiglinewithargsret{\bfcode{minLengthOfVectors}}{\emph{fgdvc\_list}}{}
Returns the minimum lenght of a all vectors from the several runs.

\end{fulllineitems}

\index{plot() (Models.Model method)}

\begin{fulllineitems}
\phantomsection\label{FittingClasses:Models.Model.plot}\pysiglinewithargsret{\bfcode{plot}}{\emph{fgdvc\_list}, \emph{Species}}{}
Plot the yield and the rates over time with two curves: one is the original data, the other the fitting curve. Also file `PyrolysisProgramName-Species.out' (e.g. `CPD-CO2.out') containing the time (s), yields (kg/kg), rates (kg/(kg s)).

\end{fulllineitems}

\index{pltRate() (Models.Model method)}

\begin{fulllineitems}
\phantomsection\label{FittingClasses:Models.Model.pltRate}\pysiglinewithargsret{\bfcode{pltRate}}{\emph{fgdvc\_list}, \emph{xValueToPlot}, \emph{yValueToPlot}}{}
Plots the rates (to select with yValueToPlot) over Time or Temperature (to slect with xValueToPlot).

\end{fulllineitems}

\index{pltYield() (Models.Model method)}

\begin{fulllineitems}
\phantomsection\label{FittingClasses:Models.Model.pltYield}\pysiglinewithargsret{\bfcode{pltYield}}{\emph{fgdvc\_list}, \emph{xValueToPlot}, \emph{yValueToPlot}}{}
Plots the yields (to select with yValueToPlot) over Time or Temperature (to slect with xValueToPlot).

\end{fulllineitems}

\index{setParamVector() (Models.Model method)}

\begin{fulllineitems}
\phantomsection\label{FittingClasses:Models.Model.setParamVector}\pysiglinewithargsret{\bfcode{setParamVector}}{\emph{ParameterList}}{}
Sets the Vector containing the kinetic parameter of the Model (refering to the child model).

\end{fulllineitems}


\end{fulllineitems}



\section{The Model children Classes}
\label{FittingClasses:the-model-children-classes}

\subsection{The Constant Rate Model}
\label{FittingClasses:the-constant-rate-model}\index{ConstantRateModel (class in Models)}

\begin{fulllineitems}
\phantomsection\label{FittingClasses:Models.ConstantRateModel}\pysiglinewithargsret{\strong{class }\code{Models.}\bfcode{ConstantRateModel}}{\emph{InitialParameterVector}}{}
The model calculating the mass with m(t)=m\_s0+(m\_s0-m\_s,e)*e**(-k*(t-t\_start)) from the ODE dm/dt = -k*(m-m\_s,e). The Parameter to optimize are k and t\_start.

\end{fulllineitems}



\subsection{The Arrhenius Model}
\label{FittingClasses:the-arrhenius-model}\index{ArrheniusModel (class in Models)}

\begin{fulllineitems}
\phantomsection\label{FittingClasses:Models.ArrheniusModel}\pysiglinewithargsret{\strong{class }\code{Models.}\bfcode{ArrheniusModel}}{\emph{InitialParameterVector}}{}
The Arrhenius model in the standart notation: dm/dt=A*(T**b)*exp(-E/T)*(m\_s-m) with the parameter a,b,E to optimize.
\index{ConvertKinFactors() (Models.ArrheniusModel method)}

\begin{fulllineitems}
\phantomsection\label{FittingClasses:Models.ArrheniusModel.ConvertKinFactors}\pysiglinewithargsret{\bfcode{ConvertKinFactors}}{\emph{ParameterVector}}{}
Dummy. Function actual has to convert the parameter into the standart Arrhenius notation.

\end{fulllineitems}

\index{calcMass() (Models.ArrheniusModel method)}

\begin{fulllineitems}
\phantomsection\label{FittingClasses:Models.ArrheniusModel.calcMass}\pysiglinewithargsret{\bfcode{calcMass}}{\emph{fgdvc}, \emph{time}, \emph{T}, \emph{Name}}{}
Outputs the mass(t) using the model specific equation.

\end{fulllineitems}


\end{fulllineitems}



\subsection{The Arrhenius Model with no correction Term}
\label{FittingClasses:the-arrhenius-model-with-no-correction-term}\index{ArrheniusModelAlternativeNotation2 (class in Models)}

\begin{fulllineitems}
\phantomsection\label{FittingClasses:Models.ArrheniusModelAlternativeNotation2}\pysiglinewithargsret{\strong{class }\code{Models.}\bfcode{ArrheniusModelAlternativeNotation2}}{\emph{InitialParameterVector}}{}
Arrhenius model with a notation having a better optimization behaviour: dm/dt=exp{[}c*(b1*(1/T(t)-1/T\_min)-b2*(1/T(t)-1/T\_max)){]}*(ms-m); with c=(1/T\_max-1/Tmin)**(-1). See the documentation for the reference. The parameters to optimize are b1 and b2.
\index{ConvertKinFactors() (Models.ArrheniusModelAlternativeNotation2 method)}

\begin{fulllineitems}
\phantomsection\label{FittingClasses:Models.ArrheniusModelAlternativeNotation2.ConvertKinFactors}\pysiglinewithargsret{\bfcode{ConvertKinFactors}}{\emph{ParameterVector}}{}
Converts the own kinetic factors back to the standard Arrhenius kinetic factors.

\end{fulllineitems}

\index{ConvertKinFactorsToOwnNotation() (Models.ArrheniusModelAlternativeNotation2 method)}

\begin{fulllineitems}
\phantomsection\label{FittingClasses:Models.ArrheniusModelAlternativeNotation2.ConvertKinFactorsToOwnNotation}\pysiglinewithargsret{\bfcode{ConvertKinFactorsToOwnNotation}}{\emph{ParameterVector}}{}
Converts the standard Arrhenius kinetic factors back to the factors of the own notation.

\end{fulllineitems}

\index{calcMass() (Models.ArrheniusModelAlternativeNotation2 method)}

\begin{fulllineitems}
\phantomsection\label{FittingClasses:Models.ArrheniusModelAlternativeNotation2.calcMass}\pysiglinewithargsret{\bfcode{calcMass}}{\emph{fgdvc}, \emph{time}, \emph{T}, \emph{Name}}{}
Outputs the mass(t) using the model specific equation.

\end{fulllineitems}

\index{setMinMaxTemp() (Models.ArrheniusModelAlternativeNotation2 method)}

\begin{fulllineitems}
\phantomsection\label{FittingClasses:Models.ArrheniusModelAlternativeNotation2.setMinMaxTemp}\pysiglinewithargsret{\bfcode{setMinMaxTemp}}{\emph{Tmin}, \emph{Tmax}}{}
Sets the temperature constants, see the equation.

\end{fulllineitems}


\end{fulllineitems}



\subsection{The Kobayashi Models}
\label{FittingClasses:the-kobayashi-models}
The first of these two models has the Parameter A1, A2, E1, and E2 to optimize. The second one has like  PC Coal Lab the four parameter A1,A2,E1 and alpha1 to optimize.
\index{Kobayashi (class in Models)}

\begin{fulllineitems}
\phantomsection\label{FittingClasses:Models.Kobayashi}\pysiglinewithargsret{\strong{class }\code{Models.}\bfcode{Kobayashi}}{\emph{InitialParameterVector}}{}
Calculates the devolatilization reaction using the Kobayashi model. The Arrhenius equation inside are in the standard notation.
\index{calcMass() (Models.Kobayashi method)}

\begin{fulllineitems}
\phantomsection\label{FittingClasses:Models.Kobayashi.calcMass}\pysiglinewithargsret{\bfcode{calcMass}}{\emph{fgdvc}, \emph{time}, \emph{T}, \emph{Name}}{}
Outputs the mass(t) using the model specific equation. The input Vector is {[}A1,E1,A2,E2,alpha1,alpha2{]}

\end{fulllineitems}


\end{fulllineitems}

\index{KobayashiPCCL (class in Models)}

\begin{fulllineitems}
\phantomsection\label{FittingClasses:Models.KobayashiPCCL}\pysiglinewithargsret{\strong{class }\code{Models.}\bfcode{KobayashiPCCL}}{\emph{InitialParameterVector}}{}
Calculates the devolatilization reaction using the Kobayashi model. The Arrhenius equation inside are in the standard notation. The fitting parameter are as in PCCL A1,A2,E1,alpha1.
\index{ConvertKinFactors() (Models.KobayashiPCCL method)}

\begin{fulllineitems}
\phantomsection\label{FittingClasses:Models.KobayashiPCCL.ConvertKinFactors}\pysiglinewithargsret{\bfcode{ConvertKinFactors}}{\emph{ParameterVector}}{}
Outputs the Arrhenius equation factors in the shape {[}A1,E1,A2,E2{]}. Here where the real Arrhenius model is in use only a dummy function.

\end{fulllineitems}

\index{E2Diff() (Models.KobayashiPCCL method)}

\begin{fulllineitems}
\phantomsection\label{FittingClasses:Models.KobayashiPCCL.E2Diff}\pysiglinewithargsret{\bfcode{E2Diff}}{}{}
Returns the dE in E2=E1+dE.

\end{fulllineitems}

\index{KobWeights() (Models.KobayashiPCCL method)}

\begin{fulllineitems}
\phantomsection\label{FittingClasses:Models.KobayashiPCCL.KobWeights}\pysiglinewithargsret{\bfcode{KobWeights}}{}{}
Returns the two Kobayashi weights alpha2.

\end{fulllineitems}

\index{calcMass() (Models.KobayashiPCCL method)}

\begin{fulllineitems}
\phantomsection\label{FittingClasses:Models.KobayashiPCCL.calcMass}\pysiglinewithargsret{\bfcode{calcMass}}{\emph{fgdvc}, \emph{time}, \emph{T}, \emph{Name}}{}
Outputs the mass(t) using the model specific equation.

\end{fulllineitems}

\index{setE2Diff() (Models.KobayashiPCCL method)}

\begin{fulllineitems}
\phantomsection\label{FittingClasses:Models.KobayashiPCCL.setE2Diff}\pysiglinewithargsret{\bfcode{setE2Diff}}{\emph{DifferenceE1E2}}{}
Sets the dE in E2=E1+dE.

\end{fulllineitems}

\index{setKobWeights() (Models.KobayashiPCCL method)}

\begin{fulllineitems}
\phantomsection\label{FittingClasses:Models.KobayashiPCCL.setKobWeights}\pysiglinewithargsret{\bfcode{setKobWeights}}{\emph{alpha2}}{}
Sets the two Kobayashi weights alpha2.

\end{fulllineitems}


\end{fulllineitems}



\chapter{The Input- Output Classes}
\label{InputOutputClasses::doc}\label{InputOutputClasses:the-input-output-classes}

\section{The General User Input File Class}
\label{InputOutputClasses:the-general-user-input-file-class}\index{ReadFile (class in InformationFiles)}

\begin{fulllineitems}
\phantomsection\label{InputOutputClasses:InformationFiles.ReadFile}\pysiglinewithargsret{\strong{class }\code{InformationFiles.}\bfcode{ReadFile}}{\emph{InputFile}}{}
general parent class for the reading objects CPDFile and FGDVCFile
\index{Fitting() (InformationFiles.ReadFile method)}

\begin{fulllineitems}
\phantomsection\label{InputOutputClasses:InformationFiles.ReadFile.Fitting}\pysiglinewithargsret{\bfcode{Fitting}}{\emph{FileNote}}{}
outputs the Fitting mode for Pyrolysis Program output (string: `constantRate','Arrhenius','Kobayashi'). Possible input: `constantRate', `Arrhenius' or `Kobayashi'

\end{fulllineitems}

\index{UsePyrolProgr() (InformationFiles.ReadFile method)}

\begin{fulllineitems}
\phantomsection\label{InputOutputClasses:InformationFiles.ReadFile.UsePyrolProgr}\pysiglinewithargsret{\bfcode{UsePyrolProgr}}{\emph{FileNote}}{}
gets the information, whether Pyrolsis Program will be in use. Enter `Yes' or `True' for the case it should be used

\end{fulllineitems}

\index{getText() (InformationFiles.ReadFile method)}

\begin{fulllineitems}
\phantomsection\label{InputOutputClasses:InformationFiles.ReadFile.getText}\pysiglinewithargsret{\bfcode{getText}}{\emph{FileNote}}{}
output the data of the line below the FileNote as a string

\end{fulllineitems}

\index{getValue() (InformationFiles.ReadFile method)}

\begin{fulllineitems}
\phantomsection\label{InputOutputClasses:InformationFiles.ReadFile.getValue}\pysiglinewithargsret{\bfcode{getValue}}{\emph{FileNote}}{}
output the data of the line below the FileNote as a float

\end{fulllineitems}

\index{readLines() (InformationFiles.ReadFile method)}

\begin{fulllineitems}
\phantomsection\label{InputOutputClasses:InformationFiles.ReadFile.readLines}\pysiglinewithargsret{\bfcode{readLines}}{}{}
reads the input File line by line

\end{fulllineitems}


\end{fulllineitems}



\section{The Class writing the FG-DVC Coal Generator Input File}
\label{InputOutputClasses:the-class-writing-the-fg-dvc-coal-generator-input-file}\index{WriteFGDVCCoalFile (class in InformationFiles)}

\begin{fulllineitems}
\phantomsection\label{InputOutputClasses:InformationFiles.WriteFGDVCCoalFile}\pysiglinewithargsret{\strong{class }\code{InformationFiles.}\bfcode{WriteFGDVCCoalFile}}{\emph{CoalGenFile}}{}
writes the file, which will be inputted into the FG-DVC coal generator
\index{setCoalComp() (InformationFiles.WriteFGDVCCoalFile method)}

\begin{fulllineitems}
\phantomsection\label{InputOutputClasses:InformationFiles.WriteFGDVCCoalFile.setCoalComp}\pysiglinewithargsret{\bfcode{setCoalComp}}{\emph{Carbon}, \emph{Hydrogen}, \emph{Oxygen}, \emph{Nitrogen}, \emph{Sulfur}, \emph{SulfurPyrite}}{}
Enter the coal composition with values in percent which have to sum up to 100

\end{fulllineitems}

\index{write() (InformationFiles.WriteFGDVCCoalFile method)}

\begin{fulllineitems}
\phantomsection\label{InputOutputClasses:InformationFiles.WriteFGDVCCoalFile.write}\pysiglinewithargsret{\bfcode{write}}{\emph{CoalsDirectory}, \emph{CoalResultFileName}}{}
writes the FG-DVC coal generator input file

\end{fulllineitems}


\end{fulllineitems}



\section{The Class reading the Operating Condition Input File}
\label{InputOutputClasses:the-class-reading-the-operating-condition-input-file}\index{OperCondInput (class in InformationFiles)}

\begin{fulllineitems}
\phantomsection\label{InputOutputClasses:InformationFiles.OperCondInput}\pysiglinewithargsret{\strong{class }\code{InformationFiles.}\bfcode{OperCondInput}}{\emph{InputFile}}{}
Reads the input file for the operating conditions and also writes the temperature-history file, required by FG-DVC.
\index{getTimePoints() (InformationFiles.OperCondInput method)}

\begin{fulllineitems}
\phantomsection\label{InputOutputClasses:InformationFiles.OperCondInput.getTimePoints}\pysiglinewithargsret{\bfcode{getTimePoints}}{\emph{FileNoteBegin}, \emph{FileNoteEnd}}{}
reads the time points in the shape `time, temperature' for the lines between the line with the FileNoteBegin and the line with the FileNoteEnd

\end{fulllineitems}

\index{writeFGDVCtTHist() (InformationFiles.OperCondInput method)}

\begin{fulllineitems}
\phantomsection\label{InputOutputClasses:InformationFiles.OperCondInput.writeFGDVCtTHist}\pysiglinewithargsret{\bfcode{writeFGDVCtTHist}}{\emph{tTPoints}, \emph{dt}, \emph{OutputFilePath}}{}
Writes output file for FG-DVC containing in first column time in s, in the second tempearure in degree Celsius. FG-DVC will import this file. The time-temperature array has to be a numpy.array, dt a float, OutputFilePath a string.

\end{fulllineitems}


\end{fulllineitems}



\chapter{The GUI Classes}
\label{GUI:the-gui-classes}\label{GUI::doc}

\section{The Main Window}
\label{GUI:the-main-window}\index{Ui\_PKP (class in GUI)}

\begin{fulllineitems}
\phantomsection\label{GUI:GUI.Ui_PKP}\pysigline{\strong{class }\code{GUI.}\bfcode{Ui\_PKP}}~\index{LoadTtFile1() (GUI.Ui\_PKP method)}

\begin{fulllineitems}
\phantomsection\label{GUI:GUI.Ui_PKP.LoadTtFile1}\pysiglinewithargsret{\bfcode{LoadTtFile1}}{}{}
Loads the temperature history nr 1 file via file browser

\end{fulllineitems}

\index{LoadTtFile2() (GUI.Ui\_PKP method)}

\begin{fulllineitems}
\phantomsection\label{GUI:GUI.Ui_PKP.LoadTtFile2}\pysiglinewithargsret{\bfcode{LoadTtFile2}}{}{}
Loads the temperature history nr 2 file via file browser

\end{fulllineitems}

\index{LoadTtFile3() (GUI.Ui\_PKP method)}

\begin{fulllineitems}
\phantomsection\label{GUI:GUI.Ui_PKP.LoadTtFile3}\pysiglinewithargsret{\bfcode{LoadTtFile3}}{}{}
Loads the temperature history nr 3 file via file browser

\end{fulllineitems}

\index{LoadTtFile4() (GUI.Ui\_PKP method)}

\begin{fulllineitems}
\phantomsection\label{GUI:GUI.Ui_PKP.LoadTtFile4}\pysiglinewithargsret{\bfcode{LoadTtFile4}}{}{}
Loads the temperature history nr 4 file via file browser

\end{fulllineitems}

\index{LoadTtFile5() (GUI.Ui\_PKP method)}

\begin{fulllineitems}
\phantomsection\label{GUI:GUI.Ui_PKP.LoadTtFile5}\pysiglinewithargsret{\bfcode{LoadTtFile5}}{}{}
Loads the temperature history nr 5 file via file browser

\end{fulllineitems}

\index{Plot1() (GUI.Ui\_PKP method)}

\begin{fulllineitems}
\phantomsection\label{GUI:GUI.Ui_PKP.Plot1}\pysiglinewithargsret{\bfcode{Plot1}}{}{}
Plots the temperature over time history (temperature history nr 1) and saves temperatuer history in ``TempHist1.dat''.

\end{fulllineitems}

\index{Plot2() (GUI.Ui\_PKP method)}

\begin{fulllineitems}
\phantomsection\label{GUI:GUI.Ui_PKP.Plot2}\pysiglinewithargsret{\bfcode{Plot2}}{}{}
Plots the temperature over time history (temperature history nr 2) and saves temperatuer history in ``TempHist2.dat''.

\end{fulllineitems}

\index{Plot3() (GUI.Ui\_PKP method)}

\begin{fulllineitems}
\phantomsection\label{GUI:GUI.Ui_PKP.Plot3}\pysiglinewithargsret{\bfcode{Plot3}}{}{}
Plots the temperature over time history (temperature history nr 3) and saves temperatuer history in ``TempHist3.dat''.

\end{fulllineitems}

\index{Plot4() (GUI.Ui\_PKP method)}

\begin{fulllineitems}
\phantomsection\label{GUI:GUI.Ui_PKP.Plot4}\pysiglinewithargsret{\bfcode{Plot4}}{}{}
Plots the temperature over time history (temperature history nr 4) and saves temperatuer history in ``TempHist4.dat''.

\end{fulllineitems}

\index{Plot5() (GUI.Ui\_PKP method)}

\begin{fulllineitems}
\phantomsection\label{GUI:GUI.Ui_PKP.Plot5}\pysiglinewithargsret{\bfcode{Plot5}}{}{}
Plots the temperature over time history (temperature history nr 5) and saves temperatuer history in ``TempHist5.dat''.

\end{fulllineitems}

\index{SaveInfos() (GUI.Ui\_PKP method)}

\begin{fulllineitems}
\phantomsection\label{GUI:GUI.Ui_PKP.SaveInfos}\pysiglinewithargsret{\bfcode{SaveInfos}}{}{}
Saves the Information when the save or the run -option is used.

\end{fulllineitems}


\end{fulllineitems}



\section{The Class Saving the information from the GUI}
\label{GUI:the-class-saving-the-information-from-the-gui}\index{InfosFromGUI (class in GUI)}

\begin{fulllineitems}
\phantomsection\label{GUI:GUI.InfosFromGUI}\pysigline{\strong{class }\code{GUI.}\bfcode{InfosFromGUI}}
Saves the information from the GUI.
\index{ArrhSpec() (GUI.InfosFromGUI method)}

\begin{fulllineitems}
\phantomsection\label{GUI:GUI.InfosFromGUI.ArrhSpec}\pysiglinewithargsret{\bfcode{ArrhSpec}}{}{}
Returns which species shall be fitted for Arrhenius.

\end{fulllineitems}

\index{ArrhSpecReverse() (GUI.InfosFromGUI method)}

\begin{fulllineitems}
\phantomsection\label{GUI:GUI.InfosFromGUI.ArrhSpecReverse}\pysiglinewithargsret{\bfcode{ArrhSpecReverse}}{\emph{SpeciesName}}{}
returns the UI columns bar index of species fitted for Arrhenius.

\end{fulllineitems}

\index{FGCoalProp() (GUI.InfosFromGUI method)}

\begin{fulllineitems}
\phantomsection\label{GUI:GUI.InfosFromGUI.FGCoalProp}\pysiglinewithargsret{\bfcode{FGCoalProp}}{}{}
Defines the way of the FG-DVC Coal Fitting and the Tar Modeling.

\end{fulllineitems}

\index{MwsHHV() (GUI.InfosFromGUI method)}

\begin{fulllineitems}
\phantomsection\label{GUI:GUI.InfosFromGUI.MwsHHV}\pysiglinewithargsret{\bfcode{MwsHHV}}{}{}
Retruns the Molar Weight of Tar and sets the Higher Heating Value.

\end{fulllineitems}

\index{OperCond() (GUI.InfosFromGUI method)}

\begin{fulllineitems}
\phantomsection\label{GUI:GUI.InfosFromGUI.OperCond}\pysiglinewithargsret{\bfcode{OperCond}}{}{}
Sets the pressure and the time step.

\end{fulllineitems}

\index{PA() (GUI.InfosFromGUI method)}

\begin{fulllineitems}
\phantomsection\label{GUI:GUI.InfosFromGUI.PA}\pysiglinewithargsret{\bfcode{PA}}{}{}
Returns the coal PA properties.

\end{fulllineitems}

\index{RunPyrolProg() (GUI.InfosFromGUI method)}

\begin{fulllineitems}
\phantomsection\label{GUI:GUI.InfosFromGUI.RunPyrolProg}\pysiglinewithargsret{\bfcode{RunPyrolProg}}{}{}
Returns which options of the three Pyrolysis programs are used.

\end{fulllineitems}

\index{RunPyrolProgReverse() (GUI.InfosFromGUI method)}

\begin{fulllineitems}
\phantomsection\label{GUI:GUI.InfosFromGUI.RunPyrolProgReverse}\pysiglinewithargsret{\bfcode{RunPyrolProgReverse}}{\emph{ModelName}}{}
Returns the GUI column bar index of the models selected.

\end{fulllineitems}

\index{SetArrhSpec() (GUI.InfosFromGUI method)}

\begin{fulllineitems}
\phantomsection\label{GUI:GUI.InfosFromGUI.SetArrhSpec}\pysiglinewithargsret{\bfcode{SetArrhSpec}}{\emph{SpeciesIndex}}{}
Sets which species shall be fitted for Arrhenius.

\end{fulllineitems}

\index{SetRunPyrolProg() (GUI.InfosFromGUI method)}

\begin{fulllineitems}
\phantomsection\label{GUI:GUI.InfosFromGUI.SetRunPyrolProg}\pysiglinewithargsret{\bfcode{SetRunPyrolProg}}{\emph{CPDIndex}, \emph{FGDVCIndex}, \emph{PCCLIndex}}{}
Saves which options of the three Pyrolysis programs are used.

\end{fulllineitems}

\index{TimeHistories() (GUI.InfosFromGUI method)}

\begin{fulllineitems}
\phantomsection\label{GUI:GUI.InfosFromGUI.TimeHistories}\pysiglinewithargsret{\bfcode{TimeHistories}}{}{}
Returns the number of time history.

\end{fulllineitems}

\index{UA() (GUI.InfosFromGUI method)}

\begin{fulllineitems}
\phantomsection\label{GUI:GUI.InfosFromGUI.UA}\pysiglinewithargsret{\bfcode{UA}}{}{}
Returns the coal UA properties.

\end{fulllineitems}

\index{WeightYR() (GUI.InfosFromGUI method)}

\begin{fulllineitems}
\phantomsection\label{GUI:GUI.InfosFromGUI.WeightYR}\pysiglinewithargsret{\bfcode{WeightYR}}{}{}
Returns the weghts for Yields and Rates

\end{fulllineitems}

\index{setFGCoalProp() (GUI.InfosFromGUI method)}

\begin{fulllineitems}
\phantomsection\label{GUI:GUI.InfosFromGUI.setFGCoalProp}\pysiglinewithargsret{\bfcode{setFGCoalProp}}{\emph{FGCoalFit}, \emph{FGTarModeling}}{}
Defines the way of the FG-DVC Coal Fitting and the Tar Modeling.

\end{fulllineitems}

\index{setMwsHHV() (GUI.InfosFromGUI method)}

\begin{fulllineitems}
\phantomsection\label{GUI:GUI.InfosFromGUI.setMwsHHV}\pysiglinewithargsret{\bfcode{setMwsHHV}}{\emph{MoleWweightTar}, \emph{HHV}}{}
Sets the Molar Weight of Tar and sets the Higher Heating Value.

\end{fulllineitems}

\index{setOperCond() (GUI.InfosFromGUI method)}

\begin{fulllineitems}
\phantomsection\label{GUI:GUI.InfosFromGUI.setOperCond}\pysiglinewithargsret{\bfcode{setOperCond}}{\emph{pressure}, \emph{timestep}}{}
Sets the pressure and the time step.

\end{fulllineitems}

\index{setPA() (GUI.InfosFromGUI method)}

\begin{fulllineitems}
\phantomsection\label{GUI:GUI.InfosFromGUI.setPA}\pysiglinewithargsret{\bfcode{setPA}}{\emph{FixedCarbon}, \emph{VolatileMatter}, \emph{Moisture}, \emph{Ash}}{}
Saves the coal PA properties.

\end{fulllineitems}

\index{setTimeHistories() (GUI.InfosFromGUI method)}

\begin{fulllineitems}
\phantomsection\label{GUI:GUI.InfosFromGUI.setTimeHistories}\pysiglinewithargsret{\bfcode{setTimeHistories}}{\emph{NrOfTs}}{}
Saves the Number time history

\end{fulllineitems}

\index{setUA() (GUI.InfosFromGUI method)}

\begin{fulllineitems}
\phantomsection\label{GUI:GUI.InfosFromGUI.setUA}\pysiglinewithargsret{\bfcode{setUA}}{\emph{UAC}, \emph{UAH}, \emph{UAN}, \emph{UAO}, \emph{UAS}}{}
Saves the coal UA properties.

\end{fulllineitems}

\index{setWeightYR() (GUI.InfosFromGUI method)}

\begin{fulllineitems}
\phantomsection\label{GUI:GUI.InfosFromGUI.setWeightYR}\pysiglinewithargsret{\bfcode{setWeightYR}}{\emph{Y}, \emph{R}}{}
Sets the weghts for Yields and Rates

\end{fulllineitems}


\end{fulllineitems}



\section{The Classes writing the Info Files}
\label{GUI:the-classes-writing-the-info-files}

\subsection{The Coal File}
\label{GUI:the-coal-file}\index{WriteCoalFile (class in writeInfoFiles)}

\begin{fulllineitems}
\phantomsection\label{GUI:writeInfoFiles.WriteCoalFile}\pysiglinewithargsret{\strong{class }\code{writeInfoFiles.}\bfcode{WriteCoalFile}}{\emph{InfosFromGUIObject}}{}
Writes the Coal.inp file using the output of the GUI.

\end{fulllineitems}



\subsection{The CPD File}
\label{GUI:the-cpd-file}\index{WriteCPDFile (class in writeInfoFiles)}

\begin{fulllineitems}
\phantomsection\label{GUI:writeInfoFiles.WriteCPDFile}\pysiglinewithargsret{\strong{class }\code{writeInfoFiles.}\bfcode{WriteCPDFile}}{\emph{InfosFromGUIObject}}{}
Writes the CPD.inp file using the output of the GUI. The number of print increment is imorted from the previous version of CPD.inp.

\end{fulllineitems}



\subsection{The FG-DVC File}
\label{GUI:the-fg-dvc-file}\index{WriteFGFile (class in writeInfoFiles)}

\begin{fulllineitems}
\phantomsection\label{GUI:writeInfoFiles.WriteFGFile}\pysiglinewithargsret{\strong{class }\code{writeInfoFiles.}\bfcode{WriteFGFile}}{\emph{InfosFromGUIObject}}{}
Writes the FGDVC.inp file using the output of the GUI. The filepaths are imorted from the previous version of FGDVC.inp.

\end{fulllineitems}



\subsection{The Operating Condition File}
\label{GUI:the-operating-condition-file}\index{WriteOCFile (class in writeInfoFiles)}

\begin{fulllineitems}
\phantomsection\label{GUI:writeInfoFiles.WriteOCFile}\pysiglinewithargsret{\strong{class }\code{writeInfoFiles.}\bfcode{WriteOCFile}}{\emph{InfosFromGUIObject}}{}
Writes the OperCond.inp file using the output of the GUI.

\end{fulllineitems}



\chapter{The Main Class}
\label{MainProgramCode:the-main-class}\label{MainProgramCode::doc}
This is the class controlling the whole process. It initialize the other classes and use their methods. If accesed as main file, it launches itself the whole process. But it is also connetced by the GUI, there the GUI tells to apply the MainProcess' methods.
\index{MainProcess (class in Pyrolysis)}

\begin{fulllineitems}
\phantomsection\label{MainProgramCode:Pyrolysis.MainProcess}\pysigline{\strong{class }\code{Pyrolysis.}\bfcode{MainProcess}}
Controls the whole process of generating input files, fitting etc.
\index{CheckFGdt() (Pyrolysis.MainProcess method)}

\begin{fulllineitems}
\phantomsection\label{MainProgramCode:Pyrolysis.MainProcess.CheckFGdt}\pysiglinewithargsret{\bfcode{CheckFGdt}}{}{}
Aborts, if FG-DVC is selected and the timestep is lower than 1.e-3 (which is FG-DVC not able to read):

\end{fulllineitems}

\index{DAF() (Pyrolysis.MainProcess method)}

\begin{fulllineitems}
\phantomsection\label{MainProgramCode:Pyrolysis.MainProcess.DAF}\pysiglinewithargsret{\bfcode{DAF}}{\emph{PAFC\_asRecieved}, \emph{PAVM\_asRecieved}}{}
calculates PAFC, PAVM  from the as recieved state to the daf state of coal

\end{fulllineitems}

\index{MakeResults\_Arrh() (Pyrolysis.MainProcess method)}

\begin{fulllineitems}
\phantomsection\label{MainProgramCode:Pyrolysis.MainProcess.MakeResults_Arrh}\pysiglinewithargsret{\bfcode{MakeResults\_Arrh}}{\emph{PyrolProgram}, \emph{File}, \emph{Fit}}{}
Generates the results for Arrhenius Rate.

\end{fulllineitems}

\index{MakeResults\_ArrhNoB() (Pyrolysis.MainProcess method)}

\begin{fulllineitems}
\phantomsection\label{MainProgramCode:Pyrolysis.MainProcess.MakeResults_ArrhNoB}\pysiglinewithargsret{\bfcode{MakeResults\_ArrhNoB}}{\emph{PyrolProgram}, \emph{File}, \emph{Fit}}{}
Generates the results for Arrhenius Rate with no correction term T**b.

\end{fulllineitems}

\index{MakeResults\_CPD() (Pyrolysis.MainProcess method)}

\begin{fulllineitems}
\phantomsection\label{MainProgramCode:Pyrolysis.MainProcess.MakeResults_CPD}\pysiglinewithargsret{\bfcode{MakeResults\_CPD}}{}{}
generates the result for CPD

\end{fulllineitems}

\index{MakeResults\_CR() (Pyrolysis.MainProcess method)}

\begin{fulllineitems}
\phantomsection\label{MainProgramCode:Pyrolysis.MainProcess.MakeResults_CR}\pysiglinewithargsret{\bfcode{MakeResults\_CR}}{\emph{PyrolProgram}, \emph{File}, \emph{Fit}}{}
Generates the results for constant Rate.

\end{fulllineitems}

\index{MakeResults\_DEAM() (Pyrolysis.MainProcess method)}

\begin{fulllineitems}
\phantomsection\label{MainProgramCode:Pyrolysis.MainProcess.MakeResults_DEAM}\pysiglinewithargsret{\bfcode{MakeResults\_DEAM}}{\emph{PyrolProgram}, \emph{File}, \emph{Fit}}{}
Generates the results for DAEM model.

\end{fulllineitems}

\index{MakeResults\_FG() (Pyrolysis.MainProcess method)}

\begin{fulllineitems}
\phantomsection\label{MainProgramCode:Pyrolysis.MainProcess.MakeResults_FG}\pysiglinewithargsret{\bfcode{MakeResults\_FG}}{}{}
generates the result for FG-DVC

\end{fulllineitems}

\index{MakeResults\_Kob() (Pyrolysis.MainProcess method)}

\begin{fulllineitems}
\phantomsection\label{MainProgramCode:Pyrolysis.MainProcess.MakeResults_Kob}\pysiglinewithargsret{\bfcode{MakeResults\_Kob}}{\emph{PyrolProgram}, \emph{File}, \emph{Fit}}{}
Generates the results for Kobayashi Rate.

\end{fulllineitems}

\index{ReadInputFiles() (Pyrolysis.MainProcess method)}

\begin{fulllineitems}
\phantomsection\label{MainProgramCode:Pyrolysis.MainProcess.ReadInputFiles}\pysiglinewithargsret{\bfcode{ReadInputFiles}}{}{}
get parameters from input files

\end{fulllineitems}

\index{SpeciesEnergy() (Pyrolysis.MainProcess method)}

\begin{fulllineitems}
\phantomsection\label{MainProgramCode:Pyrolysis.MainProcess.SpeciesEnergy}\pysiglinewithargsret{\bfcode{SpeciesEnergy}}{\emph{PyrolProgram}, \emph{File}}{}
Carries out the species and Energy balance.

\end{fulllineitems}


\end{fulllineitems}




\renewcommand{\indexname}{Index}
\printindex
\end{document}
